
\section{Format}

Boxes


horizontal box 

\textbf{Vertical Box }

\textbf{Vertical/Horizontal Box} f it is possible, the entire box is written on
a single line; otherwise, every break hint within the box leads to a
new line.  
Example

If there is enough room to print the box on the line:
   --b--b--
But ``---b---b---'' that cannot fit on the line is written
   ---b
   ---b
   ---

\textbf{Vertical Or Horizontal Box}
If there is enough room to print the box on the line:
   --b--b--
But if ``---b---b---'' cannot fit on the line, it is written as
   ---b---b
   ---
The first break hint does not lead to a new line, since there is
enough room on the line. The second one leads to a new line since
there is no more room to print the material following it. If the room
left on the line were even shorter, the first break hint may lead to a
new line and ``---b---b---'' is written as:
    ---b
    ---b
    ---



Indentation Rules

The user gets 2 ways to fix the Indentation of the newlines: 
a. Defining the box: when you open a box, you can fix the
indentation added to each new line opened within that box.
For instance: open_hovbox 1 opens a “hov” box with new lines indented
1 more than the initial indentation of the box. With output
``---[--b--b--b--'', we get:
   ---[--b--b
        --b--
with open_hovbox 2, we get
   ---[--b--b
         --b--



b. Defining the break that makes the new line. As said above, you
output a break hint using print_break sp indent. The indent integer is
used to fix the additional indentation of the new line. Namely, it is
added to the default indentation offset of the box where the break
occurs.
For instance, if [ stands for the opening of a “hov” box with 1 as
  extra indentation (as obtained by open_hovbox 1), and b is
  print_break 1 2, then from output ``---[--b--b--b--'', we get:
   ---[-- --
         --
         --
 
Advice

Never hestitate to open a box 

Output many break hints 



``\@\[``
 open a box (open_box 0). You may precise the type as an extra
  argument. For instance @[<hov n> is equivalent to open_hovbox n.


``\@\]'' close a box (close_box ()).


``\@ `` output a breakable space (print_space ()).
``@,'' output a break hint (print_cut ()).
``\@;<n m>'' emit a “full” break hint (print_break n m).
``\@.'' end the pretty-printing, closing all the boxes still opened
  (print_newline ()).flush 


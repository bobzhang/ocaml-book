\begin{quotation}

  This is a book about advanced programming in OCaml.  It's assumed
  that you are already familiar with the basic ideas of the different
  programming paradigms, for example, functional programming,
  object-oriented programming, macro programming.

  I am a graduate student of \textit{University of Pennsylvania} right
  now, I love programming and spend a lot of time digging different
  interesting languages and the underlying theories, including
  \textit{C,C++,Ocaml,Haskell,Common Lisp, Perl} and know tiny bits
  about \textit{coq,prolog,python} as well.

  \textit{Haskell} is the most elegant language that I ever used and
  is the language that brought me to the Alice's wonderland of
  function programming, while \textit{Ocaml} is the most productive
  one for me at this time, since its support of multiple paradigms and
  practicalism. Haskell's \textit{lazy evaluation} renders it
  impractical for real world programming. I love Haskell's all good
  parts except \textit{lazy evaluation}. Even for meta-programming,
  Haskell's nice support for generic programming makes it less painful
  compared with programming in \textit{camlp4}, \textit{camlp4}
  supports quasiquotation mechanism fully, however.

  Another interesting language is \textit{Common Lisp}, the most
  interesting book that I read about programming is \textit{Paradigms
    of Artificial Intelligence Programming: Case Studies in Common
    Lisp} which was written in \textit{Common Lisp}. Some other
  interesting books about \textit{Lisp} are \textit{On Lisp} and
  \textit{Let Over Lambda}. \textit{Clojure} is a modern dialect of
  \textit{Common Lisp}, while its implementation is a bit weak at this
  time. Lisp's good part mainly lies in \textit{meta-programming}.
  However, from mine point of view, type safety is a really good tool
  to maintain the quality of your pieces of software. And the good
  parts of \textit{Lisp} are largely absorbed by \textit{Haskell,
    OCaml}, it's not too hard to do meta programming in Ocaml, while
  sometimes  even more convenient.

  Feel free to contact me if you find it
  interesting. \href{mailto:hongboz@seas.upenn.edu}{hongboz@seas.upenn.edu}
  
  To conclude, \textit{OCaml} is the most productive language at this
  time, and that's why I am writing this book.

  
  This is still a book in progress. Don't distribute it without any
  permission.
\end{quotation}

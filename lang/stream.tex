

\href{http://mirror.ocamlcore.org/ocaml-tutorial.org/streams.html}{streams}

\begin{enumerate}
\item stream expression

  \begin{redcode}

let rec walk dir = 
    let items = try 
      Array.map (fun fn -> let path = Filename.concat dir fn in 
             try if Sys.is_directory path then `Dir path else `File path
             with e -> `Error(path,e) ) (Sys.readdir dir)
      with e -> [| `Error (dir,e) |] in 
      Array.fold_right 
        (fun item rest -> match item with 
            |`Dir path -> [< 'item ; walk path; rest >]
            | _ -> [< 'item; rest >]) items [< >];;


(** alternative without syntax extension *)
let rec walk dir =
  let items =
    try
      Array.map
        (fun fn ->
           let path = Filename.concat dir fn
           in
             try if Sys.is_directory path then `Dir path else `File path
             with | e -> `Error (path, e))
        (Sys.readdir dir)
    with | e -> [| `Error (dir, e) |]
  in
    Array.fold_right
      (fun item rest ->
         match item with
         | `Dir path ->
             Stream.icons item (Stream.lapp (fun _ -> walk path) rest)
         | _ -> Stream.icons item rest)
      items Stream.sempty


            
Stream.(walk "/Users/bob" |> take 10 |> iter 
s      ((function `Dir s -> "dir :" ^ s | `File s -> "file: " ^ s | `Error (s,e) -> "error: " ^ s ^ " " ^ Printexc.to_string e) |- print_string |- print_newline) );;
            
   \end{redcode}
   
   \begin{bluecode}
- : string ->
    [> `Dir of string | `Error of string * exn | `File of string ]
    Batteries.Stream.t

error: /Users/bob/.#.log Sys_error("/Users/bob/.#.log: No such file or directory")
file: /Users/bob/.aboutenvfiles
file: /Users/bob/.bash_history
file: /Users/bob/.bashrc
file: /Users/bob/.bashrc~
dir :/Users/bob/.cabal
file: /Users/bob/.cabal/.DS_Store
dir :/Users/bob/.cabal/bin
file: /Users/bob/.cabal/bin/alex
file: /Users/bob/.cabal/bin/bf

    
 \end{bluecode}
\item module Stream


\begin{alternate}
Stream.npeek;;
- : int -> 'a Batteries.Stream.t -> 'a list = <fun>
Stream.next;;
- : 'a Stream.t -> 'a = <fun>
\end{alternate}



\begin{redcode}
let lines_stream_of_channel chan = Stream.from (fun _ -> 
    try Some (input_line chan) with End_of_file -> None );;
\end{redcode}

\begin{bluecode}  
val lines_stream_of_channel : BatIO.input -> string Batteries.Stream.t =
\end{bluecode}


it raises \textit{Stream.Failure} on an empty stream,
i.e. \textit{Stream.next}

\begin{redcode}
let line_stream_of_string string =
  Stream.of_list (Str.(split (regexp "\n") string))
\end{redcode}

\item Constructing streams \\
  \begin{bluetext}
    Stream.from
    Stream.of_list
    Stream.of_string (* char t *)
    Stream.of_channel (* char t *)
  \end{bluetext}

\item Consuming streams \\

\begin{bluetext}
   Stream.peek
   Stream.junk
\end{bluetext}

\begin{bluecode}
let paragraph lines =
  let rec next para_lines i =
    match Stream.peek lines,para_lines with
    | None, [] -> None
    | Some "", [] ->
      Stream.junk lines (* still a white paragraph *)
      next para_lines i
    | Some "", _ | None, _ ->
      Some (String.concat "\n" (List.rev para_lines)) (* a new paragraph*)
    | Some line, _ ->
      Stream.junk lines ;
      next (line :: para_line ) i in
  Stream.from (next [])    
\end{bluecode}

\begin{redcode}
let stream_fold f stream init = 
    let result = ref init in 
    Stream.iter (fun x -> result := f x !result) stre  am; !result;;
  \end{redcode}

\begin{bluecode}  
val stream_fold : ('a -> 'b -> 'b) -> 'a Batteries.Stream.t -> 'b -> 'b =
  <fun>
\end{bluecode}

\begin{redcode}
let stream_concat streams = 
  let current_stream = ref None in 
  let rec next i = 
    try 
      let stream = match !current_stream with 
        | Some stream -> stream 
        | None -> 
          let stream = Stream.next streams in 
          current_stream := Some stream ; 
          stream in 
      try Some (Stream.next stream)
      with Stream.Failure -> (current_stream := None ; next i)
    with Stream.Failure -> None in 
  Stream.from next
\end{redcode}

\item \textit{copying or sharing} streams \\
  this was called \textit{dup} in Enum
  \begin{bluecode}
(** create 2 buffers to store some pre-fetched value *)
let stream_tee stream = 
  let next self other i = 
    try 
      if Queue.is_empty self 
      then 
        let value = Stream.next stream in 
        Queue.add value other ;
        Some value
      else 
        Some (Queue.take self)
    with Stream.Failure -> None in 
  let q1,q2 = Queue.create (), Queue.create () in 
  (Stream.from (next q1 q2), Stream.from (next q2 q1))
\end{bluecode}

\item convert arbitray data types to streams \\
  if the datat type defines an \textit{iter} function, and you don't
  mind using threads, you can use a \textit{producer-consumer}
  arrangement to invert control.

\begin{redcode}
let elements iter coll = 
  let channel = Event.new_channel () in 
  let producer () = 
    let _ = iter (fun x -> Event.(sync (send channel (Some x )))) coll in 
    Event.(sync (send channel None)) in 
  let consumer i = 
    Event.(sync (receive channel)) in 
  ignore (Thread.create producer ()) ; 
  Stream.from consumer    
\end{redcode}

  \begin{bluecode}
val elements : (('a -> unit) -> 'b -> 'c) -> 'b -> 'a Batteries.Stream.t =    
\end{bluecode}

Keep in mind that these techniques spawn producer threads which carry
a few risks: they only terminate when they have finished iterating,
and any change to the original data structure while iterating may
produce unexpected results.


\end{enumerate}
%%% Local Variables: 
%%% mode: latex
%%% TeX-master: "../master"
%%% End: 

Continuatioins
A conditional banch selects a continuation from the two possible
futures; rasing an exception discards. Traditional way to handle
continuations explicitly in a program is to transform a program into
cps style. Continuation captured by call/cc is the {\bf whole} continuation
that includes all the future computation.. In practice, most of the
continuations that we want to manipulate are only a part of
computation. Such continuations are called {\bf delimited continuations} or
{\bf partial continuations}.


\begin{enumerate}
\item cps transform \\
  there are multiple ways to do cps transform, here are two.

  
  \begin{bluetext}
------------------------------------------
   [x] --> x
   [\x. M] --> \k . k (\x . [M])
   [M N] --> \k. [M] (\m . m [N] k)
------------------------------------------


------------------------------------------
   [x] --> \k . k x
   [\x. M] --> \k. k (\x.[M])
   [M N] --> \k. [M] (\m . [N] (\n. m n k))
------------------------------------------


[callcc (\k. body)] = \outk. (\k. [body] outk) (\v localk. outk v)
   
  \end{bluetext}

  
\item experiment

\begin{alternate}
#load "delimcc.cma";;
\end{alternate}
\begin{alternate}
Delimcc.shift;;
- : 'a Delimcc.prompt -> (('b -> 'a) -> 'a) -> 'b = <fun>
\end{alternate}

\begin{bluetext}
reset (fun () -> M ) --> push_prompt p (fun () -> M )
shift (fun k -> M) --> shift p (fun k -> M )
\end{bluetext}
in racket you should have \textit{(require racket/control)}
and then \textit{(reset expr ...+)}
\textit{(shift id expr ...+)}


\begin{redcode}
module D = Delimcc
(** set the prompt *)  
let p = D.new_prompt ()
let (reset,shift),abort  = D.(push_prompt &&& shift &&& abort ) p;;
let foo x = reset (fun () -> shift (fun cont -> if x = 1 then cont 10 else 20 ) + 100 )
\end{redcode}

\begin{alternate}
foo 1 ;;
- : int = 110
foo 2  ;;
- : int = 20
5 * reset (fun () -> shift (fun k -> 2 * 3 ) + 3 * 4 );;
- : int = 30
reset (fun () -> 3 + shift (fun k -> 5 * 2) ) - 1 ;;
- : int = 9
\end{alternate}
\begin{bluetext}
val p : '_a D.prompt = <abstr>
val reset : (unit -> '_a) -> '_a = <fun>
val shift : (('_a -> '_b) -> '_b) -> '_a = <fun>
val abort : '_a -> 'b = <fun> 
\end{bluetext}

\begin{bluecode}
let p = D.new_prompt ()
let (reset,shift),abort  = D.(push_prompt &&& shift &&& abort ) p;;
\end{bluecode}

\begin{alternate}
reset (fun () -> if (shift (fun k -> k(2 = 3))) then "hello" else "hi ") ^ "world";;
- : string = "hi world"
reset (fun () -> if (shift (fun k -> "laji")) then "hello" else "hi ") ^ "world";;
- : string = "lajiworld"
reset (fun _ -> "hah");;
- : string = "hah"
\end{alternate}


\begin{bluecode}
let make_operator () =  
  let p = D.new_prompt () in 
  let (reset,shift),abort = D.(push_prompt &&& shift &&& abort) p in 
  p,reset,shift,abort
\end{bluecode}

Delimited continuations seems not able to handle answer type polymorphism.

\begin{bluetext}
exception Str of [`Found of int | `NotFound]  
\end{bluetext}

\begin{redcode}
let times lst  = 
  let rec times_aux lst = match lst with 
    | [] -> 1 
    | 0 :: xs -> shift (fun _ -> 0 )
    | x :: xs -> begin 
      (* printf "entering %d\n" x ; *)
      let v = x * times_aux xs in 
      (* printf "exiting %d\n" x ;  *)
      v
    end in 
  reset (fun () -> times_aux lst )
\end{redcode}

Store the continuation, the type system is not friendly to the
continutations, but fortunately we have \textit{side effects} at hand, we can
store it. (This is pretty hard in Haskell )

\begin{redcode}
let p,reset,shift,abort = make_operator() in 
  let c = ref None in 
  begin 
   reset (fun () -> 3 + shift (fun k -> c:= Some k ;  0) - 1)  ; 
   Option.get (!c) 20 
   end ;;
          Characters 81-139:
     reset (fun () -> 3 + shift (fun k -> c:= Some k ;  0) - 1)  ; 
     ^^^^^^^^^^^^^^^^^^^^^^^^^^^^^^^^^^^^^^^^^^^^^^^^^^^^^^^^^^
     Warning 10: this expression should have type unit.
   \end{redcode}
\begin{bluecode}   
- : int = 22
\end{bluecode}
\begin{redcode}
let cont = 
  let p,reset,shift,abort = make_operator() in 
  let c = ref None in 
  let rec id lst = match lst with 
    | [] -> shift (fun k -> c:=Some k ; [] )
    |x :: xs -> x :: id xs in 
  let xs = reset (fun () -> id [1;2;3;4]) in 
  xs, Option.get (!c);;
\end{redcode}
\begin{bluecode}
val cont : int list * (int list -> int list) = ([], <fun>)
\end{bluecode}
\begin{alternate}
# let a,b = cont ;;
val a : int list = []

val b : int list -> int list = <fun>
# b [];;
- : int list = [1; 2; 3; 4]
\end{alternate}



\begin{redcode}
type tree = Empty | Node of  tree * int  * tree 
let walk_tree = 
  let cont = ref None in 
  let p,reset,shift,abort = make_operator() in 
  let yield n = shift (fun k -> cont := Some k; print_int n ) in 
  let rec walk2 tree = match tree with 
    |Empty -> ()
    |Node (l,v,r) -> 
      walk2 l ;
      yield v ; 
      walk2 r in 
  fun tree -> (reset (fun _ -> walk2 tree ), cont);;
\end{redcode}
\begin{bluecode}
val walk_tree : tree_t -> unit * ('_a -> unit) option Batteries.ref =
\end{bluecode}

\begin{alternate}
# let _, cont = walk_tree tree1 ;;
1val cont : ('_a -> unit) option Batteries.ref = {contents = Some <fun>}
# Option.get !cont ();;
2- : unit = ()
# Option.get !cont ();;
3- : unit = ()
# Option.get !cont ();;
- : unit = ()
# Option.get !cont ();;
- : unit = ()
\end{alternate}

It's quite straightforward to implement yield using delimited
continuation, since each time shifting will escape the control, and you store the continuation, later it can be resumed.


\begin{bluetext}
(** defer the continuation *)  
shift (fun k -> fun () -> k "hello")
\end{bluetext}

By wrapping continuations, we can \textbf{access the information outside} of the enclosing
reset while staying within reset lexically.

suppose this type check

\begin{alternate}
  let f x = reset (fun () -> shift (fun k -> fun () -> k "hello") ^ "world" ) x
  f : unit -> string 
\end{alternate}

\item Answer type modification (serious)
  in the following context,
  \verb|reset (fun () -> [...] ^  "word" )|, the value returned by
  reset appears to be a string. An answer type is a type of the enclosing
  \emph{reset}.

\item reorder delimited continuations \\
  if we apply a continuation at the tail position, the captured computation is simply
  resumed. If we apply a continuation at the non-tail position, we can perform
  additional computation after resumed computation finishes.

  Put differently, we can switch the execution order of the surrounding context.

\begin{redcode}
let p,reset,shift,abort = make_operator () in
    reset (fun () -> 1 + (shift (fun k -> 2 * k 3 )));;
\end{redcode}
\begin{bluecode}
- : int = 8    
\end{bluecode}

\begin{redcode}
let p,reset,shift,abort = make_operator () in 
   let either a b = shift (fun k -> k a ; k b ) in 
   reset (fun () -> 
   let x = either  0  1 in 
   print_int x ; print_newline ());;
 \end{redcode}
\begin{bluecode} 
 0
 1  
\end{bluecode}
\item useful links \\
  \href{http://blog.fitzell.ca/2009/01/seaside-partial-continuations.html}{sea
    side} \\
  \href{http://pllab.is.ocha.ac.jp/~asai/cw2011tutorial/}{shift and
    reset tutorial} \\
  \href{http://pllab.is.ocha.ac.jp/~asai/cw2011tutorial/main-e.pdf}{shift
    reset tutorial} \\
  \href{http://docs.racket-lang.org/reference/cont.html#(part._.Classical_.Control_.Operators)}{racket
    control operators} \\
  \href{http://okmij.org/ftp/continuations/caml-shift.pdf}{caml-shift-paper.pdf} \\
  \href{http://okmij.org/ftp/continuations/caml-shift-talk.pdf}{caml-shift-talk} \\
\end{enumerate}
%%% Local Variables: 
%%% mode: latex
%%% TeX-master: "../master"
%%% End: 

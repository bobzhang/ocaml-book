\section{Camlp4 SourceCode Exploration}
Now we begin to explore the structure of camlp4 Source Code.  First
let's have a look at the directory structure of \verb|camlp4|
directory.


\begin{bashcode}
.
|-- boot
|-- build
|-- Camlp4
|   |-- Printers
|   `-- Struct      
|       `-- Grammar
|-- Camlp4Filters
|-- Camlp4Parsers
|-- Camlp4Printers
|-- Camlp4Top
|-- examples
|-- man
|-- test
|   `-- fixtures
`-- unmaintained   
\end{bashcode}


\subsection{Camlp4 PreCast}
\verb|Camlp4.PreCast (Camlp4/PreCast.ml)|

As above,Struct directory has module \textit{Loc, Dynloader Functor,
  Camlp4Ast.Make, Token.Make, Lexer.Make, Grammar.Static.Make,
  Quotation.Make}

File \verb|Camlp4.PreCast| Listing \ref{lst:Camlp4 PreCast}
\textbf{Re-Export} such files

\begin{bluetext}
  Struct.Loc Struct.Camlp4Ast Struct.Token Struct.Grammar.Parser
  Struct.Grammar.Static Struct.Lexer Struct.DynLoader Struct.Quotation
  Struct.AstFilters OCamlInitSyntax Printers.OCaml Printers.OCamlr
  Printers.Null Printers.DumpCamlp4Ast Printers.DumpOCamlAst
\end{bluetext}


\inputminted[fontsize=\scriptsize, 
lastline=55]{ocaml}{code/camlp4/source/precast.ml}
\captionof{listing}{Camlp4 PreCast \label{lst:Camp4 PreCast}}



\subsection{OCamlInitSyntax}

\verb|OCamlInitSyntax| Listing \ref{lst:OCamlInitSyntax} does not do
too many things, first, it initialize all the entries needed later
(they are all blank, to be extended by your functor), after
initialization, it created a submodule \verb|AntiquotSyntax|, and
initialize two entries \verb|antiquot_expr| and \verb|antiquot_patt|,
very easy.

\inputminted[fontsize=\scriptsize,
firstline=55]{ocaml}{code/camlp4/source/precast.ml}
\captionof{listing}{OCamlInitSyntax \label{lst:OCamlInitSyntax}}



\subsection{Camlp4.Sig}
For \verb|Camlp4.Sig.ml|, all are signatures, there's even no
\verb|Camlp4.Sig.mli|.



\subsection{Camlp4.Struct.Camlp4Ast.mlast} 

This file use macro\verb|INCLUDE| to include
\verb|Camlp4.Camlp4Ast.parital.ml| for reuse.


\subsection{AstFilters}    
Notice an interesting module \verb|AstFilters| Listing
\ref{lst:AstFilters}, is defined by \verb|Struct.AstFilters.Make|,
which we see in \verb|Camlp4.PreCast.ml|\ref{lst:Camlp4 PreCast} It's
very simple actually.


\inputminted[fontsize=\scriptsize,]{ocaml}{code/camlp4/source/AstFilters.ml}
\captionof{listing}{AstFilters \label{lst:AstFilters}}



\begin{ocamlcode}
(** file Camlp4Ast.mlast   in the file we have *)
Camlp4.Struct.Camlp4Ast.Make : Loc -> Sig.Camlp4Syntax
  module Ast = struct
     include Sig.MakeCamlp4Ast Loc 
  end ;
\end{ocamlcode}
\captionof{listing}{Camlp4Ast Make\label{lst:Camlp4Ast Make}}


\subsection{Camlp4.Register}
Let's see what's in \verb|Register| Listing \ref{lst:Camlp4
  Register}module


\inputminted[fontsize=\scriptsize,
]{ocaml}{code/camlp4/source/Register.ml}
\captionof{listing}{Camlp4 Register\label{lst:Camlp4 Register}}

Notice that functors Plugin, SyntaxExtension, OCamlSyntaxExtension,
OCamlSyntaxExtension, SyntaxPlugin, they did the same thing
essentially, they apply the second Funtor to
Syntax(Camlp4.PreCast.Syntax).

Functors Printer, OCamlPrinter, OCamlPrinter, they did the same thing,
apply the Make to Syntax, then register it. 

Functors Parser, OCamlParser, did the same thing. 

Functors AstFilter  did nothing interesting.

It sticks to the toplevel Listing

\inputminted[fontsize=\scriptsize, firstline=123, lastline=126,
]{ocaml}{code/camp4/source/Register.ml}
\captionof{listing}{Camlp4 Register Part 2 \label{lst:Camlp4 Register
    Part 2}}

It mainly hook some global variables, like
\verb|Camlp4.Register.loaded_modlules|, but there's no fresh meat in
this file.  To conclude, Register did nothing, except making your code
more modular, or register your syntax extension.

As we said, another utility, you can inspect what modules you have
loaded in toplevel:

\begin{ocamlcode}
Camlp4.Register.loaded_modules;;
- : string list ref =
{Pervasives.contents =
  ["Camlp4GrammarParser"; "Camlp4OCamlParserParser";
   "Camlp4OCamlRevisedParserParser"; "Camlp4OCamlParser";
   "Camlp4OCamlRevisedParser"]}
\end{ocamlcode}


\subsection{Camlp4Ast}

As the code Listing \ref{lst:Camlp4 Ast Definition} demonstrate below,
there are several categories including \textit{ident, ctyp,patt,expr,
  module\_type, sig\_item, with\_constr, binding, rec\_binding,
  module\_binding, match\_case, module\_expr,str\_item, class\_type,
  class\_sig\_item, class\_expr, class\_str\_item,}. And there are
antiquotations for each syntax category, i.e,
\textit{IdAnt,TyAnt,PaAnt,ExAnt,MtAnt, SgAnt, WcAnt, BiAnt,
  RbAnt,MbAnt,McAnt,MeAnt,StAnt,CtAnt,CgAnt, CeAnt, CrAnt}


\inputminted[fontsize=\scriptsize,
]{ocaml}{camlp4/code/ast/ast_def.ml}
\captionof{listing}{Camlp4 Ast Definition\label{lst:Camlp4 Ast Definition}}


\subsection{TestFile}
\label{sec:testfile}
Some test files are pretty useful(in the distribution of camlp4)
like \verb|test/fixtures/macro_test.ml|.

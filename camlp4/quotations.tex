\section{QuasiQuotations}

There are a lot of metaphors introducing quasiquotations in camlp4
wiki, here we did not bother it. We just told you how to program.

Quotations are delimeters, since there are so many syntaxes in ocaml,
so you need to delimit your syntax to make all up. Antiquotations are
those make your syntax interact with outside scope. They are all
expanders will be expanded by your hook.

\begin{ocamlcode}
[`QUOTATION x -> Quotation.expand _loc x Quotation.DynAst.expr_tag ]    
\end{ocamlcode}

The `QUOTATION token contains a record including the body of the
quotation and the \verb|tag|. The record is passed off to the
Quotation module to be expanded. The expander parses the quotation
string starting at some non-terminal(you specified), then runs the
result through the antiquotation expander

\begin{ocamlcode}
  |`ANTIQUOT (``exp'' | ``'' | ``anti'' as n) s ->
  <:expr< $anti:make_anti ~c:"expr" n s $>>
\end{ocamlcode}

The antiquotation creates \verb|a special AST node| to hold the body of the
antiquotation, each type in the AST has a constructor (\verb|ExAnt, TyAnt|,
etc.) \verb|c|  means context here.
Here we grep \verb|Ant|, and the output is as follows

\begin{bluetext}
  27 matches for "Ant" in buffer: Camlp4Ast.partial.ml
      5:    | BAnt of string ]
      9:    | ReAnt of string ]
     13:    | DiAnt of string ]
     17:    | MuAnt of string ]
     21:    | PrAnt of string ]
     25:    | ViAnt of string ]
     29:    | OvAnt of string ]
     33:    | RvAnt of string ]
     37:    | OAnt of string ]
     41:    | LAnt of string ]
     47:    | IdAnt of loc and string (* $s$ *) ]
     87:    | TyAnt of loc and string (* $s$ *)
     93:    | PaAnt of loc and string (* $s$ *)
    124:    | ExAnt of loc and string (* $s$ *)
    202:    | MtAnt of loc and string (* $s$ *) ]
    231:    | SgAnt of loc and string (* $s$ *) ]
    244:    | WcAnt of loc and string (* $s$ *) ]
    251:    | BiAnt of loc and string (* $s$ *) ]
    258:    | RbAnt of loc and string (* $s$ *) ]
    267:    | MbAnt of loc and string (* $s$ *) ]
    274:    | McAnt of loc and string (* $s$ *) ]
    290:    | MeAnt of loc and string (* $s$ *) ]
    321:    | StAnt of loc and string (* $s$ *) ]
    337:    | CtAnt of loc and string ]
    352:    | CgAnt of loc and string (* $s$ *) ]
    372:    | CeAnt of loc and string ]
    391:    | CrAnt of loc and string (* $s$ *) ];
\end{bluetext}

Take this as a simple example

\begin{bluetext}
camlp4rf -str 'value f = fun [ <:expr<$int:i$ >> -> i ];'
\end{bluetext}
It will be expanded like this

\begin{ocamlcode}
let f = function | Ast.ExInt (_, i) -> i
\end{ocamlcode}
\begin{bluetext}
camlp4rf -str 'value f = fun [ <:expr<$`int:i$ >> -> i ];'
\end{bluetext}
And the output is like this

\begin{ocamlcode}
let f = function | Ast.ExInt (_, i) -> i
\end{ocamlcode}
The underscore is location, so you see, the expander just make you
happy, you can write programs directly if you don't like it.

\begin{ocamlcode}
<:expr< $int: "4"$ >>;;
- : Camlp4.PreCast.Ast.expr = Camlp4.PreCast.Ast.ExInt (<abstr>, "4")
<:expr< $`int: 4$ >>;; (** the same result *)
- : Camlp4.PreCast.Ast.expr = Camlp4.PreCast.Ast.ExInt (<abstr>, "4")
<:expr< $`flo:4.1323243232$ >>;;
- : Camlp4.PreCast.Ast.expr = Camlp4.PreCast.Ast.ExFlo (<abstr>, "4.1323243232")
# <:expr< $flo:"4.1323243232"$ >>;;
- : Camlp4.PreCast.Ast.expr = Camlp4.PreCast.Ast.ExFlo (<abstr>, "4.1323243232")
(** maybe the same for flo *)
\end{ocamlcode}

\subsubsection{Quotation Expander}
\begin{ocamlcode}
    match_case:
      [ [ "["; l = LIST0 match_case0 SEP "|"; "]" -> Ast.mcOr_of_list l
        | p = ipatt; "->"; e = expr -> <:match_case< $p$ -> $e$ >> ] ]
    ;
    match_case0:
      [ [ `ANTIQUOT ("match_case"|"list" as n) s ->
            <:match_case< $anti:mk_anti ~c:"match_case" n s$ >>
        | `ANTIQUOT (""|"anti" as n) s ->
            <:match_case< $anti:mk_anti ~c:"match_case" n s$ >>
        | `ANTIQUOT (""|"anti" as n) s; "->"; e = expr ->
            <:match_case< $anti:mk_anti ~c:"patt" n s$ -> $e$ >>
        | `ANTIQUOT (""|"anti" as n) s; "when"; w = expr; "->"; e = expr ->
            <:match_case< $anti:mk_anti ~c:"patt" n s$ when $w$ -> $e$ >>
        | p = patt_as_patt_opt; w = opt_when_expr; "->"; e = expr -> <:match_case< $p$ when $w$ -> $e$ >>
      ] ]
    \end{ocamlcode}
The grammar rules indicate that the rule accept antiquotations here,
and you can do insertion.
Now let's take a furthur look at the antiquot expander

\inputminted[fontsize=\scriptsize,linenos=true,stepnumber=5]
{ocaml}{code/camlp4/antiquot_expander/antiquot_expander.ml}
\captionof{listing}{Quotation Expander\label{Quotation Expander}}

In line 40, we define \textit{antiquote\_expander} which was inherited
from \textit{Ast.map} with overriding method \textit{patt} and
\textit{expr}. Take a furthur look at method patt, we see it handles
two kinds of constructor \textit{(Ast.PaAnt \_loc s | Ast.PaStr \_loc
  s as p)} From line 125, we just \textit{add\_quotations} to the
quotation expander.

Now we want to manipulate lambda ast, we just translate the
 concrete syntax to the syntax we want.

\subsubsection{Lambda Example}
\inputminted[fontsize=\scriptsize,linenos=true,stepnumber=5]
{ocaml}{code/camlp4/lambda/first.ml}
\captionof{listing}{Lambda Extension\label{Lambda Extension}}

Listing\ref{Lambda Example} try to use built in caml syntax to save
you time.  We use antiquotation syntax, that is, going back to host
language.  You use \textit{Syntax.AntiquotSyntax.parse\_expr} and
\textit{Syntax.AntiquotSyntax.parse\_patt} here since it's
\textit{reflexitive}, it will be the same as your host language
syntax, not fixed to original syntax or revised syntax.'
Another thing to notice, here we transform string input \textit{directly} into
camlp4 AST, normally you may transform your string input into your
ast, and then transform your ast into camlp4 Ast using filter.
It's very easy to test our modules, just using toplevel
\begin{ocamlcode}
  #directory ``_build'';;
  #load ``first.cmo'';;
  let x = <:lam< \y -> y y>>;;
val x :
  [> `Lam of
       [> `Var of string ] *
       [> `App of [> `Var of string ] * [> `Var of string ] ] ] =
  `Lam (`Var "y", `App (`Var "y", `Var "y"))  
\end{ocamlcode}
Here we notice that there are a lot of tags Listing \ref{DynAst
  Tag}. Tags mean the position where you can expand your code. 
\begin{ocamlcode}
    type 'a tag = 'a Camlp4.PreCast.Quotation.DynAst.tag
    val ctyp_tag : Ast.ctyp tag
    val patt_tag : Ast.patt tag
    val expr_tag : Ast.expr tag
    val module_type_tag : Ast.module_type tag
    val sig_item_tag : Ast.sig_item tag
    val with_constr_tag : Ast.with_constr tag
    val module_expr_tag : Ast.module_expr tag
    val str_item_tag : Ast.str_item tag
    val class_type_tag : Ast.class_type tag
    val class_sig_item_tag : Ast.class_sig_item tag
    val class_expr_tag : Ast.class_expr tag
    val class_str_item_tag : Ast.class_str_item tag
    val match_case_tag : Ast.match_case tag
    val ident_tag : Ast.ident tag
    val binding_tag : Ast.binding tag
    val rec_binding_tag : Ast.rec_binding tag
    val module_binding_tag : Ast.module_binding tag
  \end{ocamlcode}
  \captionof{listing}{DynAst Tag\label{DynAst Tag}}




  \subsection{Ast Transformation}
  
Using concrete syntax instead of the abstract one is useful for 
1. Independent of the internal representation 
2. Have more concise programs
The beauty lies that you can \textit{compose} bigger AST using smaller ones
without learning about the internal AST constructor.

\inputminted[fontsize=\scriptsize,]{ocaml}{code/camlp4/compose/compose.ml}
\captionof{listing}{Compose Bigger Ast \label{Composing Bigger Ast}}

Notice that Ast.Filter does not work well in the toplevel, Quotations
work well in the toplevel. Revised syntax is more robust handling
ambiguity:

\begin{enumerate}
  \item  Tuples \verb|(t1 * .. * tN)|
   \item Sum-types \verb/[ C1 of t1 | C2 of t2 ..| Cn of tn ]/
   \item Records  \verb|{f1:t1; f2:t2; fn:tn}|
   \item Polymorphic Variants \verb/[< `c1 of t1 | .. | `cN of tN ]/
\end{enumerate}

Notice they have different separator, and delimiter, you can build an
\textit{incomplete} ast when you have no delimiters


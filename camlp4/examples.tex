\section{Examples}
\subsection{Pa\_python}

\inputminted[fontsize=\scriptsize,]{ocaml}{camlp4/code/edsl/test.ml}

Now we transform the ast to the semantics what we want, we write Ast
transformation using revised syntax(robust, and disambiguous) .

\inputminted[fontsize=\scriptsize,]{ocaml}{camlp4/code/edsl/translate.ml}

Notice, we could either modify using functor interface(tedious, maybe
robust, but it does not gurantee anything)

Test file is like this 
\inputminted[fontsize=\scriptsize,]{python}{camlp4/code/edsl/test_py.ml}

And out put
\inputminted[fontsize=\scriptsize,]{python}{camlp4/code/edsl/test_py.ppo.ml}

Everything seems pretty easy, but be careful! When you use camlp4
extensions, use ocamlobjinfo to examine which modules are exactly
linked, normally you only need to link your own module file, don't try
to link other modules, otherwise you will get trouble. 

ocamlbuild is not that smart, use \verb|ocamlbuild -clean| to keep
your source tree clean.

The myocamlbuild.ml file now seems rather trival to write 
\begin{ocamlcode}
  (fun _ -> begin 
    flag ["ocaml"; "pp"; "use_python"]
      (S[A"test.cmo"; A"translate.cmo"]);
    flag ["ocaml"; "pp"; "use_list"]
      (S[A"pa_list.cmo"]);
  end ) +> after_rules;
  (fun _ -> begin
    dep ["ocamldep";  "use_python"] ["test.cmo"; "translate.cmo"];
    dep ["ocamldep";  "use_list"] ["pa_list.cmo"];
    Options.ocaml_lflags := "-linkall" :: !Options.ocaml_lflags;

   end ) +> after_rules;
\end{ocamlcode}
Make sure you know which module you linked.


\subsection{Pa\_list}
\inputminted[fontsize=\scriptsize]{ocaml}{camlp4/examples/pa_list.ml}


Notice that here for \verb|pa_list.cmo|, we did not need to link
\verb|batteries|, since at executation time, it did not bother
\verb|Batteries| at all. Read the commented output, you see at this
phase, that \verb|Nil| was acutally translated into \verb|"Nil"|. So
actually you don't bother Batteries at all. Even you said
\verb|open Batteries|, ocaml compiler was not that stupid to link it.


The test file is also interesting.
\inputminted[fontsize=\scriptsize, fontsize=\scriptsize, ]{ocaml}{camlp4/example/test_pa_list.ml}


This extension illustrates an example to tell us how to extend your
parser. Remember, Camlp4 is a source to source level
pretty-printer. So you don't need to be responsible for which library
will be linked at the time of writing syntax extension. The user may
make sure such module or value really exist.


\subsection{Pa\_abstract}
\inputminted[fontsize=\scriptsize]{ocaml}{camlp4/examples/pa_abstract.ml}
\inputminted[fontsize=\scriptsize, fontsize=\scriptsize, ]{ocaml}{camlp4/examples/test_pa_abstract.ml}
This example shows how to customize your options for your syntax extension.


\subsection{Pa\_apply}
\inputminted[fontsize=\scriptsize]{ocaml}{camlp4/examples/pa_apply.ml}
\inputminted[fontsize=\scriptsize]{ocaml}{camlp4/examples/test_pa_apply.ml}


This is a dirty way to write a filter plugin, you may write a formal
way which uses the interface of \verb|Camlp4.Register|.


\subsection{Pa\_ctyp}
\inputminted[fontsize=\scriptsize]{ocaml}{camlp4/examples/pa_ctyp.ml}
This example shows how to make use of the existing parser or printer
of camlp4.


\subsection{Pa\_exception\_wrapper}
\inputminted[fontsize=\scriptsize]{ocaml}{camlp4/examples/pa_exception_wrapper.ml}
\inputminted[fontsize=\scriptsize]{ocaml}{camlp4/examples/test_pa_exception_wrapper.ml}

\subsection{Pa\_exception\_tracer}
\inputminted[fontsize=\scriptsize]{ocaml}{camlp4/examples/pa_exception_tracer.ml}
\inputminted[fontsize=\scriptsize, fontsize=\scriptsize,
]{ocaml}{camlp4/examples/test_pa_exception_tracer.ml}


\subsection{Pa\_freevars}


\subsection{Pa\_freevars\_filter}


\subsection{Pa\_global\_handler}

\subsection{Pa\_holes}

\subsection{Pa\_minimm}


\subsection{Pa\_plus}

\subsection{Pa\_zero}

\subsection{Parse\_arith}


\subsection{Pa\_estring}
\inputminted[fontsize=\scriptsize]{ocaml}{camlp4/example/pa_estring.ml}


\subsection{Pa\_holes}

%%% master.tex --- 
%% \revision$Header: /Users/bob/SourceCode/Notes/ocaml-hacker.tex,v 0.0 2011/10/23 02:58:53 bob Exp$




\documentclass[svgnames,12pt,a4paper]{report}
\usepackage[latin9]{inputenc}
\usepackage[letterpaper]{geometry}
\geometry{verbose,tmargin=1in,bmargin=1in,lmargin=1in,rmargin=1in}


\usepackage{amsmath}
\usepackage{amssymb}
\usepackage{graphicx}
\usepackage{float}
\usepackage{array}
\usepackage{tikz}
\usepackage{enumerate}
\usepackage{lmodern}
\usepackage[T1]{fontenc}
\usepackage{textcomp}
\usepackage{hyperref}
\usepackage{listings}
\usepackage{fancyvrb}
\usepackage{wasysym}
\usepackage[online]{suthesis-2e}
\usepackage{todonotes}
\usepackage{caption}
\usepackage{minted}


\usepackage{setspace}

\definecolor{MyDarkBlue}{rgb}{0,0.08,0.45}
\definecolor{lightgray}{rgb}{0.9,0.9,0.9}
\definecolor{lightlightgray}{rgb}{0.98,0.98,0.98}


%% begin listing configuration 
\lstset{
  basicstyle=\footnotesize\ttfamily,
  numberstyle=\tiny,
  % numbers=left,
  stepnumber=5,
  numbersep=5pt,
  tabsize=2,
  extendedchars=true,
  breaklines=true,
  keywordstyle=\color{blue}\bfseries, % to do
  stringstyle=\color{orange}\ttfamily, %to do
  identifierstyle=\color{teal},
  showstringspaces=true,
  showspaces=false,
  showtabs=false,
  % xleftmargin=17pt,
  framexrightmargin=5pt,
  framexbottommargin=4pt,
  % backgroundcolor=\color{lightgray}, % to do
  commentstyle=\color{red},% todo 
  % frame=lines,
  frame=bottom,
  backgroundcolor=\color{lightlightgray},
  framerule=1pt,
  % emph={square,root},
  % emphstyle=\underbar,
  %% language={[Objective]Caml},
}
\DeclareCaptionFont{white}{\color{white}}
\DeclareCaptionFormat{listing}{\colorbox[cmyk]{0.43, 0.35, 0.35,0.01}{\parbox{\textwidth}{\hspace{15pt}#1#2#3}}}
\captionsetup[lstlisting]{format=listing,labelfont=white,textfont=white, singlelinecheck=false, margin=0pt, font={bf,footnotesize}}

%% end


\makeatletter

\makeatother


% \newenvironment{ocamlcode}{\Verbatim[fomartcom=\color{blue}]}{\endVerbatim}
% New commands serve as shorthand for frequently used command combinations.
% \newcommand{\ind}[1]{\mathbf{1}\left(#1\right)}
% \newcommand{\bx}{\mathbf{x}}
% \newcommand{\E}{\mathbf{E}}
% \DefineVerbatimEnvironment{ocamlcode}{Verbatim}{formatcom=\color{red},fontsize=\scriptsize}
% \DefineVerbatimEnvironment{bluetext}{Verbatim}{formatcom=\color{MyDarkBlue},fontsize=\scriptsize}
% \DefineVerbatimEnvironment{ocamlcode}{Verbatim}{formatcom=\color{blue},fontsize=\scriptsize}
% \lstnewenvironment{ocamlcode}
% {\singlespacing}{}


\usepackage{etoolbox}

\setlength\partopsep{-\topsep}
\addtolength\partopsep{-\parskip}
\addtolength\partopsep{0.5cm}


%% \AtBeginEnvironment{minted}{\fontsize{10}{10}\selectfont}


%% \AtBeginEnvironment{bluetext}{\singlespacing%
%%     \fontsize{10}{10}\selectfont}


%% \AtBeginEnvironment{inputminted}{\singlespacing%
%%     \fontsize{10}{10}\selectfont}


\newcommand{\ChangeLine}[1]{%
\ifodd\value{FancyVerbLine}%
\textcolor{red}{#1}\else\textcolor{blue}{#1}\fi}


\newminted{ocaml}{fontsize=\scriptsize}
%% \newinputminted{ocaml}

%% \newminted[ocamlcode]{ocaml}{}
%% \newminted[ocamlcode]{ocaml}{}
%% \newminted[smallcode]{ocaml}{}
%% \newminted[ocamlcode]{ocaml}{frame=single,
%%  formatcom=\renewcommand{\FancyVerbFormatLine}{\ChangeLine}}


\lstnewenvironment{bluetext}
{\singlespacing\lstset{backgroundcolor=\color{lightgray}}}{}

\DefineVerbatimEnvironment{alternate}{Verbatim}%
{formatcom=\renewcommand{\FancyVerbFormatLine}{\ChangeLine},%
fontsize=\scriptsize,frame=lines}{}

% \def\dashfill{\cleaders\hbox{-}\hfill}

\begin{document}



\setcounter{tocdepth}{4}
\title{OCaml Hacks}
\author{Hongbo Zhang}

\maketitle
\prefacesection{Preface}
\begin{quotation}
  \textit{
This is a book about hacking in ocaml.
It's assumed that you already understand the underlying theory. Happy hacking
Most parts are filled with code blocks, I will add some comments in the future. Still a book in progress. Don't distribute it.}

\smiley
\end{quotation}
\prefacesection{Acknowledgements}
\todo{write later}



\newpage

\tableofcontents
\listoftodos
\vspace*{1cm}

\newpage



\newpage 
\chapter{eco-system}

% ocamlbuild


\section{ocamlbuild}
\begin{enumerate}
\item directory hierarchy \\
code : \textit{\_build}
\begin{enumerate}
\item ob \emph{automatically creates a symbol link} to the executable it
  produces in the current directory
\item ob copies the sources and compiles them in \_build (default)
\item hygiene rules at start up (.cmo, .cmi, or .o should appear
  outside of the \_build) (-no-hygiene)
\item ob must be invoked in the root directory
\end{enumerate}

\item arguments

\begin{enumerate}
\item \textit{ocamlbuild -quite xx.native -- args}
\item \textit{ocamlbuild -quite -use-ocamlfind xx.native -- args}
\item -log -verbose -clean \\
  check \textit{\_build/\_log} file for detailed building process 
\item -cflags \\
  pass flags to \textbf{ocamlc} i.e. 
  -cflags -I,+lablgtk,-rectypes. (needed at compile time)
\item -lflags \\
  needed at linking time 
\item -libs \\
   linking with \textbf{ external} libraries. i.e. \emph{-libs unix,num}.
   you may need   \emph{-cflags -I,/usr/local/lib/ocaml}  \emph{-lflags -I,/usr/local/lib/ocaml} to make it work 
\item -use-ocamlfind
\item -pkgs oUnit

\item \textit{mllib} file

  \begin{redcode}
cat top_level.mllib    
\end{redcode}
\begin{bluecode}
Dir_top_level_util
Dir_top_level  
\end{bluecode}

then you can \textit{ocamlbuild top\_level.cma}, then you can use
\textit{ocamlobjinfo} to see exactly which modules are compacted into
it.

\begin{redcode}
ocamlobjinfo _build/top_level.cma | grep Unit  
\end{redcode}

\begin{bluecode}
Unit name: Dir_top_level_util
Unit name: Dir_top_level
\end{bluecode}


\end{enumerate}

\item with lex yacc, ocamlfind 

\begin{enumerate}
\item .mll .mly supported by default, \textit{menhir (-use-menhir)} or add a line
  \textit{true : use\_menhir}

\item add a line in tags file 
  \textit{<*.ml> : pkg\_sexplib.syntax, pkg\_batteries.syntax, syntax\_camlp4o} \\
  here \textit{syntax\_camlp4o} is translated by myocamlbuild.ml to \emph{-syntax camlp4o} to pass to ocamlfind

\item another typical tags file using \textbf{ syntax extension}
  

\begin{bluetext}
<*.ml>: package(lwt.unix), package(lwt.syntax), syntax(camlp4o) -- only needs lwt.syntax when preprocessing
"prog.byte": package(lwt.unix)
\end{bluetext}

  
\end{enumerate}

\item predicates
  
\begin{enumerate}
\item simple regexes 
  <**/*.ml>    means that .ml files in \emph{current dir or sub dir}

\begin{bluetext}
  <**/*.ml> <**/*.mli> <**/*.mlpack> <**/*.ml.depends> : ocaml 
  <**/*.byte> : ocaml, byte, program 
  <*.ml> or <*.byte> or <*.native> : pkg_oUnit 
  <**/*.{native,byte}> : use_unix 
  <{batMutex,batRMutex}.{ml,mli}>: threads 
  e1 or e2 , e1 and e2, not e, true ,false 
  true:use_menhir, pkg_oUnit, pkg_batteries,
\end{bluetext}
pkg needs ocamlbuild plugin support.

\item  ocamlbuild cares white space, \textbf{ take care when write tags file}
\item foo.itarget
  

\begin{redcode}
bash$ cat foo.itarget
\end{redcode}

\begin{bluecode}
main.native
main.byte
stuff.docdir/index.html
\end{bluecode}

% $

\textbf{ ocamlbuild foo.otarget}

\item packing modules


\begin{redcode}
$ cat foo.mlpack
\end{redcode}

\begin{bluecode}
Bar
Baz 
\end{bluecode}

% $

\item document \\
  when you use -keep-code flag in myocamlbuild.ml, \textit{only} document of exposed modules are kept, not very useful \\
  \verb| flag ["ocaml"; "doc"] & S[A"-keep-code"];|
  ocamldep seems to be \textbf{ lightweight}

\item syntax extension \\ 
  Just for preprocessing, you can also use pp. \\
  \verb|<pa_*r.{ml,cmo,byte}> : pkg_dynlink , pp(camlp4rf ), use_camlp4_full| \\
  Here it not only use preporcessor, but also \textbf{ link} with it. \\
  Take ulex for example, for \textbf{ pre-processing} \\
  \verb|<*_ulex.ml> : syntax_camlp4o,pkg_ulex,pkg_camlp4.macro|,
  For \textbf{ linking} \\
  \verb|<*_ulex.{byte,native}> : pkg_ulex|. \\
  Normal for any revised syntax, you can say \\
  \verb|<*_r.ml>:syntax_camlp4r,pkg_camlp4.quotations.r,pkg_camlp4.macro,pkg_camlp4.extend| \\
  You can use \textbf{ several syntax extensions} together, as above. \\
  ``\verb|pa_vector_r.ml|'':\verb|syntax_camlp4r|,\verb|pkg_camlp4.quotations.r|,
  \verb|pkg_camlp4.extend|, \verb|pkg_sexplib.syntax|
  for \textbf{ preprocessing}, and \\
  \verb|<pa_vector_r.{cmo,byte,native}>:pkg_dynlink,use_camlp4_full,pkg_sexplib| for \textbf{ linking} . \\
  \textbf{ order matters}  \\
  For \textbf{ original} syntax, \verb|<*_o.ml> : syntax_camlp4o,pkg_sexplib.syntax| \\
  For \textbf{ filter} \verb|"map_filter_r.ml" : pp(camlp4r -filter map)|. and \\
  \verb|"wiki_r.ml" or "wiki2_r.ml"  : pp(camlp4rf -filter meta), use_camlp4_full|\\
  The .mli file also needs \verb|"wiki2_r.mli" : use_camlp4_full| \\
  for more information, check out \textbf{ camlp4/examples}. when you use pp flag, you need to specify the path to \verb|pa_xx.cmo|, so symbol link may help.
\end{enumerate}

\item debug profile

  \begin{enumerate}

  \item  use  the appropriate target extensions, .d.byte for debugging or .p.native for profiling
  \item add the debug or profile tags. You must either use \textit{-tag debug} or \textit{-tag profile}, or add a \textit{true: debug}. byte code profiler not supported in ocamlbuild.
  \end{enumerate}
\end{enumerate}
%%% Local Variables: 
%%% mode: latex
%%% TeX-master: "master"
%%% End: 



\section{Godi,otags}
\label{sec:godi}
\subsection{CheatSheet}



\begin{bashcode}
godi_make makesum
godi_make  install
godi_console info #installed software
godi_console list

godi_add ~/SourceCode/ML/godi/build/packages/All/godi-calendar-2.03.tgz
godi_console perform -build godi-ocaml-graphics  >.log 2 >1
godi_console perform (fetch, extract, patch, configure, build, install)
godi_console update
\end{bashcode}
\captionof{listing}{godi command}

\begin{bashcode}
otags -r -q -a -o ~/tags/ocaml/TAGS /dest/to/your/source  
\end{bashcode}
\captionof{listing}{otags command}


%%% Local Variables: 
%%% mode: latex
%%% TeX-master: "master"
%%% End: 


\subsection{ocamlfind}

\href{http://projects.camlcity.org/projects/dl/findlib-1.2.3/doc/ref-html/r17.html}{findlib}


\begin{itemize}
\item \emph{ocamlfind browser -all }

\item \emph{ocamlfind browser -package batteries}

\item syntax extension \\
ocamlfind ocamldep -package camlp4,xstrp4 -syntax camlp4r file1.ml file2.ml \\
ocamlfind can only handle flag camlp4r, flag camlp4o, so if you want to
use other extensions,  use -package camlp4,xstrp4, i.e. -package camlp4.macro
  
\item META file (exmaple)


\begin{bluetext}
name="toplevel"
description = "toplevel hacking"
requires = ""
archive(byte) = "dir_top_level.cmo"
archive(native) = "dir_top_level.cmx"
version = "0.1"
\end{bluetext}


\item simple Makefile for ocamlfind 


\begin{bluetext}
all:
	   @ocamlfind install toplevel META _build/*.cm[oxi]
clean: 
	   @ocamlfind remove toplevel 
\end{bluetext}


         
\end{itemize}

  

%%% Local Variables: 
%%% mode: latex
%%% TeX-master: "master"
%%% End: 




\subsection{toplevel}

\begin{enumerate}

\item \verb|#directory ``_build'' ;; #directory ``+camlp4'' ;; #load ``...''|
\item trace
\item labels (ignore labels in function types)
\item \verb|warnings print_depth print_length|
\item hacking Toploop
  \begin{itemize}
  \item re-direct 


\begin{redcode}
Toploop.execute_phrase (bool->formatter->Parsetree.toplevel_phrase->bool)
Toploop.read_interactive_input
\end{redcode}
\begin{bluecode}
- : (string -> string -> int -> int * bool) ref = (* topdirs.cmi *)
\end{bluecode}



\begin{redcode}
  Hashtbl.keys Toploop.directive_table;;
\end{redcode}

\begin{bluecode}
print_depth use principal untrace_all load list trace show directory u cd install_printer print_length labels remove_printer camlp4o quit untrace thread camlp4r  
\end{bluecode}


\begin{redcode}
Topdirs.(dir_load,dir_use,dir_install_printer,dir_trace,dir_untrace,dir_untrace_all,load_file,dir_quit,dir_cd);;  
\end{redcode}

\begin{bluecode}
- : (Format.formatter -> string -> unit) *
    (Format.formatter -> string -> unit) *
    (Format.formatter -> Longident.t -> unit) *
    (Format.formatter -> Longident.t -> unit) *
    (Format.formatter -> Longident.t -> unit) *
    (Format.formatter -> unit -> unit) *
    (Format.formatter -> string -> bool) * (unit -> unit) * (string -> unit)
\end{bluecode}




\item store env

  \begin{bluecode}
let env = !Toploop.toplevel_env
... blabbla ...     
Toploop.toplevel_env := env     
\end{bluecode}
\begin{bluecode}
Toploop.initialize_toplevel_env ()  
\end{bluecode}
  \item \textbf{sample file  for references } in findlib


\begin{lstlisting}[caption=Toplevel Code Sample,label=toplevel]
(* For Ocaml-3.03 and up, so you can do: #use "topfind" and get a
 * working findlib toploop.
 * First test whether findlib_top is already loaded. If not, load it now.
 * The test works by executing the toplevel phrase "Topfind.reset" and
 * checking whether this causes an error.
 *)
let exec_test s =
  let l = Lexing.from_string s in
  let ph = !Toploop.parse_toplevel_phrase l in
  let fmt = Format.make_formatter (fun _ _ _ -> ()) (fun _ -> ()) in
  try
    Toploop.execute_phrase false fmt ph
  with
      _ -> false
in
if not(exec_test "Topfind.reset;;") then (
  Topdirs.dir_load Format.err_formatter "/Users/bob/SourceCode/ML/godi/lib/ocaml/pkg-lib/findlib/findlib.cma";
  Topdirs.dir_load Format.err_formatter "/Users/bob/SourceCode/ML/godi/lib/ocaml/pkg-lib/findlib/findlib_top.cma";
);;
\end{lstlisting}

    
  \item \href{file:/Users/bob/SourceCode/ML/godi/build/distfiles/findlib-1.2.7/src/findlib/topfind.ml}{topfind.ml} \\
    ideas : we can write \textbf{some utils} to check code later 
    yeah. A poor man's code search tool (in the library \verb|dir_top_level|)


\begin{alternate}
se;;
- : ?ignore_module:bool -> (string -> bool) -> string -> string list =
se ~ignore_module:false (FILTER _*  "char" space* "->" space* "bool") "String";;
\end{alternate}

\begin{lstlisting}
module Dont_use_this_name_ever :
    val contains : string -> char -> bool
    val contains_from : string -> int -> char -> bool
    val rcontains_from : string -> int -> char -> bool
    val filter : (char -> bool) -> string -> string
    module IString : sig type t = String.t val compare : t -> t -> int end
    module NumString : sig type t = String.t val compare : t -> t -> int end
    module Exceptionless :
    module Cap :
        val filter : (char -> bool) -> [> `Read ] t -> 'a t
        val contains : [> `Read ] t -> char -> bool
        val contains_from : [> `Read ] t -> int -> char -> bool
        val rcontains_from : [> `Read ] t -> int -> char -> bool
        module Exceptionless :
\end{lstlisting}


    

\begin{redcode}
Hashtbl.add
    Toploop.directive_table
    "require"
    (Toploop.Directive_string
       (fun s ->
	  protect load_deeply (Fl_split.in_words s)
       ))
;;
Hashtbl.add Toploop.directive_table "pwd"
(Toploop.Directive_none (fun _ -> 
  print_endline (Sys.getcwd ())));;
#pwd;;
\end{redcode}

\begin{bluecode}
/Users/bob/SourceCode/Notes
\end{bluecode}



  \end{itemize}
\end{enumerate}


%%% Local Variables: 
%%% mode: latex
%%% TeX-master: "master"
%%% End: 


% section svn
\section{Git}
\begin{itemize}
\item ignore set \\
  \verb|_log _build *.native *.byte *.d.native *.p.byte|
\end{itemize}


\chapter{lexing}
\section{lexing-ulex-ocamllex}
\label{sec:parsing-lexing-1}
\section{Lexing}
Ulex support\textbf{unicode}, while ocamllex don't, the tags file is
as follows

\begin{bluetext} 
$ cat tags
<*_ulex.ml> : syntax_camlp4o,pkg_ulex
<*_ulex.{byte,native}> : pkg_ulex
\end{bluetext} %$

Use default myocamlbuild.ml, like \verb|ln -s ~/myocamlbuild.ml| and
make a symbol link \verb|pa_ulex.cma| to camlp4 directory,this is
actually not necessary but sometimes for \verb| debugging purpose|,
as follows, this is pretty easy 
\verb| camlp4o pa_ulex.cma -printer OCaml test_ulex.ml -o test_ulex.ppo|
So, you should do symbol link and write a very simple plugin like this

\inputminted[fontsize=\scriptsize,firstline=117,lastline=125]{ocaml}{lexing/code/ulex/myocamlbuild.ml}
And your tags file should be like this 
\begin{bluetext}
<test1.ml> : camlp4o, use_ulex
<test1.{cmo,byte,native}> : pkg_ulex
\end{bluetext}
You can analyze the \verb|_build/_log| to know how it works.
\begin{bluetext}
### Starting build.
# Target: test1.ml.depends, tags: { camlp4o, extension:ml,
# file:test1.ml, ocaml, ocamldep, quiet, traverse, use_ulex }
ocamlfind ocamldep -pp 'camlp4o pa_ulex.cma' -modules test1.ml >
# test1.ml.depends # cached
\end{bluetext}

The nice thing is that you can \verb|ocamlbuild test1.pp.ml| directly
to view the source. A nice feature.


Ulex does not support \verb| as | syntax as ocamllex.  Its extended
syntax is like this:
\begin{ocamlcode} 
let regexp number = ['0'-'9'] + 
let regexp line = [^'\n']* ('\n' ?)  
let u8l = Ulexing.utf8_lexeme 
let rec lexer1 arg1 arg2 .. = lexer 
   |regexp -> action |..  
and lexer2 arg1 arg2 .. = lexer
   |regexp -> action |...
\end{ocamlcode}

\textbf{Roll back} 

Ulexing.rollback lexbuf, so for string lexing, you
can rollback one char, and \textit{plugin your string lexer}, but
\textit{not generally usefull}, ulex \textit{does not support shortest
mode yet}. Sometimes the semantics of rolling back is not what you
want as recursive descent parser.

\textbf{Abstraction with macro package} 

Since you need inline to do macro prepossessing, so use syntax
extension macro to \textbf{ inline} your code,


\begin{bluetext}
 <*_ulex.ml> : syntax_camlp4o,pkg_ulex,pkg_camlp4.macro
 <*_ulex.{byte,native}> : pkg_ulex
\end{bluetext}

Attention!  Since you use ocamlbuild to build, then you need to copy
you include files to \verb|_build| if you use relative path in
\textbf{INCLUDE} macro, otherwise you should use absolute path.

 You can predefine some regexps (copied from ocaml source code)
\verb| parsing/lexer.ml|.

\inputminted[fontsize=\scriptsize, fontsize=\scriptsize, ]{ocaml}{/Users/bobzhang1988/predefine_ulex.ml}

You can also use myocamlbuild plugin to write a dependency to avoid
all these problems. But I am not sure which is one is better, copy
paste or using \verb|INCLUDE| macro. Maybe we are over-engineering.


\subsection{Ulex interface}

Roughly equivalent to the module Lexing, except that its lexbuffers
handles Unicode code points  OCaml type \verb|int| in the range
\verb|0.. 0x10ffff| instead of bytes (OCamltype : \verb|char|).

You can customize implementation for lex buffers, define a module L
which implements \emph{start,next,mark, and backtrack and the Error
  exception}.  They need not work on a type named lexbuf, you can use
the type name you want.  Then, just do in your \emph{ulex-processed}
source, before the first lexer specification \verb|module Ulexing = L|
If you inspect the processed output by camlp4, you can see that the
generated code \emph{introducing Ulexing } very \emph{late} and
actually use very limited functions, other functions are just provided
for your convenience, and it did not have any type annotations, so you
really can customize it. I think probably ocamllex can do the similar
trick.
    
\inputminted[fontsize=\scriptsize,
             fontsize=\scriptsize, ]{ocaml}{lexing/code/ulex/ulex_intf.mli}
    
Ulex does not handle line position, you have only global char
position, but we are using emacs, not matter too much

\textbf{ATTENTION}

When you use ulex to generate the code, make sure to write the
interface by yourself, the problem is that when you use the default
interface, it will generate \verb|__table__|, and different file may
overlap this name, when you open the module, it will cause a disaster,
so the best to do is \textbf{write your .mli} file.

And when you write lexer, make sure you write the default branch,
check the generated code, otherwise its behavior is weird.


\begin{bluetext}
camlp4of -parser macro pa_ulex.cma test_calc.ml -printer o
or
ocamlbuild basic_ulex.pp.ml
\end{bluetext}

\textbf{A basic Example}

Here is the example of simple basic lexer 
\inputminted[fontsize=\scriptsize, fontsize=\scriptsize, ]{ocaml}{lexing/code/ulex/basic_ulex.ml}
Notice that \verb|ocamlnet| provides a fast \verb|Ulexing module|,
probably you can change its interal representation.


I have written a helper package to make lexer more available
\todo{parser-help to coordinate menhir and ulex}


\section{Ocamlllex}
\href{http://caml.inria.fr/pub/docs/manual-ocaml/manual026.html}{ocamllex}

\begin{enumerate}
\item \textit{module Lexing}
  \begin{ocamlcode}
    se_str "from" "Lexing";;
  \end{ocamlcode}
  
\begin{ocamlcode}  
  val from_string : string -> lexbuf
  val from_function : (string -> int -> int) -> lexbuf
  val from_input : BatIO.input -> Lexing.lexbuf
  val from_channel : BatIO.input -> Lexing.lexbuf    
\end{ocamlcode}

\item syntax \\

  \begin{bluetext}
    {header}
    let ident = regexp ...
    rule entrypoint [arg1 .. argn ] =
       parse regexp {action }
       | ..
       | regexp {action}
    and entrypoint [arg1 .. argn] =
       parse ..
    and ... 
    {trailer}
  \end{bluetext}

  The parse keyword can be replaced by shortest keyword.

  Typically, the header section contains the \textit{open} directives
  required by the actions

  All identifiers starting with \verb|__ocaml_lex| are reserved for use by
  \textbf{ocamllex}
\item example
  for me, best practice is put some test code in the trailer part, and
  use \textit{ocamlbuild fc\_lexer.byte --} to verify, or write a
  makefile. you can write several indifferent rule in a file using and.

  \begin{bluetext}

(* verbatim translate *)
rule translate = parse 
  | "current_directory" {print_string (Sys.getcwd ()); translate lexbuf}
  | _ as c {print_char c ; translate lexbuf}
  | eof {exit 0}

{
  let _ = 
    let chan = open_in "fc_lexer.mll" in begin
    translate (Lexing.from_channel chan ); 
    close_in chan 
    end 

}
    
\end{bluetext}

\begin{alternate}
Legacy.Printexc.print;;
- : ('a -> 'b) -> 'a -> 'b = <fun>  
\end{alternate}
\item caveat \\
  the longest(shortest) win, then consider the order of each regexp
  later.
  Actions are evaluated after the \textit{lexbuf} is bound to the
  current lexer buffer and the identifier following the keyword
  \textit{as} to the matched string.
\item position \\
  The lexing engine manages only the \textit{pos\_cnum} field of
  \textit{lexbuf.lex\_curr\_p} with the number of chars read from the
  start of lexbuf. you are responsible for the other fields to be
  accurate.
  i.e.
  \begin{bluetext}
let incr_linenum lexbuf = Lexing.(
    let pos = lexbuf.lex_curr_p in
    lexbuf.lex_curr_p <- { pos with
      pos_lnum = pos.pos_lnum + 1; (* line number *)
     pos_bol = pos.pos_cnum; (* the offset of the beginning of the
     line *)
    })    
  \end{bluetext}

\item combine with ocamlyacc \\
normally  just add \textit{open Parse} in the header, and use the
token defined in \textit{Parse}

\item tips \\
  \begin{enumerate}
  \item keyword table
    \begin{bluetext}
      {let keyword_table = Hashtbl.create 72
       let _ = ...
     }
     rule token = parse
     | ['A'-'z' 'a'-'z'] ['A'-'z' 'A'-'z' '0'-'9' '_'] * as id
     {try Hashtbl.find keyword_table id with Not_found -> IDENT id}
     | ... 
    \end{bluetext}
  \item for sharing \textbf{why ocamllex sucks}\\
    some complex regexps are not easy to write, like string, but sharing
    is hard. To my knowledge, cpp preprocessor is fit for this task here.
    camlp4 is not fit, it will check other syntax, if you use ulex, camlp4
    will do this job.
    So, my Makefile is part like this
    \begin{bluetext}
lexer : 
	cpp fc_lexer.mll.bak > fc_lexer.mll 
	ocamlbuild -no-hygiene fc_lexer.byte --       
    \end{bluetext}
    even so, sharing is still very hard, since the built in compiler used another way to write string lexing. painful too sharing. so ulex wins in both aspects.
    sharing in ulex is much easier.
  \end{enumerate}

  
\end{enumerate}
%%% Local Variables: 
%%% mode: latex
%%% TeX-master: "master"
%%% End: 




\chapter{parsing}
\section{ocamlyacc or menhir}
\label{sec:ocamlyacc}



\begin{enumerate}
\item syntax \\

  \begin{bluetext}
    % {header
    % }
    %%
    Grammar rules 
    %%
    trailer 
  \end{bluetext}

A tiny example as follows (It has a subtle bug, readers should find it)  
  \begin{bluecode}

% {
  open Printf
  let parse_error s = 
    print_endline "error\n";
    print_endline s ; 
    flush stdout 
%}


%token <float> NUM 
%token PLUS MINUS MULTIPLY DIVIDE CARET UMINUS
%token NEWLINE 

%start input 
%type <unit> input 
%type <float> exp 
%% /* rules and actions */
    
input: /* empty */ {}
    | input line {}
; 

line: NEWLINE {}
    |exp NEWLINE  {printf "\t%.10g\n" $1 ; flush stdout}
;

exp: NUM { $1 }
    |exp exp PLUS {$1 +. $2 }
    |exp exp MINUS {$1 -. $2 }
    |exp exp MULTIPLY {$1 *. $2 }
    |exp exp DIVIDE {$1 /. $2 }
    |exp exp CARET {$1 ** $2 }
    |exp UMINUS {-. $1 }
; 

%%
\end{bluecode}

Notice that start non-terminal can be given \textit{several}, then you will
have a different .mli file, notice that it's different from ocamllex,
ocamlyacc will generate a .mli file, so here we get the output
interface as follows:

\begin{bluetext}
  %type <type> nonterminal ... nonterminal
  %start symbol ... symbol
\end{bluetext}

\begin{bluecode}
type token =
  | NUM of (float)
  | PLUS
  | MINUS
  | MULTIPLY
  | DIVIDE
  | CARET
  | UMINUS
  | NEWLINE
val input :
  (Lexing.lexbuf  -> token) -> Lexing.lexbuf -> unit
val exp :
  (Lexing.lexbuf  -> token) -> Lexing.lexbuf -> float
\end{bluecode}


first gammar
\begin{bluetext}
  input : /*empty*/ {} | input line {}; 
\end{bluetext}
Notice here we \textbf{preferred left-recursive} in yacc.
The underlying theory for LALR prefers LR. because all the elements
\textit{must be shifted onto the stack before the rule can be applied even once.}
empty corresponds Ctrl-d.
\begin{bluetext}
  exp : NUM | exp exp PLUS | exp exp MINUS  ... ; 
\end{bluetext}

Here is our lexer
\begin{bluetext}
{
  open Rpcalc
  open Printf
  let first = ref true
}
let digit = ['0'-'9']
rule token = parse 
  |[' ' '\t' ] {token lexbuf}
  |'\n' {NEWLINE}
  | (digit+ | "." digit+ | digit+ "." digit*) as num 
      {NUM (float_of_string num)}
  |'+' {PLUS}
  |'-' {MINUS}
  |'*' {MULTIPLY}
  |'/' {DIVIDE}
  |'^' {CARET}
  |'n' {UMINUS}
  |_ as c  {printf "unrecognized char %c" c ; token lexbuf}
  |eof {
    if !first then begin first := false; NEWLINE end 
    else raise End_of_file }


{
  let main ()  = 
    let file = Sys.argv.(1) in 
    let chan = open_in file in 
    try 
      let lexbuf = Lexing.from_channel chan in 
      while true do 
        Rpcalc.input token lexbuf 
      done 
    with End_of_file -> close_in chan 

 let _ = Printexc.print main ()

}
\end{bluetext}

we write driver function in lexer for convenience, since lexer depends
on yacc. \textit{Printex.print}
\item precedence associatitvity \\
  operator precedence is determined by the line ordering of the
  declarations; 
  \textit{\%prec} in the grammar section, the \textit{\%prec} simply
  instructs ocamlyacc that the rule \textit{|Minus exp } has the same
  precedence as NEG
  \textit{\%left,\%right,\%nonassoc}
  \begin{enumerate}
  \item The associatitvity of an operator op determines how repeated
    uses of the operator nest: whether \textit{x op y op z} is parsed
    by grouping \textit{x} with \textit{y} or. nonassoc will consider
    it as an error
  \item All the tokens declared in a single precedence declaration
    have equal precedence and nest together according to their
    associatitvity
  \end{enumerate}

  
  \begin{bluetext}
%{
  open Printf
  open Lexing 
  let parse_error s = 
    print_endline "impossible happend! panic \n";
    print_endline s ; 
    flush stdout 
%}

%token NEWLINE 
%token LPAREN RPAREN 
%token <float> NUM 
%token PLUS MINUS MULTIPLY DIVIDE CARET 


%left PLUS MINUS MULTIPLY DIVIDE NEG 
%right CARET 

%start input 
%start exp
%type <unit> input 
%type <float> exp 

%% /* rules and actions */


input: /* empty */ {}
    | input line {}
; 

line: NEWLINE {}
    |exp NEWLINE  {printf "\t%.10g\n" $1 ; flush stdout}
;

exp: NUM { $1 }
    | exp PLUS exp		{ $1 +. $3 }
    | exp MINUS exp		{ $1 -. $3 }
    | exp MULTIPLY exp		{ $1 *. $3 }
    | exp DIVIDE exp		{ $1 /. $3 }
    | MINUS exp %prec NEG	{ -. $2 }
    | exp CARET exp		{ $1 ** $3 }
    | LPAREN exp RPAREN	        { $2 }
;

%%
  \end{bluetext}
  % $
  notice here the \textit{NEG} is a place a holder, it takes the
  place, but it's not a token. since here we need \textit{MINUS} has
  different levels. the interface file is as follows

  \begin{bluetext}
type token =
  | NEWLINE
  | LPAREN
  | RPAREN
  | NUM of (float)
  | PLUS
  | MINUS
  | MULTIPLY
  | DIVIDE
  | CARET

val input :
  (Lexing.lexbuf  -> token) -> Lexing.lexbuf -> unit
val exp :
  (Lexing.lexbuf  -> token) -> Lexing.lexbuf -> float
  \end{bluetext}
  
\item error recovery \\
  by default, the parser function raises exception after calling \textit{parse\_error}
  The ocamlyacc reserved word \textit{error}

  \begin{bluetext}
    line: NEWLINE |exp NEWLINE | error NEWLINE {}
  \end{bluetext}
  if an expression that cannot be evaluated is read, the error will be
  recognized by the third rule for line, and parsing will continue
  (parse\_error is still called). This form of error recovery deals
  with syntax errors. There are also other kinds of errors.

\item location tracking \\
  it's very easy. First, remember to use \textit{Lexing.new\_line} to
  track your line number, then use
  \textit{rhs\_start\_pos, rhs\_end\_pos} to track the symbolposition.
  1 for the leftmost component.
\begin{bluetext}
            Parsing.(
              let start_pos = rhs_start_pos 3 in 
              let end_pos = rhs_end_pos 3 in 
              printf "%d.%d --- %d.%d: dbz"
                start_pos.pos_lnum (start_pos.pos_cnum -start_pos.pos_bol)
                end_pos.pos_lnum (end_pos.pos_cnum - end_pos.pos_bol); 
              1.0
            )    
\end{bluetext}
For groupings, use the following function \textit{symbol\_start\_pos,
  symbol\_end\_pos}

\textit{symbol\_start\_pos} is set to the beginning of the leftmost
component, and \textit{symbol\_end\_pos} to the end of the rightmost component.
\item a complex example

  \begin{bluetext}
%{
  open Printf
  open Lexing 
  let parse_error s = 
    print_endline "impossible happend! panic \n";
    print_endline s ; 
    flush stdout 
  let var_table = Hashtbl.create 16 
%}


%token NEWLINE 
%token LPAREN RPAREN EQ
%token <float> NUM 
%token PLUS MINUS MULTIPLY DIVIDE CARET 
%token <string> VAR 
%token <float->float>FNCT /* built in function */

%left PLUS MINUS
%left MULTIPLY DIVIDE
%left NEG 

%right CARET 
%start input 
%start exp
%type <unit> input 
%type <float> exp 

%% /* rules and actions */


input: /* empty */ {}
    | input line {}
; 

line: NEWLINE {}
    |exp NEWLINE  {printf "\t%.10g\n" $1 ; flush stdout}
    |error NEWLINE {}
;

exp: NUM { $1 }
    | VAR 
        {try Hashtbl.find var_table $1 
          with Not_found -> 
            printf "unbound value '%s'\n" $1;
            0.0
        }
    | VAR EQ exp 
        {Hashtbl.replace var_table $1 $3; $3}
    | FNCT LPAREN exp RPAREN
        { $1 $3 }
    | exp PLUS exp		{ $1 +. $3 }
    | exp MINUS exp		{ $1 -. $3 }
    | exp MULTIPLY exp		{ $1 *. $3 }
    | exp DIVIDE exp		
        { if $3 <> 0. then $1 /. $3 
          else 
            Parsing.(
              let start_pos = rhs_start_pos 3 in 
              let end_pos = rhs_end_pos 3 in 
              printf "%d.%d --- %d.%d: dbz"
                start_pos.pos_lnum (start_pos.pos_cnum -start_pos.pos_bol)
                end_pos.pos_lnum (end_pos.pos_cnum - end_pos.pos_bol); 
              1.0
            )}
    | MINUS exp %prec NEG	{ -. $2 }
    | exp CARET exp		{ $1 ** $3 }
    | LPAREN exp RPAREN	        { $2 }
;

%%



(** lexer file *)
{
  open Rpcalc
  open Printf
  let first = ref true
}


let digit = ['0'-'9']
let id = ['a'-'z']+
rule token = parse 
  |[' ' '\t' ] {token lexbuf}
  |'\n' {Lexing.new_line lexbuf ; NEWLINE}
  | (digit+ | "." digit+ | digit+ "." digit*) as num 
      {NUM (float_of_string num)}
  |'+' {PLUS}
  |'-' {MINUS}
  |'*' {MULTIPLY}
  |'/' {DIVIDE}
  |'^' {CARET}
  |'(' {LPAREN}
  |')' {RPAREN}
  |"sin" {FNCT(sin)}
  |"cos" {FNCT(cos) }
  |id as x {VAR x}
  |'=' {EQ}
  |_ as c  {printf "unrecognized char %c" c ; token lexbuf}
  |eof {
    if !first then begin first := false; NEWLINE end 
    else raise End_of_file }


{
  let main ()  = 
    let file = Sys.argv.(1) in 
    let chan = open_in file in 
    try 
      let lexbuf = Lexing.from_channel chan in 
      while true do 
        Rpcalc.input token lexbuf 
      done 
    with End_of_file -> close_in chan 

 let _ = Printexc.print main ()

}

\end{bluetext}
%$ in my opinion, the best practice is first modify .mly file, then
change .mll file later
\item shift reduce conflict \\
  
  \begin{bluetext}


%token ID COMMA COLON
%token BOGUS /* NEVER LEX */
%start def 
%type <unit>def
%%
def:    param_spec return_spec COMMA {}
        ;
param_spec:  ty {}
        |    name_list COLON ty {}
        ;

/*
return_spec:
             ty {}
        |    name COLON ty {}

        |    ID BOGUS {}   // This rule is never used 
        ;
*/

/* another way to fix the prob */

return_spec : ty {}
        | ID COLON ty {}

ty:        ID {}
        ;
name:        ID {}
        ;
name_list:
             name {}
        |    name COMMA name_list {}
        ;

    
      \end{bluetext}


    \item shift-reduce conflict \\
      a very nice tutorial
      \href{http://www.cs.uiuc.edu/class/sp10/cs421/lectures/lecture%2010%20supp.pdf}{shift-reduce}
        the prec trick is covered not correctly in this tutorial.
        
        The symbols are declared to associate to the left, right,
        nonassoc. The symbols are \textit{usually} tokens, they can
        also be \textit{dummy} nonterminals, for use with the \%prec
        directive in the rule.

        \begin{enumerate}
        \item Tokens and rules have precedences. The precedence of a
          \textit{rule} is the precedence of its \textit{rightmost}
          terminal. you can override this default by using the
          \textit{\%prec} directive in the rule
        \item A reduce/reduce conflict is resolved in favor of the
          first ruel(in the order given by the source file)
        \item A shift/reduce conflict is resolved by comparing the
          \textit{predecence of the rule to be reduced} with the \textit{precedence of
          the token to be shifted}. If the predecence of the rule is
          higher, then the rule will be reducecd; if the predecence of
          the token is higher then token will be shifted.
        \item A shift/reduce conflict between a rule and a token with
          the same precedence will be resolved using the
          associativity.
        \item when a shift/reduce can not be resolved, a warning, and
          in favor of \textit{shift}
        \end{enumerate}
        \begin{bluecode}

%{%}


%token OPAREN CPAREN ID SEMIC DOT INT EQUAL 

%start stmt 
%type <int> stmt 

%%
stmt: methodcall {0} | arrayasgn {0}
; 

/*
previous 
methodcall: target OPAREN CPAREN SEMIC {0}
; 
target:  ID DOT ID {0} |ID {0}
;

our strategy was to remove the "extraneous" non-terminal in the 
methodcall production, by moving one of the right-hand sides of target 
to the methodcall production 

*/

methodcall: target OPAREN CPAREN SEMIC {0} | ID OPAREN CPAREN SEMIC {0}
; 
target:  ID DOT ID {0}
;
arrayasgn: ID OPAREN  INT CPAREN EQUAL INT SEMIC {0}
;


           
\end{bluecode}

\begin{bluecode}
  %{
%}

%token RETURN ID SEMI EQ PLUS

%start methodbody
%type <unit> methodbody

%%

methodbody: stmtlist RETURN ID {}
;
/*
stmtlist: stmt stmtlist {} | stmt {}
;
the strategy here is simple, we use left-recursion instead of 
right-recursion
*/

stmtlist: stmtlist stmt {} | stmt {}
;

stmt: RETURN ID SEMI {} | ID EQ ID PLUS ID {}
;

\end{bluecode}


\begin{bluecode}
%{

%}

%token PLUS TIMES ID LPAREN RPAREN


%left PLUS 
%left TIMES /* weird ocamlyacc can not detect typo TIMEs */ 

/*
here we add assiocaitivity and precedence
*/

%start expr 
%type <unit> expr 


%%

expr: expr PLUS expr {} 
  | expr TIMES expr {}
  | ID {}
  | LPAREN expr RPAREN {}
;
  
\end{bluecode}


\begin{bluecode}
%{

%}

%token ID EQ LPAREN RPAREN IF ELSE THEN


%nonassoc THEN
%nonassoc ELSE

/*
here we used a nice trick to 
handle such ambiguity. set precedence of THEN, ELSE
both needed
*/

%start stmt 
%type <unit> stmt 

%%


stmt: ID EQ ID {}
  | IF LPAREN ID RPAREN THEN stmt {}
  | IF LPAREN ID RPAREN THEN stmt ELSE stmt {}

  
;
/*
It's tricky here we modify the grammar an unambiguous one 
*/


/*
stmt      : matched {}
          | unmatched {}
          ;

matched   : IF '(' ID ')' matched ELSE matched {}
          ;

unmatched : IF '(' ID ')' matched {}
          | IF '(' ID ')' unmatched {}
          | IF '(' ID ')' matched ELSE unmatched {}
          ;
*/
%%

\end{bluecode}


\end{enumerate}
%%% Local Variables: 
%%% mode: latex
%%% TeX-master: "master"
%%% End: 



\chapter{Camlp4}
\label{sec:camlp4}
Camlp4 stands for preprocess-pretty-printer for \verb|OCaml|, it's
extremely powerful and  hard to grasp as well.


  \section{Predicate  Parser}


\todo{predicate parsing stuff}
  \begin{bluetext}
    a : a x | b (x can be anything)
    =>
    a : b r
    r : x r | e
    -----------------------
    exp : exp op exp | prim
    =>
    exp : prim expR
    expR : op exp expR | e 
  \end{bluetext}

  \section{Basics Structure}


\begin{alternate}
bash-3.2$ camlp4 -where
/Users/bob/SourceCode/ML/godi/lib/ocaml/std-lib/camlp4
bash-3.2$ which camlp4
/Users/bob/SourceCode/ML/godi/bin/camlp4
\end{alternate}


You can grep all executables relevant to camlp4 using a one-line bash
as follows:
\begin{bluetext}
find $(dirname $(which ocaml)) -type f -perm -og+rx | grep camlp4 |
while read ss ; do echo $(basename $ss) ; done
\end{bluetext}

\begin{bluetext}  
camlp4
camlp4boot
camlp4o
camlp4o.opt
camlp4of
camlp4of.opt
camlp4oof
camlp4oof.opt
camlp4orf
camlp4orf.opt
camlp4prof
camlp4r
camlp4r.opt
camlp4rf
camlp4rf.opt
mkcamlp4
safe_camlp4
\end{bluetext}

So the tools at hand are \textbf{camlp4, camlp4o, camlp4of, camlp4oof,
  camlp4orf, camlp4r, camlp4rf }

\begin{bluetext}
camlp4 -h

Usage: camlp4 [load-options] [--] [other-options]
Options:
<file>.ml        Parse this implementation file
<file>.mli       Parse this interface file
<file>.(cmo|cma) Load this module inside the Camlp4 core
  -I <directory>   Add directory in search patch for object files.
  -where           Print camlp4 library directory and exit.
  -nolib           No automatic search for object files in library directory.
  -intf <file>     Parse <file> as an interface, whatever its extension.
  -impl <file>     Parse <file> as an implementation, whatever its extension.
  -str <string>    Parse <string> as an implementation.
  -unsafe          Generate unsafe accesses to array and strings.
  -noassert        Obsolete, do not use this option.
  -verbose         More verbose in parsing errors.
  -loc <name>      Name of the location variable (default: _loc).
  -QD <file>       Dump quotation expander result in case of syntax error.
  -o <file>        Output on <file> instead of standard output.
  -v               Print Camlp4 version and exit.
  -version         Print Camlp4 version number and exit.
  -vnum            Print Camlp4 version number and exit.
  -no_quot         Don't parse quotations, allowing to use, e.g. "<:>" as token.
  -loaded-modules  Print the list of loaded modules.
  -parser <name>   Load the parser Camlp4Parsers/<name>.cm(o|a|xs)
  -printer <name>  Load the printer Camlp4Printers/<name>.cm(o|a|xs)
  -filter <name>   Load the filter Camlp4Filters/<name>.cm(o|a|xs)
  -ignore          ignore the next argument
  --               Deprecated, does nothing    
\end{bluetext}

Useful options 

\verb|-str|

\verb|-loaded-modules|

\verb|-parser <name>| load the parser \textit{Camlp4Parsers/<name>.cm(o|a|xs)}


\verb|-printer <name>| load the printer
\textit{Camlp4Printerss/<name>.cm(o|a|xs)}

\verb|-filter <name>| load the filter 
\textit{Camlp4Filters/<name>.cm(o|a|xs).}


\verb|-printer o| means print in original syntax. 


These command lineoptions are all handled in \emph|Camlp4Bin.ml |

\verb|Camlp4o -h| 
There are options added by loaded object files


\verb| -add_locations| Add locations as comment


\verb| -no_comments|


\verb| -curry-constr |


\verb| -sep | Use this string between parsers 


That reflective is true means when extending the syntax of the host
language will \textbf{ also extend the embedded one}


  \begin{tabular}{|c|c|c|c|c|}
    \hline
                      & host     & embedded & reflective & 3.09 equivalent     \\
    camlp4of          & original & original & Yes        & N/A                 \\
    camlp4rf          & revised  & revised  & Yes        & N/A                 \\
    camlp4r-parser rq & revised  & revised  & No         & camlp4r q\_MLast.cmo \\
    camlp4orf         & original & revised  & No         & camlp4o q\_MLast.cmo \\
    camlp4oof         & original & original & No         & N/A                 \\
    \hline
  \end{tabular} \\
  
Camlp4r
    \begin{enumerate}
    \item parser \\
      RP, RPP(RevisedParserParser)
    \item printer \\
      OCaml
    \end{enumerate}


Camlp4rf (extended from camlp4r)
    \begin{enumerate}
    \item parser \\
      RP,RPP, GrammarP, ListComprehension, MacroP, QuotationExpander
    \item printer \\
      OCaml
    \end{enumerate}


Camlp4o (extended from camlp4r)

    \begin{enumerate}
    \item parser \\
      OP, OPP, RP,RPP
    \end{enumerate}

Camlp4of (extended from camlp4o)


    \begin{enumerate}
    \item parser \\
      GrammarParser, ListComprehension, MacroP, QuotatuinExpander
    \item printer 
    \end{enumerate}


Without ocamlbuild, ocamlfind, a simple build would be like this 


\verb|ocamlc -pp camlp4o.opt error.ml|
  

\begin{alternate}
camlp4of -str "let a = [x| x <- [1.. 10] ] " 
let a = [ 1..10 ]
camlp4o -str 'true && false'
true && false
\end{alternate}


\begin{ocamlcode}
(** camlp4of -str "let q = <:str_item< let f x = x >>"*)
let q =
  Ast.StSem (_loc,
    (Ast.StVal (_loc, Ast.ReNil,
       (Ast.BiEq (_loc,
          (Ast.PaId (_loc, (Ast.IdLid (_loc, "f")))),
          (Ast.ExFun (_loc,
             (Ast.McArr
                (_loc,
                (Ast.PaId (_loc, (Ast.IdLid (_loc, "x")))),
                (Ast.ExNil _loc), (Ast.ExId (_loc, (Ast.IdLid (_loc, "x")))))))))))),
    (Ast.StNil _loc))
\end{ocamlcode}
\verb|camlp4of -p r -str 'you code'| is a good way to learn the
corresponding revised syntax.

You can also customize you options in your filter 
\inputminted[fontsize=\scriptsize, fontsize=\scriptsize, firstline=19,lastline=26]{ocaml}{camlp4/examples/pa_abstract.ml}


Now we begin to explore the structure of camlp4 Source Code 

First let's have a look at the directory structure of camlp4 directory.
\begin{bluetext}
|<.>
|--<boot>
|--<build>
|--<Camlp4>
|----<Printers>
|----<Struct>       -- important
|------<Grammar> 
|--<Camlp4Filters>  -- important 
|--<Camlp4Parsers>  -- important 
|--<Camlp4Printers> 
|--<Camlp4Top>
|--<examples>       -- important
|--<man>
|--<test>
|----<fixtures>
|--<unmaintained>   -- many useful extensions unmatained
|----<compile>
|----<etc>
|----<extfold>      -- fold extension 
|----<format>
|----<lefteval>
|----<lib>
|----<ocamllex>
|----<ocpp>
|----<odyl>
|----<olabl>
|----<scheme>
|----<sml>
\end{bluetext}



\verb|Camlp4.PreCast (Camlp4/PreCast.ml)|

Struct directory has module \textit{Loc, Dynloader Functor,
  Camlp4Ast.Make, Token.Make, Lexer.Make, Grammar.Static.Make,
  Quotation.Make}

Camlp4.PreCast \textbf{re-export} such files

    \begin{bluetext}
    Struct/Loc.ml 
    Struct/Camlp4Ast.mlast 
    Struct/Token.ml 
    Struct/Grammar/Parser.ml 
    Struct/Grammar/Static.ml 
    Struct/Lexer.mll 
    Struct/DynLoader.ml 
    Struct/Quotation.ml 
    Struct/AstFilters.ml 
    OCamlInitSyntax.ml 
    Printers/OCaml.ml 
    Printers/OCamlr.ml
    Printers/Null.ml 
    Printers/DumpCamlp4Ast.ml
    Printers/DumpOCamlAst.ml 
    \end{bluetext}


\inputminted[fontsize=\scriptsize, 
             lastline=55]{ocaml}{camlp4/code/PreCast_OCamlInitSyntax.ml}


If we want to define our special syntax, we could do it like this 
\inputminted[fontsize=\scriptsize,]{ocaml}{camlp4/code/my_own_syntax.ml}


Here we see we could get any parser, any printer we want, very convenient.
Notice \verb|Gram.Entry| is \textbf{ dynamic, extensible}

\inputminted[fontsize=\scriptsize,
             firstline=55]{ocaml}{camlp4/code/PreCast_OCamlInitSyntax.ml}
\verb|OCamlInitSyntax| does not do too many things, first, it
initialize all the entries needed later (they are all blank, to be
extended by your functor), after initialization, it created a
submodule \verb|AntiquotSyntax|, and initialize two entries
\verb|antiquot_expr| and \verb|antiquot_patt|, very easy. 

\verb|Camlp4.Sig.ml| All are signatures, there's even no
\verb|Camlp4.Sig.mli|.

\verb|Camlp4.Struct.Camlp4Ast.mlast| This file use macro
\verb|INCLUDE| to include \verb|Camlp4.Camlp4Ast.parital.ml| for
reuse.
    
Notice an interesting module \verb|AstFilters|, is defined by 

\verb|Struct.AstFilters.Make|
It's very simply actually.
\label{AstFilters}
\inputminted[fontsize=\scriptsize,]{ocaml}{camlp4/code/AstFilters.ml}



\begin{ocamlcode}
(** file Camlp4Ast.mlast 
  in the file we have *)
Camlp4.Struct.Camlp4Ast.Make : Loc -> Sig.Camlp4Syntax
  module Ast = struct
     include Sig.MakeCamlp4Ast Loc 
  end ;
\end{ocamlcode}


Let's see what's in \verb|Register| module
\inputminted[fontsize=\scriptsize,
            ]{ocaml}{camlp4/code/Register.ml}


Notice that functors Plugin, SyntaxExtension, OCamlSyntaxExtension,
OCamlSyntaxExtension, SyntaxPlugin, they did the same thing
essentially, they apply the second Funtor to Syntax(Camlp4.PreCast.Syntax).

Functors Printer, OCamlPrinter, OCamlPrinter, they did the same thing,
apply the Make to Syntax, then register it. 

Functor Parser, OCamlParser, did the same thing. 

Functor AstFilter  did nothing interesting.

it sticks to the toplevel 

\inputminted[fontsize=\scriptsize,
             firstline=123,
             lastline=126,
            ]{ocaml}{camlp4/code/Register.ml}.

It mainly hook some global variables, like
\verb|Camlp4.Register.loaded_modlules|, but there's no fresh meat in
this file.
To conclude, Register did nothing, except making your code more
modular, or register your syntax extension.

As we said, another utility, you can inspect what modules you have
loaded in toplevel:
\begin{ocamlcode}
Camlp4.Register.loaded_modules;;
- : string list ref =
{Pervasives.contents =
  ["Camlp4GrammarParser"; "Camlp4OCamlParserParser";
   "Camlp4OCamlRevisedParserParser"; "Camlp4OCamlParser";
   "Camlp4OCamlRevisedParser"]}
\end{ocamlcode}


%%% Local Variables: 
%%% mode: latex
%%% TeX-master: "../master"
%%% End: 


  \section{Ast Transformation}
\label{transform}

The filter \emph{Camlp4MapGenerator} reads \emph{OCaml} type
definitions and generate a class that implements a map traversal.  The
generated class have a method per type you can override to implement a
\emph{map traversal}.

Camlp4 uses the \textbf{ filter} iteself to bootstrap.


\begin{bluecode}
(** file Camlp4Ast.mlast *)
class map = Camlp4MapGenerator.generated;
class fold = Camlp4FoldGenerator.generated;
\end{bluecode}

As above, \verb|Camlp4.Ast| has a corresponding map traversal object,
which could be used by you: (the class was generated by our filter)
\verb|Ast.map| is a class
\begin{bluecode}
let b = new Camlp4.PreCast.Ast.map ;;
val b : Camlp4.PreCast.Ast.map = <obj>
\end{bluecode}

\inputminted[firstline=1,lastline=9]{ocaml}{camlp4/code/ast_add_zero.ml}
you can write it without sytax extension(very tedious),
\inputminted[firstline=11,lastline=31]{ocaml}{camlp4/code/ast_add_zero.ml}
To make life easier, you can write like this 
\inputminted[firstline=32,lastline=38]{ocaml}{camlp4/code/ast_add_zero.ml}

In the module \verb|Camlp4.PreCast.AstFilters|, there are some
utiliies to do filter over the ast.
\begin{bluecode}
    type 'a filter = 'a -> 'a
    val register_sig_item_filter : Ast.sig_item filter -> unit
    val register_str_item_filter : Ast.str_item filter -> unit
    val register_topphrase_filter : Ast.str_item filter -> unit
    val fold_interf_filters : ('a -> Ast.sig_item filter -> 'a) -> 'a -> 'a
    val fold_implem_filters : ('a -> Ast.str_item filter -> 'a) -> 'a -> 'a
    val fold_topphrase_filters :
      ('a -> Ast.str_item filter -> 'a) -> 'a -> 'a
\end{bluecode}


You can also generate map traversal for ocaml type. \emph{put your
  type definition before} you macro, like this
\inputminted{ocaml}{camlp4/code/ast_map.ml}
Without filter, you would write the transformer by hand like this 
\inputminted{ocaml}{camlp4/code/ast_map_o.ml}


Camlp4 use the filter in \verb|antiquot_expander|, for example in
\textit{Camlp4Parsers/Camlp4QuotationCommon.ml}, in the definition of
\verb|add_quotation|, we have


\begin{bluecode}
value antiquot_expander = object
  inherit Ast.map as super ;
  method patt : patt -> patt ...
  method expr : expr -> expr ...
let expand_expr loc loc_name_opt s =
  let ast = parse_quot_string entry_eoi loc s in
  let _ = MetaLoc.loc_name.val := loc_name_opt in
  let meta_ast = mexpr loc ast in
  let exp_ast = antiquot_expander#expr meta_ast in
  exp_ast in
\end{bluecode}


Notice that it first invoked \verb|parse_quot_string|, then do some
transformation, \textbf{ that's how quotation works} !, it will be
changed to your customized quotation parser, and when it goes to
antiquot syntax, it will go back to \textbf{ host language
  parser}. Since the host language parser also support quotation
syntax (due to \textbf{ reflexivity}), so you \textbf{ nest your
  quotation whatever you want.}


  \section{Revised syntax}


\begin{alternate}
  '\''
  '''
  let x = 3
  value  x = 42 ; (str_item) (do't forget ;)
  let x = 3 in x + 8
  let x = 3 in x + 7 (expr)

  -- signature
  val x : int
  value x : int ;

  -- abstract module types
  module type MT
  module type MT = 'a

  -- currying functor 
  type t = Set.Make(M).t
  type t = (Set.Make M).t

  --
  e1;e2;e3
  do{e1;e2;e3}

  --
  while e1 do e2 done
  while e1 do {e2;e3 }
  for i = e1 to e2 do e1;e2 done
  for i = e1 to e2 do {e1;e2;e3}

  --
  () always needed

  x::y
  [x::y]
  x::y::z
  [x::[y::[z::t]]]
  x::y::z::t
  [x;y;z::t]

  match e with
  [p1 -> e1
  |p2 -> e2];


  fun x -> x
  fun [x->x]

  
  value rec  fib = fun [
  0|1  -> 1 
  |n -> fib (n-1) + fib (n-2)
  ];


  fun x y (C z) -> t
  fun x y -> fun [C z -> t]
  -- the curried pattern matching can be done with "fun", but
  -- only irrefutable

  -- legall

  fun []

  match e with []

  try e with []


  -- pattern after "let" and "value" must be irrefutable

  let f (x::y) = ...
  let f = fun [ [x::y] ->  ... ]


  x.f <- y
  x.f := y
  x:=!x + y
  x.val := x.val + y

  --
  int list
  list int


  ('a,bool) foo
  foo 'a bool (*camlp4o -str "type t = ('a,bool) foo" -printer r  -> type t = foo 'a bool*)

  type 'a foo = 'a list list
  type foo 'a = list (list a)

  int * bool
  (int * bool )


  -- abstract type are represented by a unbound type variable
  type 'a foo
  type foo 'a = 'b

  type t = A of i | B
  type t = [A of i | B]


  -- empty is legal
  type foo = []


  type t= C of t1 * t2
  type t = [C of t1 and t2]


  C (x,y)
  C x y


  type t = D of (t1*t2)
  type t = [D of (t1 * t2)]


  D (x,y)
  D (x,y)


  type t = {mutable x : t1 }
  type t = {x : mutable t1}


  if a then b
  if a then b else ()


  a or b & c
  a || b && c


  (+)
  \+


  (mod)
  \mod



  (*  new syntax
     it's possible to group together several declarations
     either in an interface or in an implementation by enclosing
     them between "declare" and "end" *)
     
declare
  type foo = [Foo of int | Bar];
  value f : foo -> int ;
end ;


   [<'1;'2;s;'3>]
   [:`1; `2 ; s; `3 :]

   parser [
     [: `Foo  :] -> e 
     |[: p = f :] -> f ]


   parser []
   match e with parser []


   -- support where syntax
   value e = c
     where c = 3 ;


   -- parser
   value x = parser [
   [: `1; `2  :] -> 1 
   |[: `1; `2 :] -> 2 
   ];

   -- object
   class ['a,'b] point
   class point ['a,'b]
   

   class c = [int] color
   class c = color [int]

   -- signature
   class c : int -> point
   class c : [int] -> point 
   

   method private virtual
   method virtual private

   --
   object val x = 3 end
   object value x = 3; end


   object constraint 'a = int end
   object type 'a = int ; end

   -- label type 
   module type X = sig val x : num:int -> bool  end ;
   module type X = sig value x : ~num:int -> bool ; end;

   --
   ~num:int
   ?num:int


\end{alternate}


Inside a \verb|<< do { ... } >> |you can use
\verb|<< let var = expr1; expr2 >>| like 
\verb|<< let var = expr1 in expr2>>| .

The main goal is to facilitate imperative coding inside a << do {} >>:
\begin{ocamlcode}
do {
 let x = 42;
 do_that_on x;
 let y = x + 2;
 play_with y;
}
\end{ocamlcode}


That's nice but undocumented \textbf{Without} such a syntax the
regular one will make you nest \verb|do { ... }| notations.
\begin{ocamlcode}
do {
 foo 1;
 let x = 43 in do {
    bar x;
 };
 (* x should be out of the scope *)
}
\end{ocamlcode}
Alas \verb|<< let ... in >>| and \verb|<< let ... ; >>| have the same
semantics inside a \verb|<< do { ... } >> |what I regret because
\verb|<< let ... in >>| is not local anymore.

In plain OCaml it's different since \verb|<< ; >>| is a binary
operator so you must see \verb|<< let a = () in a; a >>| like
\verb|<< let a = () in (a; a) >>|.


Another utility to learn some revised syntax

\begin{bluetext}
camlp4o -printer r -str '{ s with foo = bar }' 
{(s) with foo = bar;};

camlp4o -printer r -str 'type t = [`A | `B ]'

type t = [= `A | `B ];
\end{bluetext}
%%% Local Variables: 
%%% mode: LaTex
%%% TeX-master: "../master"
%%% End: 

  \section{Experimentation Environment}

On Toplevel {\bf via findlib}
\begin{ocamlcode}
ocaml
#camlp4r;
#load "camlp4rf.cma"
\end{ocamlcode}

Using ocamlobjinfo to search modules:     
\begin{bluetext}
ocamlobjinfo `camlp4 -where`/camlp4fulllib.cma | grep -i unit
Unit name: Camlp4_import
Unit name: Camlp4_config
Unit name: Camlp4
Unit name: Camlp4AstLoader
Unit name: Camlp4DebugParser
Unit name: Camlp4GrammarParser
Unit name: Camlp4ListComprehension
Unit name: Camlp4MacroParser
Unit name: Camlp4OCamlParser
Unit name: Camlp4OCamlRevisedParser
Unit name: Camlp4QuotationCommon
Unit name: Camlp4OCamlOriginalQuotationExpander
Unit name: Camlp4OCamlRevisedParserParser
Unit name: Camlp4OCamlParserParser
Unit name: Camlp4OCamlRevisedQuotationExpander
Unit name: Camlp4QuotationExpander
Unit name: Camlp4AstDumper
Unit name: Camlp4AutoPrinter
Unit name: Camlp4NullDumper
Unit name: Camlp4OCamlAstDumper
Unit name: Camlp4OCamlPrinter
Unit name: Camlp4OCamlRevisedPrinter
Unit name: Camlp4AstLifter
Unit name: Camlp4ExceptionTracer
Unit name: Camlp4FoldGenerator
Unit name: Camlp4LocationStripper
Unit name: Camlp4MapGenerator
Unit name: Camlp4MetaGenerator
Unit name: Camlp4Profiler
Unit name: Camlp4TrashRemover
Unit name: Camlp4Top
\end{bluetext}


Using \textbf{script} (oco using original syntax is ok), but when
using ocr(default revised syntax), it will have some problems,
i.e. .ocamlinit, and other startup files including findlib, so you'd
better not use revised syntax in the toplevel. here I use
.ocamlinitr (revised syntax) for ocr, but it still have some problem
with findlib, (internal, hard to solve), but it does not really matter.


\begin{alternate}
bash-3.2$ cat /usr/local/bin/oco
ledit -x -h ~/.ocaml_history ocaml dynlink.cma camlp4of.cma -warn-error +a-4-6-27..29
cat `which ocr`
ledit -x -h ~/.ocaml_history ocaml dynlink.cma camlp4rf.cma -init ~/.ocamlinitr -warn-error +a-4-6-27..29
\end{alternate}
% $ 


%%% Local Variables: 
%%% mode: latex
%%% TeX-master: "../master"
%%% End: 

  \section{Extensible Parser}

Camlp4's extensible parser is deeply combined with its own lexer, use
menhir if it is very complex and not ocaml-oriented. It is very hard
to debug in itself. So I suggest it is used to do simple
ocaml-oriented parsing.


First example (a simple calculation)


\inputminted{ocaml}{code/camlp4/simple_calc.ml}


The tags file is 
\begin{bluetext}
<simple_calc.ml> : pp(camlp4of)
<simple_calc.{cmo,byte,native}> : use_dynlink, use_camlp4_full
\end{bluetext}


For oco in \textbf{ toplevel }, extensible parser works \textbf{ quite
  well in original syntax}, so if you don't do quasiquoation in
toplevel, \textit{feel free to use original syntax}.  

Some keywords for extensible paser


  \begin{bluecode}
    EXTEND END  LIST0 LIST1 SEP TRY SELF OPT  FIRST LAST  LEVEL AFTER BEFORE
  \end{bluecode}

SELF represents either the \textbf{current level}, \textbf{the next
  level} or the \textbf{ first level} depending on the \textbf{
  associativity} and the \textbf{position} of the SELF in the rule .

The identifier NEXT, which is a call to the next level of the current
entry.


A brief introduction to its mechanism \\
  There are four generally four phases
  \begin{enumerate}[1]
  \item collection of new keywords, and update of the lexer associated
    to the grammar
  \item representation of the grammar as a tree data structure
  \item left-factoring of each precedence level \\
    when there's a common perfix of symblos(a symbol is a keyword,
    token, or entry ), the parser does not branch until the common parser
    has been parsed. \textbf{that's how grammars are implemented, first the
      corresponding tree is generated, then the parser is generated for
      the tree.}
    some tiny bits 
    \begin{enumerate}[(i)]
    \item Greedy first \\
      when one rule is a prefix of another. 
      \textbf{a token or keyword is preferred over
    epsilon, the empty string (this also holds for other ways that a
    grammar can match epsilon )} factoring happens when the parser is
  built .
    \item \textbf{ explicit token or keyword trumps an entry}
      so you have two prductions, with the same prefix, except the last
      one. one is another entry, and the other is a token, \textbf{the parser will
  first try the token, if it succeeds, it stops, otherwise they try
  the entry.} This sounds weird, but it is reasonable, after
left-factorization, the parser pays no cost when it tries just a
token, it's amazing that even more tokens, the token rule still wins,
and \textbf{even the token rule fails after consuming some tokens, it can
  even transfer to the entry rule }, local try????? .
  \textbf{it seems that after factorization, the rule order may be
    changed }. \\
  \item the data structure representing the grammar is then passed as
    argument to a generic parser 
    \end{enumerate}
  \end{enumerate}


It's really hard to understand how it really works. Here are some
experiments I did, but did not know how to explain

\inputminted{ocaml}{camlp4/code/parser1.ml}
We see that \verb|MGram.Entry.print| is a good utility. 

The processed code is not too indicative, all the dispatch magic
hides in \verb|MGram.extend| function (or |Insert.extend| function)
\textit{camlp4/Camlp4/Struct/Grammar/Insert.ml}

\begin{bluecode}
value extend entry (position, rules) =
      let elev = levels_of_rules entry position rules in
      do {
        entry.edesc := Dlevels elev;
        entry.estart :=
          fun lev strm ->
            let f = Parser.start_parser_of_entry entry in
            do { entry.estart := f; f lev strm };
        entry.econtinue :=
          fun lev bp a strm ->
            let f = Parser.continue_parser_of_entry entry in
            do { entry.econtinue := f; f lev bp a strm }
      };
\end{bluecode}


Factoring only happens in the same level within a rule.

You can do explicit backtracking by hand (npeek trick)
\inputminted{ocaml}{calmp4/code/parser2.ml}





  
\begin{enumerate}[(a)]
  \item left factorization \\
    take rules as follows as an example 
    \begin{bluecode}
  "method"; "private"; "virtual"; l = label; ":"; t = poly_type
  "method"; "virtual"; "private"; l = label; ":"; t = poly_type
  "method"; "virtual"; l = label; ":"; t = poly_type
  "method"; "private"; l = label; ":"; t = poly_type; "="; e = expr
  "method"; "private"; l = label; sb = fun_binding
  "method"; l = label; ":"; t = poly_type; "="; e = expr
  "method"; l = label; sb = fun_binding
\end{bluecode}

The rules are inserted in a tree and the result looks like:
\begin{bluecode}
  "method"
     |-- "private"
     |       |-- "virtual"
     |       |       |-- label
     |       |             |-- ":"
     |       |                  |-- poly_type
     |       |-- label
     |             |-- ":"
     |             |    |-- poly_type
     |             |            |-- ":="
     |             |                 |-- expr
     |             |-- fun_binding
     |-- "virtual"
     |       |-- "private"
     |       |       |-- label
     |       |             |-- ":"
     |       |                  |-- poly_type
     |       |-- label
     |             |-- ":"
     |                  |-- poly_type
     |-- label
           |-- ":"
           |    |-- poly_type
           |            |-- "="
           |                 |-- expr
           |-- fun_binding
      
    \end{bluecode}

This tree is built as long as rules are inserted.
\item \textbf{start and continue}
  At each entry level, the rules are separated into \textbf{two
    trees}:
  \begin{enumerate}[(a)]
  \item The tree of the rules not starting with neither the current entry name
    nor by ``SELF''(start)
  \item The tree of the rules starting with the current entry or by
    SELF, this symbol \textbf{itself not being included} in the tree
  \end{enumerate}

  They determine two functions :
  \begin{enumerate}
  \item The function named {\color{red} ``start''}, analyzing the first tree
  \item The function named {\color{red} ``continue''}, taking, as parameter, a value
    previously parsed, and analyzing the second tree. 
  \end{enumerate}

  A call to an entry, correspond to a call to the \textbf{``start''} function of
  the \textbf{``first''} level of the entry.

  For the ``start'', it tries its tree, if it works, it calls the
  ``continue'' function of the same level, giving the result of ``start''
  as parameter. If this ``continue'' fails, return itself. (continue may
  do some more interesting stuff). If the ``start'' function fails, the
  ``start'' of the next level is tested until it fails. 


  For the ``continue'', it first tries the ``continue'' function of the
  \textbf{next} level. (here + give into *), if it fails or it's the
  last level, it then tries itself, giving the result as parameter. If
  it still fails, return its extra parameter.

  A special case for rules ending with SELF or the current entry
  name. For this last symbol, there's a call to the ``start'' function
  of \textbf{the current level (RIGHTA) or the next level (OTHERWISE)}

  When a SELF or the current entry name is encountered in the middle
  of the rule, there's a call to the start of the \textbf{first level} of the
  current entry.

  Each entry has a start and continue

\begin{bluecode}
(* list of symbols, possible empty *)
LIST0 : LIST0 rule | LIST0 [ <rule definition> -> <action> ]
(* with a separator *)
LIST0 : LIST0 rule SEP <symbol>
| LIST0 [<rule definition > -> <action>] SEP <symbol>
  LIST1 rule
| LIST1 [<rule definition > -> <action > ]
| LIST1 rule SEP <symbol>
| LIST1 [<rule definition > -> <action >] SEP <symbol>
OPT <symbol>
SELF
TRY (* backtracking *)
FIRST LAST LEVEL level, AFTER level, BEFORE level 
\end{bluecode}

STREAM PARSER 
  \begin{enumerate}[(a)]
  \item stream parser

\begin{alternate}
let rec p = parser [< '"foo"; 'x ; '"bar">] -> x | [< '"baz"; y = p >] -> y;;
val p : string Batteries.Stream.t -> string = <fun>
\end{alternate}

\begin{redcode}
camlp4of  -str "let rec p = parser [< '\"foo\"; 'x ; '\"bar\">] -> x | [< '\"baz\"; y = p >] -> y;;"
\end{redcode}

\begin{bluecode}
(** normal pattern : first peek, then junk it *)
let rec p (__strm : _ Stream.t) =
  match Stream.peek __strm with
  | Some "foo" ->
      (Stream.junk __strm;
       (match Stream.peek __strm with
        | Some x ->
            (Stream.junk __strm;
             (match Stream.peek __strm with
              | Some "bar" -> (Stream.junk __strm; x)
              | _ -> raise (Stream.Error "")))
        | _ -> raise (Stream.Error "")))
  | Some "baz" ->
      (Stream.junk __strm;
       (try p __strm with | Stream.Failure -> raise (Stream.Error "")))
  | _ -> raise Stream.Failure
\end{bluecode}

\begin{redcode}
camlp4of -str "let rec p = parser [< x = q >] -> x | [< '\"bar\">] -> \"bar\""
\end{redcode}

\begin{bluecode}
let rec p (__strm : _ Stream.t) =
  try q __strm
  with
  | Stream.Failure -> (* limited backtracking *)
      (match Stream.peek __strm with
       | Some "bar" -> (Stream.junk __strm; "bar")
       | _ -> raise Stream.Failure)
       
\end{bluecode}


Grammar


\begin{bluecode}
se (FILTER _* "Exc_located") "Loc" ;;
exception Exc_located of t * exn 
(** an exception containing an exception *)
se (FILTER _* "type" space+ "t") "Loc";; 
type t = Camlp4.PreCast.Loc.t
\end{bluecode}

we can re-raise the exception so it gets \textit{printed} .

A literal string (like ``foo'') indicates a \textbf{KEYWORD} token ;
using it in a grammar \textbf{registers the keyword} with the lexer. When
it is promoted as a key word, it will no longer be used as a \textbf{
  LIDENT}, so for example, the parser parser, will \textbf{break some valid
programs} before, because \textbf{parser} is now a keyword. This is the
convention, to make things simple, you can find other ways to overcome
the problem, but it's too complicated. you can also say (x= KEYWORD)
or pattern match syntax (`LINDENT x) to get the actual token
constructor. The parser \textbf{ignores} extra tokens after a success.

\item LEVELS \\
  they can be labeled following an entry, like (expr LEVEL "mul"). However,
  explicitly specifying a level when calling an entry defeats the
  start/continue mechanism.
\item NEXT LIST0 SEP OPT TRY \\
  NEXT refers to the entry being defined at the following level
  regardless of assocaitivity or position.
  LIST0 elem SEP sep .
  Both LIST0 and OPT can match the epsilon, but its priority is lower.
  For TRY, non-local backtracking, a Stream.Error will be converted to
  a Stream.Failure.
  \begin{bluecode}
    expr : [[ TRY f1 -> "f1" | f2 -> "f2" ]]
  \end{bluecode}
  
\item nested rule (only one level )

  \begin{bluecode}
    [x = expr ; ["+" | "plus" ]; y = expr -> x + y ]
  \end{bluecode}
\item EXTEND is an expression (of type unit) \\
  it can be evaluated at toplevel, but also inside a function, when
  the syntax extension takes place when the function is called.
\item Translated sample code   
  \begin{bluecode}
open Camlp4.PreCast  
module MGram = MakeGram(Lexer) 
EXTEND MGram 
   GLOBAL: m_expr ;
    m_expr : 
     [[ "foo"; f  -> print_endline "first"
      | "foo" ; "bar"; "bax" -> print_endline "second"]
     ]; 
    f : [["bar"; "baz" ]];  END;;


(** translated code output *)
open Camlp4.PreCast
module MGram = MakeGram(Lexer)
let _ =
  let _ = (m_expr : 'm_expr MGram.Entry.t) in
  let grammar_entry_create = MGram.Entry.mk in
  let f : 'f MGram.Entry.t = grammar_entry_create "f"
  in
    (MGram.extend (m_expr : 'm_expr MGram.Entry.t)
       ((fun () ->
           (None,
            [ (None, None,
               [ ([ MGram.Skeyword "foo"; MGram.Skeyword "bar";
                    MGram.Skeyword "bax" ],
                  (MGram.Action.mk
                     (fun _ _ _ (_loc : MGram.Loc.t) ->
                        (print_endline "second" : 'm_expr))));
                 ([ MGram.Skeyword "foo";
                    MGram.Snterm (MGram.Entry.obj (f : 'f MGram.Entry.t)) ],
                  (MGram.Action.mk
                     (fun _ _ (_loc : MGram.Loc.t) ->
                        (print_endline "first" : 'm_expr)))) ]) ]))
          ());
     MGram.extend (f : 'f MGram.Entry.t)
       ((fun () ->
           (None,
            [ (None, None,
               [ ([ MGram.Skeyword "bar"; MGram.Skeyword "baz" ],
                  (MGram.Action.mk
                     (fun _ _ (_loc : MGram.Loc.t) -> (() : 'f)))) ]) ]))
          ()))
        \end{bluecode}
        
\item if there are unexpected symbols after a correct expression, the trailing symbols are ignored.

\begin{bluecode}
let expr_eoi = Grammar.Entry.mk "expr_eoi" ;;
EXTEND expr_eoi : [[ e = expr ; EOI -> e]]; END ;;
\end{bluecode}

The keywords are stored {\bf in a hashtbl}, so it can be updated
dynamically.

\item level \\
  \begin{bluetext}
  rule ::= list-of-symbols-seperated-by-semicolons -> action 
  level ::=  optional-label optional-associativity
  [list-of-rules-operated-by-bars] 
  entry-extension ::=
  identifier : optional-position  [ list-of-levels-seperated-by-bars ] 
  optional-position ::= FIRST | LAST | BEFORE label | AFTER label |
  LEVEL label  
  \end{bluetext}
\item insert  \\
  when you extend an entry, by default \textbf{ the first level of the
    extension extends the first level of the entry}

for example you a grammar like this : 

\begin{bluecode}
    ["add" LEFTA
    [SELF; "+" ; SELF | SELF; "-" ; SELF]
    | "mult" RIGHTA
    [SELF; "*" ; SELF | SELF; "/" ; SELF]
    | "simple" NONA
    [ "("; SELF; ")"  | INT ]]   
\end{bluecode}

  \begin{bluecode}
EXTEND expr : [[ x = expr ; "plus1plus" ; y = expr -> x + 1 + y ]];
END ;;    
\end{bluecode}
This extends the first level  ``add''. you can double check by printing
the result 

\begin{redcode}
MGram.Entry.print Format.std_formatter m_expr ;;
\end{redcode}

\begin{bluecode}  
expr: [ "add" LEFTA
  [ SELF; "plus1plus"; SELF (** interesting *)
  | SELF; "+"; SELF
  | SELF; "-"; SELF ]
| "mult" RIGHTA
  [ SELF; "*"; SELF
  | SELF; "/"; SELF ]
| "simple" NONA
  [ "("; SELF; ")"
  | INT ((_)) ] ]  
\end{bluecode}

create a new level in the last position 
\begin{redcode}
EXTEND MGram  m_expr: LAST [[x = SELF ; "plus1plus" ; y = SELF ]]; END;;
MGram.Entry.print Format.std_formatter m_expr ;;
\end{redcode}

\begin{bluecode}
expr: [ "add" LEFTA
  [ SELF; "plus1plus"; SELF
  | SELF; "+"; SELF
  | SELF; "-"; SELF ]
| "mult" RIGHTA
  [ SELF; "*"; SELF
  | SELF; "/"; SELF ]
| "simple" NONA
  [ "("; SELF; ")"
  | INT ((_)) ]
| LEFTA
[ SELF; "plus1plus"; SELF ] ] ; 
\end{bluecode}

insert in the level ``mult'' in the first position 
\begin{redcode}
EXTEND MGram  m_expr: LEVEL "mult" [[x = SELF ; "plus1plus" ; y = SELF ]]; END ;;
# MGram.Entry.print Format.std_formatter m_expr ;;
\end{redcode}

\begin{bluecode}
expr: [ "add" LEFTA
  [ SELF; "plus1plus"; SELF
  | SELF; "+"; SELF
  | SELF; "-"; SELF ]
| "mult" RIGHTA
  [ SELF; "plus1plus"; SELF (* added entry*)
  | SELF; "*"; SELF
  | SELF; "/"; SELF ]
| "simple" NONA
  [ "("; SELF; ")"
  | INT ((_)) ]
| LEFTA
  [ SELF; "plus1plus"; SELF ] ]  
\end{bluecode}

insert a new level before ``mult'' 
\begin{redcode}
EXTEND MGram  m_expr: BEFORE "mult" [[x = SELF ; "plus1plus" ; y = SELF ]]; END ;;
# MGram.Entry.print Format.std_formatter m_expr ;;
\end{redcode}

\begin{bluecode}
expr: [ "add" LEFTA
  [ SELF; "plus1plus"; SELF
  | SELF; "+"; SELF
  | SELF; "-"; SELF ]
| LEFTA
  [ SELF; "plus1plus"; SELF ]
| "mult" RIGHTA
  [ SELF; "plus1plus"; SELF
  | SELF; "*"; SELF
  | SELF; "/"; SELF ]
| "simple" NONA
  [ "("; SELF; ")"
  | INT ((_)) ]
| LEFTA
  [ SELF; "plus1plus"; SELF ] ]
\end{bluecode}

\begin{redcode}
se (FILTER _* "val" _* "expr" space+ ":" ) "Syntax" ;;
\end{redcode}

\begin{redcode}  
Gram.Entry.print Format.std_formatter Syntax.expr;;
\end{redcode}


Code listed below  is the expr parse tree
\inputminted{ocaml}{camlp4/code/expr_parse_tree.ml}


A syntax extension of \verb|let try|
\begin{alternate}
let try a = 3 in true with Not_found -> false || false;;
true  
\end{alternate}

First, it uses start parser to parse \textit{let try a = 3 in true
  with Not\_found -> false}, then it calls the cont parser, and the
next level cont parser, etc, and then it succeeds. This also applies
to ``apply'' level.


A tiny extension(you modify the Camlp4.PreCast.Gram, it will be
reflected on the fly)
\inputminted{ocaml}{camlp4/code/syntax_extension_a.ml}

\item when two rules overlapping, the EXTEND statement replaces the
    old version by the new one and displays a warning. 

\begin{redcode}
se (FILTER _* "warning") "Syntax"
\end{redcode}
\begin{bluecode}
type warning = Loc.t -> string -> unit
val default_warning : warning
val current_warning : warning ref
val print_warning : warning
\end{bluecode}
  

  \end{enumerate}

\end{enumerate}


%%% Local Variables: 
%%% mode: latex
%%% TeX-master: "../master"
%%% End: 

  \section {Rewrite of Jake's blog}

Jake's blog is a very comprehensive tutorial for camlp4 introduction.



\subsection{Part1}
Easy to experiment in the toplevel, using my previous {\bf oco},
\begin{bluecode}
open Camlp4.PreCast ;;
let _loc = Loc.ghost ;;
(** 
   blabla...
   An idea, how about writing another pretty printer,
   the printer is awful*)
\end{bluecode}

\subsection{Part2}
  Just ast transform, easy to experiment in toplevel as well.
  \inputminted{ocaml}{camlp4/code/jake/part2.ml}

  Anyother way to verify? The output of printers does not seem to
  guarantee its correctness.  When you do antiquotation, in the cases
  of inserting an AST rather than a string, usually you \textit{do
    not} need tags, when you inserting a string, probably you
  \textit{need it.}


\subsection{Part3 : Quotations in Depth}


\begin{bluecode}
[`QUOTATION x -> Quotation.expand _loc x Quotation.DynAst.expr_tag ]    
\end{bluecode}

The `QUOTATION token contains a record including the body of the
quotation and the \verb|tag|. The record is passed off to the
Quotation module to be expanded. The expander parses the quotation
string starting at some non-terminal(you specified), then runs the
result through the antiquotation expander

\begin{bluecode}
  |`ANTIQUOT (``exp'' | ``'' | ``anti'' as n) s ->
  <:expr< $anti:make_anti ~c:"expr" n s $>>
\end{bluecode}

The antiquotation creates \verb|a special AST node| to hold the body of the
antiquotation, each type in the AST has a constructor (\verb|ExAnt, TyAnt|,
etc.) \verb|c|  means context here.


Here we grep \verb|Ant|, and the output is as follows
\begin{bluetext}
  27 matches for "Ant" in buffer: Camlp4Ast.partial.ml
      5:    | BAnt of string ]
      9:    | ReAnt of string ]
     13:    | DiAnt of string ]
     17:    | MuAnt of string ]
     21:    | PrAnt of string ]
     25:    | ViAnt of string ]
     29:    | OvAnt of string ]
     33:    | RvAnt of string ]
     37:    | OAnt of string ]
     41:    | LAnt of string ]
     47:    | IdAnt of loc and string (* $s$ *) ]
     87:    | TyAnt of loc and string (* $s$ *)
     93:    | PaAnt of loc and string (* $s$ *)
    124:    | ExAnt of loc and string (* $s$ *)
    202:    | MtAnt of loc and string (* $s$ *) ]
    231:    | SgAnt of loc and string (* $s$ *) ]
    244:    | WcAnt of loc and string (* $s$ *) ]
    251:    | BiAnt of loc and string (* $s$ *) ]
    258:    | RbAnt of loc and string (* $s$ *) ]
    267:    | MbAnt of loc and string (* $s$ *) ]
    274:    | McAnt of loc and string (* $s$ *) ]
    290:    | MeAnt of loc and string (* $s$ *) ]
    321:    | StAnt of loc and string (* $s$ *) ]
    337:    | CtAnt of loc and string ]
    352:    | CgAnt of loc and string (* $s$ *) ]
    372:    | CeAnt of loc and string ]
    391:    | CrAnt of loc and string (* $s$ *) ];
\end{bluetext}




ANTIQUOTATION example 


\begin{alternate}
<:expr< $int: "4"$ >>;;
- : Camlp4.PreCast.Ast.expr = Camlp4.PreCast.Ast.ExInt (<abstr>, "4")
<:expr< $`int: 4$ >>;; (** the same result *)
- : Camlp4.PreCast.Ast.expr = Camlp4.PreCast.Ast.ExInt (<abstr>, "4")
<:expr< $`flo:4.1323243232$ >>;;
- : Camlp4.PreCast.Ast.expr = Camlp4.PreCast.Ast.ExFlo (<abstr>, "4.1323243232")
# <:expr< $flo:"4.1323243232"$ >>;;
- : Camlp4.PreCast.Ast.expr = Camlp4.PreCast.Ast.ExFlo (<abstr>, "4.1323243232")
(** maybe the same for flo *)
\end{alternate}



\begin{bluetext}
    match_case:
      [ [ "["; l = LIST0 match_case0 SEP "|"; "]" -> Ast.mcOr_of_list l
        | p = ipatt; "->"; e = expr -> <:match_case< $p$ -> $e$ >> ] ]
    ;
    match_case0:
      [ [ `ANTIQUOT ("match_case"|"list" as n) s ->
            <:match_case< $anti:mk_anti ~c:"match_case" n s$ >>
        | `ANTIQUOT (""|"anti" as n) s ->
            <:match_case< $anti:mk_anti ~c:"match_case" n s$ >>
        | `ANTIQUOT (""|"anti" as n) s; "->"; e = expr ->
            <:match_case< $anti:mk_anti ~c:"patt" n s$ -> $e$ >>
        | `ANTIQUOT (""|"anti" as n) s; "when"; w = expr; "->"; e = expr ->
            <:match_case< $anti:mk_anti ~c:"patt" n s$ when $w$ -> $e$ >>
        | p = patt_as_patt_opt; w = opt_when_expr; "->"; e = expr -> <:match_case< $p$ when $w$ -> $e$ >>
      ] ]
    \end{bluetext}
    
You can see that \verb|match_case0|, if we use the list antiquotation,
the first case in \verb|match_case0| returns an antiquotation with tag
\verb|listmatch_case|,and we get the following expansion


\inputminted{ocaml}{camlp4/code/jake/antiquot_expander.ml}

Here we see the ambiguity of original syntax,

\begin{bluetext}
<< type t = [ $list:List.map (fun c -> <:ctyp< $uid:c$ >>)$]  >>
\end{bluetext}

In original syntax, it does not know it's variant context, or just
type synonm. (you can add a constructor to make it clear)

\subsection{Part4 parsing ocaml itself using camlp4}

We have to use revised syntax here, because when using quasiquotation,
it has ambiguity to get the needed part, revised syntax was designed
to reduce the ambiguity. 

The following code is a greate file parsing ocaml itself.
Do not use MakeSyntax below, since it will introduce unnecessary
abstraction type, which makes sharing code very difficult

\inputminted{ocaml}{camlp4/code/jake/otags.ml}


\begin{bluecode}
let sig = 
  let str = eval "module X = Camlp4.PreCast ;;" 
  and _loc = Loc.ghost in 
  Stream.of_string str |> Syntax.parse_interf  _loc ;;

open Camlp4.PreCast.Syntax.Ast

(* output
SgMod (<abstr>, "X",
 MtSig (<abstr>,
  SgSem (<abstr>,
   SgSem (<abstr>,
    SgTyp (<abstr>,
     TyDcl (<abstr>, "camlp4_token", [],
      TyMan (<abstr>,
       TyId (<abstr>,
        IdAcc (<abstr>,
         IdAcc (<abstr>, IdUid (<abstr>, "Camlp4"), IdUid (<abstr>, "Sig")),
         IdLid (<abstr>, "camlp4_token"))),
       TySum (<abstr>,
        TyOr (<abstr>,
         TyOr (<abstr>,
          TyOr (<abstr>,
           TyOr (<abstr>,
            TyOr (<abstr>,
             TyOr (<abstr>,
              TyOr (<abstr>,
               TyOr (<abstr>,
                TyOr (<abstr>,
                 TyOr (<abstr>,
                  TyOr (<abstr>,
                   TyOr (<abstr>,
                    TyOr (<abstr>,
                     TyOr (<abstr>,
                      TyOr (<abstr>,
                       TyOr (<abstr>,
                        TyOr (<abstr>,
                         TyOr (<abstr>,
                          TyOr (<abstr>,
                           TyOr (<abstr>,
                            TyOf (<abstr>,
                             TyId (<abstr>, IdUid (<abstr>, "KEYWORD")),
                             TyId (<abstr>, IdLid (<abstr>, "string"))),
                            TyOf (<abstr>,
                             TyId (<abstr>, IdUid (<abstr>, "SYMBOL")),
                             TyId (<abstr>, IdLid (<abstr>, "string")))),
                           TyOf (<abstr>,
                            TyId (<abstr>, IdUid (<abstr>, "LIDENT")),
                            TyId (<abstr>, IdLid (<abstr>, "string")))),
                          TyOf (<abstr>,
                           TyId (<abstr>, IdUid (<abstr>, "UIDENT")),
                           TyId (<abstr>, IdLid (<abstr>, "string")))),
                         TyOf (<abstr>,
                          TyId (<abstr>, IdUid (<abstr>, "ESCAPED_IDENT")),
                          TyId (<abstr>, IdLid (<abstr>, "string")))),
                        TyOf (<abstr>,
                         TyId (<abstr>, IdUid (<abstr>, "INT")),
                         TyAnd (<abstr>,
                          TyId (<abstr>, IdLid (<abstr>, "int")),
                          TyId (<abstr>, IdLid (<abstr>, "string"))))),
                       TyOf (<abstr>,
                        TyId (<abstr>, IdUid (<abstr>, "INT32")), ...)),
                      ...),
                     ...),
                    ...),
                   ...),
                  ...),
                 ...),
                ...),
               ...),
              ...),
             ...),
            ...),
           ...),
          ...),
         ...))),
      ...)),
    ...),
   ...)))
*)
\end{bluecode}


\subsection{Part5 Structure Item Filters}
Because I use revised syntax, and take a reference of the
documenation, my ast filter is much nicer than jaked's.  the
documentation of quasiquotation from the wiki page is quite helpful

\inputminted{ocaml}{camlp4/code/jake/pa_filter.ml}

For locally used syntax extension, I found write some tiny bits
ocamlbuild code pretty convenient. In \verb|myocamlbuild.ml|, only
needs to append some code like this

\inputminted[firstline=101,lastline=112]{ocaml}{camlp4/code/jake/myocamlbuild.ml}

The tags file then will be like this 
\begin{bluetext}
<pa_filter.{ml}> : pp(camlp4rf)
<pa_filter.{cmo}> : use_camlp4
<test_filter.ml> : camlp4o, use_filter
\end{bluetext}

Using Register Filter has some limitations, like it first parse the
whole file, and then transform each \verb|structure item| one by one,
so the previously generated code will \verb|not have | an effect on
the later code. This is probably what not you want.


You syntax extension may depends on other modules, make sure your
\verb|pa_xx.cma| contains all the modules statically. You can write a
\verb|pa_xx.mllib|, or link the module to \verb|cma| file by hand.

For instance, you \verb|pa_filter.cma| depends on \verb|Util|, then
you will 

\verb|ocalmc -a pa_filter.cmo util.cmo -o pa_filer.cma|

then you could use \verb|camlp4o -parser pa_filter.cma|, it works.

If you write \verb|pa_xx.mllib| file, it would be something like

\begin{bluetext}
pa_filter
util
\end{bluetext}
If you want to use other libraries to write syntax extension, make
sure you link \textbf{all} libraries, including recursive dependency,
i.e, the require field of batteries.
\begin{bluetext}
ocamlc -a  -I +num -I `ocamlfind query batteries` nums.cma unix.cma
bigarray.cma str.cma batteries.cma pa_filter.cma -o x.cma
\end{bluetext}
You must link all the libraries, even you don't need it at all. This
is the defect of the OCaml compiler.
\verb|-linkall| here links submodules, recursive linking needs you say
it clearly.

We can also test our filter seriously as follows 
\verb|camlp4of -parser _build/filter.cmo filter_test.ml -filter lift -printer o |


By the \textbf{lift filter} you can see its \textbf{internal
  representation}, textual code does not gurantee its correctness, but
the AST representation could gurantee its correctness.


Built in filters as follows :
\begin{enumerate}[(a)]
\item fold map
  \begin{bluetext}
  class x = Camlp4MapGenerator.generated ;
  class x = Camlp4FoldGenerator.generated ;
  \end{bluetext}

\item meta \\ 
  lifting function from a type definition -- these functions are what
  \emph{Camlp4AstLifter uses} to lift the AST, and also how
  \emph{quotations are implemented  }
\item LocationStripper (replace location with Loc.ghost) \\
  might be useful when you compare two asts? YES!
  idea? how to use lifter at toplevel, how to beautify our code,
  without the horribling output? (I mean, the qualified name is horrible)
\item Camlp4Profiler \\
  inserts profiling code
\item Camlp4TrashRemover \\
\item Camlp4ExceptionTracer 
\end{enumerate}

\subsection{Part6 Extensible Parser (moved to extensible parser part)}

\subsection{Part7 Revised Syntax }
  Revised syntax provides more context in the form of extra brackets
  etc. so that antiquotation works more smoothly.
\subsection{Part8, 9 Quotation}
  \begin{enumerate}[(a)]
  \item Quotation.add  quotation\_expander 

\begin{redcode}
se (FILTER _* "expand_fun") "Quotation";;
\end{redcode}

\begin{bluecode}
type 'a expand_fun = Ast.loc -> string option -> string -> 'a
val add : string -> 'a DynAst.tag -> 'a expand_fun -> unit
val find : string -> 'a DynAst.tag -> 'a expand_fun      
\end{bluecode}

other useful functions 
  \begin{bluecode}
type 'a expand_fun = Ast.loc -> string option -> string -> 'a
val add : string -> 'a DynAst.tag -> 'a expand_fun -> unit
val find : string -> 'a DynAst.tag -> 'a expand_fun
val default : string ref  (* default quotations *)
val parse_quotation_result :
      (Ast.loc -> string -> 'a) ->
      Ast.loc -> Camlp4.Sig.quotation -> string -> string -> 'a
val translate : (string -> string) ref
val expand : Ast.loc -> Camlp4.Sig.quotation -> 'a DynAst.tag -> 'a
val dump_file : string option ref
\end{bluecode}

in previous camlp4, Quotation provides a string to string
transformation, then it default uses Syntax.expr or Syntax.patt to
parse the returned string. following drawbacks
\begin{itemize}
\item needs a \textbf{more} parsing phase
\item the resulting string may be syntactically incorrect, difficult
  to \textbf{debug}
\end{itemize}

\item quotation expander \\ 
  when without antiquotaions, a parser is enought, other things are
  quite mechanical

\begin{bluecode}
open Camlp4.PreCast 
module Jq_ast = struct 
  type float' = float 
  type t = 
      Jq_null 
    |Jq_bool of bool 
    |Jq_number of float' 
    |Jq_string of string 
    |Jq_array of t list 
    |Jq_object of (string*t) list 
end
include Jq_ast 
module MetaExpr = struct 
  (** the generator scans all the types defined in the current module
      then generate code for the last-appearing recursive bundle
  *)
  let meta_float' _loc f = <:expr< $`flo:f$ >>
  include Camlp4Filters.MetaGeneratorExpr(Jq_ast)
  (* due to this can not run in toplevel *)
end 
module MetaPatt = struct 
  let meta_float' _loc f = <:patt< $`flo:f$ >>
  include Camlp4Filters.MetaGeneratorPatt(Jq_ast)  
end 
module MGram = MakeGram(Lexer)
let json_parser = MGram.Entry.mk "json" 
  EXTEND MGram 
  GLOBAL : json_parser ; 
  json_parser : 
    [["null" -> Jq_null 
     |"true" -> Jq_bool true
     |"false" -> Jq_bool false 
     | n = [x = INT -> x | y = FLOAT -> y  ] -> Jq_number (float_of_string n )
     | s = STRING -> Jq_string s 
     | "["; xs = LIST0 SELF SEP "," ; "]" -> Jq_array xs 
     | "{"; kvs = LIST0 [s = STRING; ":"; v = json_parser -> (s,v)] SEP ","; 
       "}" -> Jq_object kvs 
     ]] ; END 
let json_eoi = MGram.Entry.mk "json_eoi"  
  EXTEND MGram 
  GLOBAL: json_eoi ; 
  json_eoi : [[x = json_parser ; EOI -> x ]] ; END 
let test = 
  MGram.parse_string json_eoi (Loc.mk "<string>") 
    "[true,false]"
\end{bluecode}
  
Mechanical installation to get a quotation expander 
\begin{redcode}
module Q = Syntax.Quotation 
(* #directory "/Users/bob/SourceCode/OCaml/Parsing/camlp4/_build";; *)
(* camlp4of -filter meta json.ml -printer o *)
let (|>) x f = f x 
let parse_quot_string _loc s = 
  MGram.parse_string  json_eoi _loc s 
let expand_expr _loc _ s = 
  s 
  |> parse_quot_string _loc 
  |> MetaExpr.meta_t _loc 

(* to make it able to appear in the toplevel *)
let expand_str_item _loc _ s = 
  (**insert an expression as str_item *)
   <:str_item@_loc< $exp: expand_expr _loc None s $ >>
let expand_patt _loc _ s  = 
  s 
  |> parse_quot_string _loc 
  |> MetaPatt.meta_t _loc 

let _  = 
  Q.add "json" Q.DynAst.expr_tag expand_expr ;
  Q.add "json" Q.DynAst.patt_tag expand_patt ;
  Q.add "json" Q.DynAst.str_item_tag expand_str_item ;
  Q.default := "json"

(** make quotation from a parser *)
let install_quotation my_parser (me,mp) name =
  let module Q = Syntax.Quotation in 
  let expand_expr _loc _ s = s |>  my_parser _loc |> me _loc in
  let expand_str_item _loc _ s =  <:str_item@_loc< $exp: expand_expr
  _loc None s $>> in
  let expand_patt _loc _ s = s |> my_parser _loc |> mp _loc in
  Q.add name Q.DynAst.expr_tag expand_expr ;
  Q.add "json" Q.DynAst.patt_tag expand_patt ;
  Q.add "json" Q.DynAst.str_item_tag expand_str_item 

\end{redcode}
\begin{bluetext}
val install_quotation :
  (Camlp4.PreCast.Ast.loc -> string -> 'a) ->
  (Camlp4.PreCast.Ast.loc -> 'a -> Camlp4.PreCast.Ast.expr) *
  (Camlp4.PreCast.Ast.loc -> 'a -> Camlp4.PreCast.Ast.patt) -> string -> unit =
  <fun>  
\end{bluetext}
\begin{bluecode}
"json.ml" : pp(camlp4of -filter meta)
<json.{cmo,byte,native}> : pkg_dynlink, use_camlp4_full
\end{bluecode}
so in the toplevel

\begin{redcode}
#directory "/Users/bob/SourceCode/OCaml/Parsing/camlp4/_build";;
#load "json.cmo" ;
open Json;  (* for Jq_ast module, you can find other ways to work
around this *)
\end{redcode}

\begin{alternate}
 << [ 3 ,4 ]>>;;
- : Json.Jq_ast.t = Json.Jq_ast.Jq_array [Json.Jq_ast.Jq_number 3.; Json.Jq_ast.Jq_number
4.]
\end{alternate}


\item antiquotation expander \\

  the meta filter treat any other constructor \textbf{ending in Ant}
specially 

instead of 
\begin{bluecode}
  |Jq_Ant(loc,s) -> <:expr< Jq_Ant ($meta_loc loc$, $meta_string s$) >>
\end{bluecode}
they have
\begin{redcode}
  |Jq_Ant(loc,s) -> ExAnt(loc,s) 
\end{redcode}

Instead of lifting the constructor, they translate it directly to
ExAnt or PaAnt.

\textbf{Attention, there is no semi or comma required in GLOBAL list,
  GLOBAL: json\_eoi  json ; (just whitespace ) }


  \begin{bluecode}
open Camlp4.PreCast  
module Jq_ast = struct 
  type float' = float 
  type t = 
      Jq_null 
    |Jq_bool of bool 
    |Jq_number of float' 
    |Jq_string of string 

    |Jq_array of t  
    |Jq_object of t 
    |Jq_colon of t * t  (* to make an object *)
    |Jq_comma of t * t  (* to make an array *)
    |Jq_Ant of Loc.t * string 
    |Jq_nil (* similiar to StNil *)
  let rec t_of_list lst = match lst with 
    |[] -> Jq_nil 
    | b::bs -> Jq_comma (b, t_of_list bs)
end 

include Jq_ast 

module MGram = MakeGram(Lexer) 

let json  = MGram.Entry.mk "json"
let json_eoi = MGram.Entry.mk "json_eoi" 


EXTEND  MGram 
  GLOBAL: json_eoi  json ; 
  json_eoi : [[ x = json ; EOI -> x ]]; 

  json : 
    [[ "null" -> Jq_null 
     |"true" -> Jq_bool true 
     |"false" -> Jq_bool false 

     | `ANTIQUOT (""|"bool"|"int"|"floo"|"str"|"list"|"alist" as n , s) -> 
       Jq_Ant(_loc, n ^ ": " ^ s )

     | n = [ x = INT-> x | x =  FLOAT -> x ] -> Jq_number (float_of_string n)
     | "["; es = SELF ; "]" -> Jq_array es 
     | "{";  kvs = SELF ;"}" -> Jq_object kvs 

     | k= SELF; ":" ; v = SELF -> Jq_colon (k, v)
     | a = SELF; "," ; b = SELF -> Jq_comma (a, b)
     | -> Jq_nil  (* camlp4 parser epsilon has a lower priority *)

     ]];
END ;;

module AQ = Syntax.AntiquotSyntax 
module Q = Syntax.Quotation 
let destruct_aq s = 
  let pos = String.index s ':' in 
  let len = String.length s in 
  let name = String.sub s 0 pos in
  let code = String.sub s (pos+1) (len-pos-1) in 
  name, code

(** alternative*)  
let destruct_aq2  = function (RE (_* Lazy as  name ) ":" (_* as content)) -> name,content;;
\end{bluecode}

\begin{alternate}
let /(_* Lazy as x) ":" (_* as rest ) / = "ghsoghos:ghsogh: ghsohgo";;
val rest : string = "ghsogh: ghsohgo"

val x : string = "ghsoghos"  
\end{alternate}

\begin{bluecode}
let try /(_* Lazy as x) ":" (_* as rest ) / = "ghsoghosghsog ghsohgo"
in (x,rest)
with Match_failure _ -> ("","");;  
\end{bluecode}
notice that  Syntax.AntiquotSyntax.(parse\_expr,parse\_patt)
Syntax.(parse\_implem, parse\_interf)

\begin{bluecode}
        val parse_expr : Ast.loc -> string -> Ast.expr
        val parse_patt : Ast.loc -> string -> Ast.patt
    val parse_implem :
    val parse_interf :
\end{bluecode}

\begin{bluecode}
let aq_expander = object 
  inherit Ast.map as super 
  method expr = function 
    |Ast.ExAnt(_loc, s) -> 
      let n, c = destruct_aq s in 
      (** first round*)
      let e = AQ.parse_expr _loc c in 
      begin match n with 
        |"bool" -> <:expr< Jq_ast.Jq_bool $e$ >> (* interesting *)
        |"int" -> <:expr< Jq_ast.Jq_number (float $e$ ) >>
        |"flo" -> <:expr< Jq_ast.Jq_number $e$ >>
        |"str" -> <:expr< Jq_ast.Jq_string $e$ >>
        | "list" -> <:expr< Jq_ast.t_of_list $e$ >>
        |"alist" -> 
          <:expr<
            Jq_ast.t_of_list 
            (List.map (fun (k,v) -> Jq_ast.Jq_colon (Jq_ast.Jq_string k, v))
            $e$ )
          >>
        |_ -> e 
      end 
    |e -> super#expr e 
  method patt = function 
    | Ast.PaAnt(_loc,s) -> 
      let n,c = destruct_aq s in 
      AQ.parse_patt _loc c  (* ignore the tag *)
    | p -> super#patt p 
end 
module MetaExpr = struct 
  (** the generator scans all the types defined in the current module
      then generate code for the last-appearing recursive bundle
  *)
  let meta_float' _loc f = <:expr< $`flo:f$ >>
  include Camlp4Filters.MetaGeneratorExpr(Jq_ast)
end 
module MetaPatt = struct 
  let meta_float' _loc f = <:patt< $`flo:f$ >>
  include Camlp4Filters.MetaGeneratorPatt(Jq_ast)  
end 
let (|>) x f = f x 
let parse_quot_string _loc s = 
  let q = !Camlp4_config.antiquotations in 
  (** checked by the lexer to allow antiquotation 
      the flag is initially set to false, so antiquotations 
      appearing outside a quotation won't be parsed 
      *)
Camlp4_config.antiquotations := true ; 
let res =  MGram.parse_string  json_eoi _loc s in 
 Camlp4_config.antiquotations := q ; 
 res 
let expand_expr _loc _ s = 
  s 
  |> parse_quot_string _loc 
  |> MetaExpr.meta_t _loc 
  |> aq_expander#expr 
(* so it can appear in the toplevel *)
let expand_str_item _loc _ s = 
  (**insert an expression as str_item *)
   <:str_item@_loc< $exp: expand_expr _loc None s $ >>
let expand_patt _loc _ s  = 
  s 
  |> parse_quot_string _loc 
  |> MetaPatt.meta_t _loc 
  |> aq_expander#patt 
let _  = 
  Q.add "json" Q.DynAst.expr_tag expand_expr ;
  Q.add "json" Q.DynAst.patt_tag expand_patt ;
  Q.add "json" Q.DynAst.str_item_tag expand_str_item ;
  Q.default := "json"

\end{bluecode}
\begin{alternate}
MGram.parse_string json_eoi Loc.ghost "[1,2]";; 
 - : t = Jq_array (Jq_comma (Jq_number 1., Jq_number 2.)) 
MGram.parse_string json_eoi Loc.ghost "[1,2,]";;
- : t = Jq_array (Jq_comma (Jq_comma (Jq_number 1., Jq_number 2.), Jq_nil))
MGram.parse_string json_eoi Loc.ghost "1,2";;
- : t = Jq_comma (Jq_number 1., Jq_number 2.)
let alist = ["haha", <<1>>;"bob",<<3>>]  in <:json< [1 , $alist:alist$ ]>>;;
\end{alternate}

\begin{bluecode}
- : Json_anti.Jq_ast.t =
Json_anti.Jq_ast.Jq_array
 (Json_anti.Jq_ast.Jq_comma (Json_anti.Jq_ast.Jq_number 1.,
   Json_anti.Jq_ast.Jq_comma
    (Json_anti.Jq_ast.Jq_colon (Json_anti.Jq_ast.Jq_string "haha",
      Json_anti.Jq_ast.Jq_number 1.),
    Json_anti.Jq_ast.Jq_comma
     (Json_anti.Jq_ast.Jq_colon (Json_anti.Jq_ast.Jq_string "bob",
       Json_anti.Jq_ast.Jq_number 3.),
     Json_anti.Jq_ast.Jq_nil))))
   \end{bluecode}

\begin{alternate}   
let b = << $ << 1 >>  $ >>  = << 1 >>;;
val b : bool = true
\end{alternate}

\begin{bluetext}
<< $ << 1 >> $>> --> parsing (my parser)
Jq_Ant(_loc, "<< 1 >> ") --> lifting  (mechnical)
Ex_Ant(_loc, "<< 1 >>") --> parsing  (the host parser )
<:expr< Jq_number 1. >>   --> antiquot_expand (my anti_expander )
<:expr < Jq_number 1. >> 
*)
"json_anti.ml" : pp(camlp4of -filter meta)
<json_anti.{cmo,byte,native}> : pkg_dynlink, use_camlp4_full
  \end{bluetext}

\end{enumerate}


\subsection{ Part 10 Lexer }
  Just follow the signature of module type Lexer is enough.
  generally you have to provide module
  Loc, Token, Filter, Error, and mk
  mk is essential

  \begin{bluecode}
val mk : unit -> Loc.t -> char Stream.t -> (Token.t * Loc.t ) Stream.t     
  \end{bluecode}

  the verbose part lies in that you have to use the Camlp4.Sig.Loc,
  usually you have to maintain a mutable context, so when you lex a
  token, you can query the context to get Loc.t. you can refer Jake's jq\_lexer.ml
  for more details. How about using lexer, parser all by myself?
  The work need to be done lies in you have to supply a plugin of type
  expand\_fun, which is \\
  \verb|type 'a expand_fun = Ast.loc -> string option -> string -> 'a|
  so if you dont use ocamllexer, why bother the grammar module, just
  use lex yacc will make life easier, and you code will run faster . 

\begin{bluecode}
type pos = {
  line : int;
  bol  : int;
  off  : int
};
type t = {
  file_name : string;
  start     : pos;
  stop      : pos;
  ghost     : bool
};
open Camlp4.PreCast 
module Loc = Camlp4.PreCast.Loc 
module Error : sig 
  type t 
  exception E of t 
  val to_string : t -> string 
  val print : Format.formatter -> t -> unit 
end =  struct
  type t = string 
  exception E of string 
  let print = Format.pp_print_string (* weird, need flush *)
  let to_string  x  =  x
end
let _ = 
  let module M = Camlp4.ErrorHandler.Register (Error) in ()
let (|> ) x  f = f x 
module Token : sig 
  module Loc : Camlp4.Sig.Loc 
  type t 
  val to_string : t -> string 
  val print : Format.formatter -> t -> unit 
  val match_keyword : string -> t -> bool 
  val extract_string : t -> string 
  module Filter : sig 
    (* here t refers to the Token.t *)
    type token_filter = (t,Loc.t) Camlp4.Sig.stream_filter 
    type t 
    val mk : (string->bool)-> t 
    val define_filter : t -> (token_filter -> token_filter) -> unit 
    val filter : t -> token_filter 
    val keyword_added : t -> string -> bool -> unit 
    val keyword_removed : t -> string -> unit 
  end
  module Error : Camlp4.Sig.Error  
end = struct 
  (** the token need not to be a variant with arms with KEYWORD
      EOI, etc, although conventional
  *)
  type t = 
    | KEYWORD of string 
    | NUMBER of string 
    | STRING of string 
    | ANTIQUOT of string * string 
    | EOI
  let to_string t = 
    let p = Printf.sprintf in 
    match t with 
      |KEYWORD s -> p "KEYWORD %S" s 
      |NUMBER s -> p "NUMBER %S" s 
      |STRING s -> p "STRING %S" s 
      |ANTIQUOT (n,s) -> p "ANTIQUOT %S: %S" n s 
      |EOI -> p "EOI"
  let print fmt x = x |> to_string |> Format.pp_print_string fmt 
  let match_keyword kwd = function 
    |KEYWORD k when kwd = k -> true 
    |_ -> false 

  let extract_string = function 
    |KEYWORD s | NUMBER s | STRING s -> s 
    |tok -> invalid_arg ("can not extract a string from this token : "
                         ^ to_string tok)

  module Loc = Camlp4.PreCast.Loc 
  module Error = Error 
  module Filter = struct 
    type token_filter = (t * Loc.t ) Stream.t -> (t * Loc.t) Stream.t 

    (** stub out *)    
    (** interesting *)
    type t = unit 

    (** the argument to mk is a function indicating whether 
        a string should be treated as a keyword, and the default 
        lexer uses it to filter the token stream to convert identifiers
        into keywords. if we want our parser to be extensible, we should
        take this into account 
    *)
    let mk _ = ()
    let filter _ x  = x
    let define_filter _ _ = ()
    let keyword_added _ _ _ = ()
    let keyword_removed _ _ = ()
  end 
end 
module L = Ulexing 
INCLUDE "/Users/bob/predefine_ulex.ml" 
(* let rec token  c = lexer  *)
(*   | eof -> EOI  *)
(*   | newline -> token *)
(** TOKEN ERROR LOC 
    mk : unit -> Loc.t -> char Stream.t -> (Token.t * Loc.t) Stream.t

    Loc.of_tuple : 
    string * int * int * int * int * int * int * bool -> 
    Loc.t
*)
    
  \end{bluecode}


%%% Local Variables: 
%%% mode: latex
%%% TeX-master: "../master"
%%% End: 

  
\section{Useful links}
  \href{http://brion.inria.fr/gallium/index.php/Abstract\_Syntax\_Tree}{Abstract\_Syntax\_Tree} \\
  \href{http://elehack.net/michael/blog/2010/06/ocaml-syntax-extension}{elehack} \\
  \href{http://andreiformiga.com/blog/?p=99}{meta-guide} \\
  \href{http://www.wisdomandwonder.com/link/5302/resources-for-learning-camlp4}{camlp4} \\
  \href{http://www.pps.jussieu.fr/~li/software/}{zheng.li}\\
  \href{http://alaska-kamtchatka.blogspot.com/2010/08/what-can-pado-for-you.html}{pa-do}\\

  \href{http://brion.inria.fr/gallium/index.php/Category:Camlp4}{Wiki}\\
  



\chapter{practical parts}
\section{batteries}
\paragraph{syntax extension}

Not of too much use , \textbf{Never use it in the toplevel}
\begin{itemize}
\item comprehension (M.filter, concat, map, filter\_map, enum, of\_enum) \\
  since it's at preproccessed stage, you can use some trick \\
  \verb|let module Enum = List in | will change the semantics \\
  \verb|let open Enum in| doesn't make sense, since it uses qualified name inside
\end{itemize}
\subsection{Dev}
\begin{itemize}
\item make changes in both .ml and .mli files
\end{itemize}

\subsection{BOLT}
\label{sec:bolt}

\section{Mikmatch}
\label{sec:mikmatch}
Directly supported in toplevel
Regular expression \emph{share} their own namespace.
\begin{enumerate}
\item compile
\begin{bluetext}
"test.ml" : pp(camlp4o -parser pa_mikmatch_pcre.cma)
<test.{cmo,byte,native}> : pkg_mikmatch_pcre
-- myocamlbuild.ml use default 
\end{bluetext}
\item toplevel
\begin{redcode}
ocaml
#camlp4o ;;
#require "mikmatch_pcre" ;; (* make sure to follow the order strictly *)
\end{redcode}
\item debug

  \begin{bluetext}
camlp4of -parser pa_mikmatch_pcre.cma -printer o test.ml
(* -no_comments does not work    *)
\end{bluetext}

\item structure \\
  regular expressions can be used to match strings, it must be preceded by
  the RE keyword, or placed between slashes (/../).

  \begin{bluecode}
    match ... with pattern -> ...
    function pattern -> ...
    try ... with pattern -> ... 
    let /regexp/ = expr in expr
    let try (rec) let-bindings in expr with pattern-match
    (only handles exception raised by let-bindings)
    MACRO-NAME regexp -> expr ((FILTER | SPLIT) regexp)
    
  \end{bluecode}

  \begin{alternate}
let x = (function (RE digit+) -> true | _ -> false) "13232";;
val x : bool = true
# let x = (function (RE digit+) -> true | _ -> false) "1323a2";;
val x : bool = true
# let x = (function (RE digit+) -> true | _ -> false) "x1323a2";;
val x : bool = false    
\end{alternate}

\begin{bluecode}
let get_option () = match Sys.argv with 
     [| _ |] -> None 
    |[| _ ; RE (lower+ as key) "=" (_* as data) |] -> Some(key,data)
    |_ -> failwith "Usage: myprog [key=val]";;
val get_option : unit -> (string * string) option = <fun>
\end{bluecode}

\begin{alternate}
let option = try get_option () with Failure (RE "usage"~) -> None ;;
val option : (string * string) option = None  
\end{alternate}


\item \textbf{sample regex}
  built in regexes
  \begin{bluetext}
    lower, upper, alpha(lower|upper), digit, alnum, punct
    graph(alnum|punct), blank,cntrl,xdigit,space
    int,float
    bol(beginning of line)
    eol
    any(except newline)
    bos, eos
  \end{bluetext}
  \begin{alternate}
let f = (function (RE int as x : int) -> x ) "132";;
val f : int = 132
let f = (function (RE float as x : float) -> x ) "132.012";;
val f : float = 132.012
let f = (function (RE lower as x ) -> x ) "a";;
val f : string = "a"
let src = RE_PCRE int ;;
val src : string * 'a list = ("[+\\-]?(?:0(?:[Xx][0-9A-Fa-f]+|(?:[Oo][0-7]+|[Bb][01]+))|[0-9]+)", [])
let x = (function (RE _* bol "haha") -> true | _ -> false) "x\nhaha";;
val x : bool = true
\end{alternate}

\begin{bluecode}
RE hello = "Hello!" 
RE octal  = ['0'-'7']
RE octal1 = ["01234567"]
RE octal2 = ['0' '1' '2' '3' '4' '5' '6' '7']
RE octal3 = ['0'-'4' '5'-'7']
RE octal4 = digit # ['8' '9']  (* digit is a predefined set of characters *)
RE octal5 = "0" | ['1'-'7']
RE octal6 = ['0'-'4'] | ['5'-'7']
RE not_octal = [ ^ '0'-'7'] (* this matches any character but an octal digit *)
RE not_octal' = [ ^ octal]  (* another way to write it *)
\end{bluecode}

\begin{redcode}
RE paren' = "(" _* Lazy ")"
(* _ is wild pattern, paren is built in *)
let p = function (RE (paren' as x )) -> x ;;
\end{redcode}

\begin{alternate}
p "(xx))";;
- : string = "(xx)"
# p "(x)x))";;
- : string = "(x)"
\end{alternate}

\begin{bluecode}
RE anything  = _*         (* any string, as long as possible *)
RE anything' = _* Lazy    (* any string, as short as possible *)
RE opt_hello  = "hello"?      (* matches hello if possible, or nothing *)
RE opt_hello' = "hello"? Lazy (* matches nothing if possible, or hello *)
RE num = digit+        (* a non-empty sequence of digits, as long as possible;
                          shortcut for: digit digit* *)
RE lazy_junk = _+ Lazy (* match one character then match any sequence
                          of characters and give up as early as possible *)

RE at_least_one_digit = digit{1+}     (* same as digit+ *)
RE at_least_three_digits = digit{3+}
RE three_digits = digit{3}
RE three_to_five_digits = digit{3-5}
RE lazy_three_to_five_digits = digit{3-5} Lazy

let test s = match s with 
    RE "hello" -> true 
  | _ -> false 
\end{bluecode}


It's important to know that matching process will try \textit{any} possible combination until
the pattern is matched. However the combinations are tried from left to right, and
repeats are either greedy or lazy. (greedy is default). laziness triggered by the presence
of the Lazy keyword.

\item fancy features of regex
  \begin{enumerate}[(a)]
  \item normal

    \begin{redcode}
let x = match "hello world" with
  RE "world" -> true
 | _ -> false;;
\end{redcode}

    \begin{bluecode}
val x : bool = false
  \end{bluecode}

\item pattern match syntax 
  (the let constructs can be used directly with a
  regexp pattern, but \textbf{let RE ... = ... }does not look nice, the
  sandwich notation (/.../) has been introduced )

  \begin{alternate}
Sys.ocaml_version;;
- : string = "3.12.1"
# RE num = digit + ;;
\end{alternate}

\begin{redcode}

RE num = digit + ;;
  
let  /(num as major : int ) "." (num as minor : int)

( "." (num as patchlevel := fun s -> Some (int_of_string s)) 
| ("" as patchlevel := fun s -> None ))

( "+" (_*  as additional_info := fun s -> Some s )
| ("" as additional_info  := fun s -> None )) eos

/ = Sys.ocaml_version ;;

\end{redcode}

we always use \textbf{as} to extract the information.      

\begin{bluecode}      
val additional_info : string option = None
val major : int = 3
val minor : int = 12
val patchlevel : int option = Some 1    
\end{bluecode}


\item File processing (Mikmatch.Text)

  \begin{bluecode}
    val iter_lines_of_channel : (string -> unit) -> in_channel -> unit
    val iter_lines_of_file : (string -> unit) -> string -> unit
    val lines_of_channel : in_channel -> string list
    val lines_of_file : string -> string list
    val channel_contents : in_channel -> string
    val file_contents : ?bin:bool -> string -> string
    val save : string -> string -> unit
    val save_lines : string -> string list -> unit
    exception Skip
    val map : ('a -> 'b) -> 'a list -> 'b list
    val rev_map : ('a -> 'b) -> 'a list -> 'b list
    val fold_left : ('a -> 'b -> 'a) -> 'a -> 'b list -> 'a
    val fold_right : ('a -> 'b -> 'b) -> 'a list -> 'b -> 'b
    val map_lines_of_channel : (string -> 'a) -> in_channel -> 'a list
    val map_lines_of_file : (string -> 'a) -> string -> 'a list
\end{bluecode}
\item \textbf{Mikmatch.Glob} (pretty useful)

  \begin{bluecode}
    val scan :
      ?absolute:bool ->
      ?path:bool ->
      ?root:string ->
      ?nofollow:bool -> (string -> unit) -> (string -> bool) list -> unit
    val lscan :
      ?rev:bool ->
      ?absolute:bool ->
      ?path:bool ->
      ?root:string list ->
      ?nofollow:bool ->
      (string list -> unit) -> (string -> bool) list -> unit
    val list :
      ?absolute:bool ->
      ?path:bool ->
      ?root:string ->
      ?nofollow:bool -> ?sort:bool -> (string -> bool) list -> string list
    val llist :
      ?rev:bool ->
      ?absolute:bool ->
      ?path:bool ->
      ?root:string list ->
      ?nofollow:bool ->
      ?sort:bool -> (string -> bool) list -> string list list
    \end{bluecode}

    here we want to get  \verb|~/.*/*.conf| file
    X.list (predicates corresponding to each layer .
    \begin{alternate}
let xs = let module X = Mikmatch.Glob in X.list ~root:"/Users/bob" [FILTER "." ; FILTER _* ".conf" eos ] ;;
val xs : string list = [".libfetion/libfetion.conf"]
\end{alternate}

\begin{redcode}
let xs =
  let module X = Mikmatch.Glob in
  X.list ~root:"/Users/bob" [const true; FILTER _* ".pdf" eos ]
  in print_int (List.length xs) ;;
\end{redcode}
\begin{bluecode}
455
\end{bluecode}


\item Lazy or Greedy 

  \begin{redcode}
match "acbde (result), blabla... " with 
RE _* "(" (_* as x) ")" -> print_endline x | _ -> print_endline "Failed";;
\end{redcode}
\begin{bluecode}
result
\end{bluecode}

\begin{redcode}
 match "acbde (result),(bla)bla... " with 
 RE _* Lazy "(" (_* as x) ")" -> print_endline x | _ -> print_endline "Failed";;
\end{redcode}
\begin{bluecode}
result),(bla
\end{bluecode}

\begin{alternate}
let / "a"? ("b" | "abc" ) as x / = "abc" ;; (* or patterns, the same as before*)
val x : string = "ab"
# let / "a"? Lazy ("b" | "abc" ) as x / = "abc" ;;
val x : string = "abc"
\end{alternate}

In place conversions of the substrings can be performed, using
either the predefined converters \textit{int, float}, or custom converters

\begin{alternate}
let z  = match "123/456" with RE (digit+ as x : int ) "/" (digit+ as y : int) -> x ,y ;;
val z : int * int = (123, 456)
\end{alternate}

Mixed pattern
\begin{alternate}
let z = match 123,45, "6789" with i,_, (RE digit+ as j : int) | j,i,_ -> i * j + 1;;
val z : int = 835048
\end{alternate}

\item Backreferences \\
  Previously matched substrings can be matched again using backreferences.

  \begin{alternate}
let z =  match "abcabc" with RE _* as x !x -> x ;;
val z : string = "abc"    
\end{alternate}

\item Possessiveness prevent backtracking 

  \begin{alternate}
let x = match "abc" with RE _* Possessive _ -> true | _ -> false;;
val x : bool = false    
  \end{alternate}

\item macros
  \begin{enumerate}

\item FILTER macro 
  \begin{alternate}
let f = FILTER int eos;;
val f : ?share:bool -> ?pos:int -> string -> bool = <fun>
# f "32";;
- : bool = true
# f "32a";;
- : bool = false    
\end{alternate}

\item REPLACE macro 
  \begin{alternate}
let remove_comments = REPLACE "#" _* Lazy eol -> "" ;;
val remove_comments : ?pos:int -> string -> string = <fun>
# remove_comments "Hello #comment \n world #another comment" ;;
- : string = "Hello \n world "
let x = (REPLACE "," -> ";;" ) "a,b,c";;
val x : string = "a;;b;;c"
\end{alternate}

\item REPLACE\_FIRST macro
\item SEARCH(\_FIRST) COLLECT COLLECTOBJ MACRO

  \begin{alternate}
let search_float = SEARCH_FIRST float as x : float -> x ;;
val search_float : ?share:bool -> ?pos:int -> string -> float = <fun>
search_float "bla bla -1.234e12 bla";;
- : float = -1.234e+12
let get_numbers = COLLECT float as x : float -> x ;;
val get_numbers : ?pos:int -> string -> float list = <fun>
get_numbers "1.2   83  nan  -inf 5e-10";;
- : float list = [1.2; 83.; nan; neg_infinity; 5e-10]
let read_file = Mikmatch.Text.map_lines_of_file (COLLECT float as x : float -> x );;
val read_file : string -> float list list = <fun>

(** Negative assertions *)
let get_only_numbers =  COLLECT < Not alnum . > (float as x : float) < . Not alnum > -> x

let list_words = COLLECT (upper | lower)+ as x -> x ;;
val list_words : ?pos:int -> string -> string list = <fun>
# list_words "gshogh sghos sgho ";;
- : string list = ["gshogh"; "sghos"; "sgho"]
RE pair = "(" space* (digit+ as x : int) space* ","  space* ( digit + as y : int ) space* ")";;
 # let get_objlist = COLLECTOBJ pair;;
val get_objlist : ?pos:int -> string -> < x : int; y : int > list =
  \end{alternate}  
\item SPLIT macro
  \begin{alternate}
let ys = (SPLIT space* [",;"] space* ) "a,b,c, d, zz;";;
val ys : string list = ["a"; "b"; "c"; "d"; "zz"]
let f = SPLIT space* [",;"] space* ;;
val f : ?full:bool -> ?pos:int -> string -> string list = <fun>
\end{alternate}

Full is false by default. When true, it considers the regexp
as a separator between substrings even if the first or the last one
is empty. will add some whitespace trailins
\begin{alternate}
f ~full:true "a,b,c,d;"  ;;
- : string list = ["a"; "b"; "c"; "d"; ""]
\end{alternate}
\item MAP macro (a weak lexer) (MAP regexp -> expr ) \\
  splits the given string into fragments: the fragments that do not match the pattern are returned as \textit{`Text s}. Fragments that match the pattern are replaced by the result of expr 

\begin{alternate}
let f = MAP ( "+" as x = `Plus ) -> x ;;
val f : ?pos:int -> ?full:bool -> string -> [> `Plus | `Text of string ] list =
let x =  (MAP ',' -> `Sep ) "a,b,c";;
val x : [> `Sep | `Text of string ] list = [`Text "a"; `Sep; `Text "b"; `Sep; `Text "c"]
\end{alternate}

\begin{redcode}
let f = MAP ( "+" as x = `Plus ) | ("-" as x = `Minus) | ("/" as x = `Div)
  | ("*" as x = `Mul) | (digit+ as x := fun s -> `Int (int_of_string s)) 
  | (alpha [alpha digit] +  as x := fun s -> `Ident s) -> x ;;
\end{redcode}

\begin{bluecode}
val f :
  ?pos:int ->
  ?full:bool ->
  string ->
  [> `Div
   | `Ident of string
   | `Int of int
   | `Minus
   | `Mul
   | `Plus
   | `Text of string ]
list = <fun>
\end{bluecode}
\begin{redcode}
# f "+-*/";;
\end{redcode}

\begin{bluecode}
- : [> `Div
     | `Ident of string
     | `Int of int
     | `Minus
     | `Mul
     | `Plus
     | `Text of string ]
    list
=
[`Text ""; `Plus; `Text ""; `Minus; `Text ""; `Mul; `Text ""; `Div; `Text ""]
\end{bluecode}

\begin{bluecode}
let xs = Mikmatch.Text.map (function `Text (RE space* eos) -> raise Mikmatch.Text.Skip | token -> token) (f "+-*/");;
val xs :
  [> `Div
   | `Ident of string
   | `Int of int
   | `Minus
   | `Mul
   | `Plus
   | `Text of string ]
  list = [`Plus; `Minus; `Mul; `Div]
\end{bluecode}


\item lexer (ulex is faster and more elegant)

  \begin{bluecode}
let get_tokens = f |- Mikmatch.Text.map (function `Text (RE space* eos)
-> raise Mikmatch.Text.Skip | `Text x -> invalid_arg x | x
-> x) ;;

val get_tokens :
  string ->
  [> `Div
   | `Ident of string
   | `Int of int
   | `Minus
   | `Mul
   | `Plus
   | `Text of string ]
  list = <fun>

get_tokens "a1+b3/45";;
- : [> `Div
     | `Ident of string
     | `Int of int
     | `Minus
     | `Mul
     | `Plus
     | `Text of string ]
    list
= [`Ident "a1"; `Plus; `Ident "b3"; `Div; `Int 45]  
\end{bluecode}

\item SEARCH macro (location)

  \begin{alternate}
let locate_arrows = SEARCH %pos1 "->" %pos2 ->  Printf.printf "(%i-%i)" pos1 (pos2-1);;
val locate_arrows : ?pos:int -> string -> unit = <fun>
# locate_arrows "gshogho->ghso";;
(7-8)- : unit = ()
let locate_tags = SEARCH "<" "/"? %tag_start (_* Lazy as tag_contents) %tag_end ">" -> Printf.printf "%s %i-%i" tag_contents tag_start (tag_end-1);;
\end{alternate}

\end{enumerate}

\item debug 
  \begin{alternate}
let src = RE_PCRE <Not alnum . > (float as x : float ) < . Not alnum > in print_endline (fst src);;
(?<![0-9A-Za-z])([+\-]?(?:(?:[0-9]+(?:\.[0-9]*)?|\.[0-9]+)(?:[Ee][+\-]?[0-9]+)?|(?:[Nn][Aa][Nn]|[Ii][Nn][Ff])))(?![0-9A-Za-z])
\end{alternate}




\item ignore the case
  \begin{alternate}
match "OCaml" with RE "O" "caml"~ -> print_endline "success";;
success    
\end{alternate}


\item zero-width assertions

  \begin{bluecode}
RE word =  < Not alpha . >    alpha+ < . Not alpha>
RE word' = < Not alpha . >    alpha+ < Not alpha >
\end{bluecode}

\begin{redcode}
RE triplet = <alpha{3} as x>
let print_triplets_of_letters = SEARCH triplet -> print_endline x
print_triplets_of_letters "helhgoshogho";;
\end{redcode}
\begin{bluecode}
hel
elh
lhg
hgo
gos
osh
sho
hog
ogh
gho
- : unit = ()
\end{bluecode}
\begin{redcode}
(SEARCH alpha{3} as x -> print_endline x ) "hello world";;
\end{redcode}

\begin{bluecode}
hel
wor
\end{bluecode}
\begin{redcode}
(SEARCH <alpha{3} as x> -> print_endline x ) "hello world";;
\end{redcode}
\begin{bluecode}
hel
ell
llo
wor
orl
rld
\end{bluecode}
\begin{redcode}
(SEARCH alpha{3} as x -> print_endline x ) ~pos:2 "hello world";;
\end{redcode}
\begin{bluecode}
llo
wor
\end{bluecode}





\item dynamic regexp 

  \begin{alternate}
let get_fild x = SEARCH_FIRST @x "=" (alnum* as y) -> y;;
val get_fild : string -> ?share:bool -> ?pos:int -> string -> string = <fun>
# get_fild "age" "age=29 ghos";;
- : string = "29"    
\end{alternate}



\item reuse \\ 
  using macro INCLUDE 

\item view patterns
  
  \begin{bluecode}
let view XY = fun obj -> try Some (obj#x, obj#y) with _ -> None ;;
val view_XY : < x : 'a; y : 'b; .. > -> ('a * 'b) option = <fun>
# let test_orign = function 
   %XY (0,0) :: _ -> true 
  |_ -> false 
;;
      val test_orign : < x : int; y : int; .. > list -> bool = <fun>    


let view Positive = fun x -> x > 0 
let view Negative = fun x -> x <= 0 

let test_positive_coords = function 
  %XY ( %Positive, %Positive ) -> true 
  | _ -> false

  (** lazy pattern is already supported in OCaml *)
let test x = match x with 
    lazy v -> v

type 'a lazy_list = Empty | Cons of ('a * 'a lazy_list lazy_t)
    

let f = fun (Cons (_ , lazy (Cons (_, lazy (Empty)) ) )) -> true ;;
let f = fun %Cons (x1, %Cons (x2 %Empty)) -> true  (* simpler *)
\end{bluecode}


implementation 
let view X = f is translated into:
   let view\_X = f

Similarly, we have local views:
let view X = f in ...

Given the nature of camlp4, this is the simplest solution that allows us to make views available to other modules, since they are just functions, with a standard name. When a view X is encountered in a pattern, it uses the view\_X function. The compiler will complain if doesn't have the right type, but not the preprocessor.

About inline views: since views are simple functions, we could insert functions directly in patterns. I believe it would make the pattern really difficult to read, especially since views are expected to be most useful in already complex patterns.


About completeness checking: our definition of views doesn't allow the compiler to warn against incomplete or redundants pattern-matching. We have the same situation with regexps. What we define here are incomplete or overlapping views, which have a broader spectrum of applications than views which are defined as sum types.

\item tiny use
  \begin{alternate}
  se (FILTER _* "map_lines_of_file" ) "Mikmatch";;
  val map_lines_of_file : (string -> 'a) -> string -> 'a list    
  \end{alternate}


\begin{bluecode}
let _  = Mikmatch.map_lines_of_file  
  (function x -> 
    match x with 
      | RE "\xbegin{bluecode}" -> "\n" ^ x 
      | RE "\xend{bluecode}" -> x ^ ``\n'' 
      | _  -> x  )
  "/Users/bob/SourceCode/Notes/ocaml-hacker.tex"  
  |> List.enum 
  |> File.write_lines "/Users/bob/SourceCode/Notes/ocaml-hacker-back-up.tex";;
\end{bluecode}

\end{enumerate}

\end{enumerate}

%%% Local Variables: 
%%% mode: latex
%%% TeX-master: "../master"
%%% End: 

\section{pa-do}
\label{sec:pa-do}

\section{num}
\begin{itemize}
\item delimited overloading
\end{itemize}

\section{caml-inspect}
\label{sec:caml-inspect}
It's mainly used to debug programs or presentation.
\href{http://lambdamuesli.blogspot.com/}{blog}
\begin{enumerate}
\item usage
  \begin{bluetext}
#require "inspect";;
open Inspect ;;

Sexpr.(dump (test_data ()))
Sexpr.(dump dump) (** can dump any value, including closure *)
Dot.(dump_osx dump_osx)
\end{bluetext}



\item \textit{module Dot}
  \begin{bluetext}
    dump
    dump_to_file
    dump_with_formatter
    dump_osx
  \end{bluetext}
\item \textit{module Sexpr}
  \begin{bluetext}
    dump
    dump_to_file
    dump_with_formatter
  \end{bluetext}

\item principle \\
  OCaml values all share a \textit{common low-level} representation.
  The basic building block that is used by the runtime-system(which is
  written in the C programming languag) to represent any value in the
  OCaml universe is the value type. Values are always
  \textit{word-sized}. A word is either 32 or 64 bits
  wide(\textit{Sys.word\_size})

  A value can either be a pointer to a block of values in the OCaml
  heap, a pointer to an object outside of the heap, or an unboxed
  integer. Naturally, blocks in the heap are garbage-collected.

  To distinguish between unboxed integers and pointers, the system uses
  the least-significant bit of the value as a flag. If the LSB is set,
  the value is unboxed. If the LSB is cleared, the value is a pointer to
  some other region of memory. This encoding also explains why the int
  type in OCaml is only 31 bits wide (63 bits wide on 64 bit platforms).


  Since blocks in the heap are garbage-collected, they have strict
  structure constraints. Information like the tag of a block and its
  size(in words) is encoded in the header of each block.

  There are two categories of blocks with respect to the garbage collector:
  \begin{enumerate}
  \item Structured blocks \\
  May only contain well-formed values, as they are
  recursively traversed by the garbage collector.
  \item Raw blocks    \\
  are not scanned by the garbage collector, and can thus
  contain arbitrary values.
  \end{enumerate}
  Structured blocks have tag values lower than
  \textit{Obj.no\_scan\_tag}, while raw blocks have tags equal or
  greater than \textit{Obj.no\_scan\_tag}.
  

  The type of a block is its tag, which is stored in the block header.(\textit{Obj.tag})

  \begin{redcode}
Obj.(let f ()= repr |- tag in no_scan_tag, f () 0, f () [|1.;2.|], f
() (1,2) ,f ()[|1,2|]);;
\end{redcode}

\begin{bluecode}
- : int * int * int * int * int = (251, 1000, 254, 0, 0)    
\end{bluecode}

\begin{redcode}
se_str "_tag" "Obj";;  
\end{redcode}

\begin{bluecode}
    external tag : t -> int = "caml_obj_tag"
    external set_tag : t -> int -> unit = "caml_obj_set_tag"
    val lazy_tag : int
    val closure_tag : int
    val object_tag : int
    val infix_tag : int
    val forward_tag : int
    val no_scan_tag : int
    val abstract_tag : int
    val string_tag : int
    val double_tag : int
    val double_array_tag : int
    val custom_tag : int
    val final_tag : int
    val int_tag : int
    val out_of_heap_tag : int
    val unaligned_tag : int  
\end{bluecode}

\begin{enumerate}

\item \textit{0 to Obj.no\_scan\_tag-1} 
  A structured block (an array of Caml objects). Each field is a value.
\item \textit{Obj.closure\_tag}: A closure representing a functional value. The
first word is a pointer to a piece of code, the remaining words are
values containing the environment.
\item \textit{Obj.string\_tag}: A character string.
\item \textit{Obj.double\_tag}: A double-precision floating-point number.
\item \textit{Obj.double\_array\_tag}: An array or record of double-precision
floating-point numbers.
\item \textit{Obj.abstract\_tag}: A block representing an abstract datatype.
\item \textit{Obj.custom\_tag}: A block representing an abstract datatype with
  user-defined finalization, comparison, hashing, serialization and
  deserialization functions attached
\item \textit{Obj.object\_tag}: A structured block representing an object. The first
  field is a value that describes the class of the object. The second
  field is a unique object id (see \textit{Oo.id}). The rest of the block
  represents the variables of the object.
\item \textit{Obj.lazy\_tag, Obj.forward\_tag}: These two block types
  are used by the runtime-system to implement lazy-evaluation.
\item \textit{Obj.infix\_tag}: A special block contained within a
  closure block  
\end{enumerate}

\item representation

  For atomic types
  \begin{enumerate}
  \item int, char (ascii code) : Unboxed integer values
  \item float : Blocks with tag \textit{Obj.dobule\_tag}
  \item string : Blocks with tag \textit{Obj.string\_tag}
  \item int32, int64, nativeint : Blocks with \textit{Obj.custom\_tag}
  \end{enumerate}
  For Tuples and records: Blocks with tag 0
  \begin{alternate}
Obj.((1,2) |> repr |> tag);;
- : int = 0    
\end{alternate}
For normal array(except float array), Blocks with tag 0

For Arrays and records of floats: Block with tag
\textit{Obj.double\_array\_tag}

For concrete types, 
\begin{enumerate}
\item Constant ctor : Represented by unboxed integers(0,1,...).
\item Non-Constant ctor: Block with a tag lower than
  \textit{Obj.no\_scan\_tag} that encodes the constructor, numbered in
  order of declaration, starting at 0.
\end{enumerate}

For objects: Blocks with tag \textit{Obj.object\_tag}. The first field
refers to the class of the object and its associated method suite. The
second field contains a unique object ID. The remaining fields are the
instance variables of the object.

For polymorphic variants: Variants are similar to constructed
terms. There are a few differences
\begin{enumerate}
\item Variant constructors are identified by their hash value
\item Non-constant variant constructors are not flattened. They are
  always block of size 2, where the first field is the hash. The
  second field can either contain a single value or a pointer to
  another structured block(just like a tuple)
\end{enumerate}

\end{enumerate}


%%% Local Variables: 
%%% mode: latex
%%% TeX-master: "../master"
%%% End: 

\section{ocamlgraph}
\label{sec:ocamlgraph}

\textit{ocamlgraph} is a sex library which deserve well-documentation.

Check the file \textit{pack.ml}, it provides a succint interface.

\begin{ocamlcode}
  module Digraph = Generic(Imperative.Digraph.AbstractLabeled(I) (I))
  module Graph = Generic(Imperative.Graph.AbstractLabeled(I) (I))
\end{ocamlcode}

The Imperative implementation is in \textit{imperative.ml}, 

A nice trick, to bind open command to use graphviz to open the file,
then it will do the sync up automatically

\inputminted[fontsize=\scriptsize
]{ocaml}{code/graph/dag.ml}
\captionof{listing}{Play with DAG  \label{DAG}}

Different modules have corresponding algorithms
\textit{Graph.Pack} requires its label being integer 



\subsection{Undirected graph}
\label{sec:undirected-graph}

so, as soon as you want to label your vertices with strings and your
edges with floats, you should use functor. Take
\textit{ConcreteLabeled} as an example


  \begin{ocamlcode}
open Graph 
module V = struct
  type t = string
  let compare = Pervasives.compare
  let hash = Hashtbl.hash
  let equal = (=)
end
module G = Imperative.Digraph.Concrete (V)
let g = G.create ()
let _ = G.(begin 
  add_edge g "a" "b";
  add_edge g "a" "c";
  add_edge g "b" "d";
  add_edge g "b" "d"
end )
module Display = struct 
  include G
  let vertex_name v = (V.label v)
  let graph_attributes _ = []
  let default_vertex_attributes _ = []
  let vertex_attributes _ = []
  let default_edge_attributes _ = []
  let edge_attributes _ = []
  let get_subgraph _ = None
end 
module Dot_ = Graphviz.Dot(Display)
let _ = 
  let out = open_out "g.dot" in
  finally (fun _ -> close_out out) (fun g -> 
    let fmt =
      (out |> Format.formatter_of_output) in 
    Dot_.fprint_graph fmt g ) g
  \end{ocamlcode}
  
  It seems that Graphviz.Dot is used to display directed graph, Graphviz.Neato is used to display undirected graph.

  here is a useful example to visualize the output generated by ocamldep.
  \begin{ocamlcode}
open Batteries_uni 
open Graph 
module V = struct
  type t = string
  let compare = Pervasives.compare
  let hash = Hashtbl.hash
  let equal = (=)
end
module StringDigraph = Imperative.Digraph.Concrete (V)
module Display = struct 
  include StringDigraph
  open StringDigraph
  let vertex_name v = (V.label v)
  let graph_attributes _ = []
  let default_vertex_attributes _ = []
  let vertex_attributes _ = []
  let default_edge_attributes _ = []
  let edge_attributes _ = []
  let get_subgraph _ = None
end 

module DisplayG = Graphviz.Dot(Display)


let dot_output g file = 
  let out = open_out file in
  finally (fun _ -> close_out out) (fun g -> 
    let fmt =
      (out |> Format.formatter_of_output) in 
    DisplayG.fprint_graph fmt g ) g


let g_of_edges edges = StringDigraph.(
  let g = create () in 
  let _ = Stream.iter (fun (a,b) -> add_edge g a  b) edges in
  g 
)

let line = "path.ml: Hashtbl Heap List Queue Sig Util"

let edges_of_line line = 
  try 
    let (a::b::res) = 
      Pcre.split ~pat:".ml:" ~max:3  line in 
    let v_a = 
      let _ =  a.[0]<- Char.uppercase a.[0] in 
      a  in 
    let v_bs = 
      (Pcre.split ~pat:"\\s+" b ) |> List.filter (fun x -> x <> "") in 
    let edges = List.map (fun v_b -> v_b, v_a ) v_bs in 
    edges 
  with exn -> invalid_arg ("edges_of_line : " ^ line)

let lines_stream_of_channel chan = Stream.from (fun _ -> 
    try Some (input_line chan) with End_of_file -> None );;

let edges_of_channel chan = Stream.(
  let lines = lines_stream_of_channel chan in 
  let edges = lines |> map (edges_of_line |- of_list) |> concat in 
  edges 
)


let graph_of_channel = edges_of_channel |- g_of_edges 

let _ = 
  let stdin = open_in Sys.argv.(1) in 
  let g = graph_of_channel stdin in begin
  Printf.printf "writing to dump.dot\n";
  dot_output g "dump.dot";
  Printf.printf "finished\n"
  end
\end{ocamlcode}


\end{enumerate}
%%% Local Variables: 
%%% mode: latex
%%% TeX-master: "../master"
%%% End: 



\begin{enumerate}
\item debug \\
  tags file
\begin{bluetext}
  "monad_test.ml" : pp(camlp4o -parser pa_monad.cmo)
  camlp4o -parser pa_monad.cmo monad_test.ml -printer o

  (** filter *)

  let a = perform let b = 3 in  b
  let bind x f = f x 
  let c = perform c <-- 3 ; c 
  (* output
  let a = let b = 3 in b
  let bind x f = f x
  let c = bind 3 (fun c -> c)
  *)



let bind x f = List.concat (List.map f x)
let return x = [x]
let bind2 x f = List.concat (List.map f x)

let c = perform 
    x <-- [1;2;3;4]; 
    y <-- [3;4;4;5]; 
    return (x+y)


let d = perform with bind2 in 
    x <-- [1;2;3;4]; 
    y <-- [3;4;4;5]; 
    return (x+y)

let _ = List.iter print_int c 
let _ = List.iter print_int d 

(*
let bind x f = List.concat (List.map f x)
let return x = [ x ]
let bind2 x f = List.concat (List.map f x)
let c =
  bind [ 1; 2; 3; 4 ]
    (fun x -> bind [ 3; 4; 4; 5 ] (fun y -> return (x + y)))
let d =
  bind2 [ 1; 2; 3; 4 ]
    (fun x -> bind2 [ 3; 4; 4; 5 ] (fun y -> return (x + y)))
let _ = List.iter print_int c
let _ = List.iter print_int d
*)  

\end{bluetext}

\item translation rule \\
  it's simple. \textbf{perform} or \textbf{perform with bind in } then
  it will translate all phrases ending with \textit{;}; \textit{x <--
 me;} will be translated into \textit{me >>= (fun x -> )};
\textit{me;} will be translated into \textit{me >>= (fun \_ -> ... )}
you should refer \textit{pa\_monad.ml} for more details 
\textit{perform with exp1 and exp2 in exp3} uses the first given
expression as bind and the second as match-failure function.
\textit{perform with module Mod in exp } use the function named bind
from module Mod. In addition ues the module's failwith in refutable patterns

\begin{alternate}
  let a = perform with (flip Option.bind) in a <-- Some 3;  b<-- Some 32; Some (a+ b) ;;
  val a : int option = Some 35
\end{alternate}

it will be translated into
\begin{bluetext}
let a =
  flip Option.bind (Some 3)
    (fun a -> flip Option.bind (Some 32) (fun b -> Some (a + b)))
\end{bluetext}
\item ParameterizedMonad \\

\begin{bluecode}
class ParameterizedMonad m where
  return :: a -> m s s a
  (>>=) :: m s1 s2 t -> (t -> m s2 s3 a) -> m s1 s3 a

data Writer cat s1 s2 a = Writer {runWriter :: (a, cat s1 s2)}

instance (Category cat) => ParameterizedMonad (Writer cat) where
  return a = Writer (a,id)
  m >>= k = Writer $ let
    (a,w) = runWriter
    (b,w') = runWriter (k a)
    in (b, w' . w)
    
\end{bluecode}
% $

\begin{bluetext}
  
module State : sig 
  type ('a,'s) t = 's -> ('a * 's)
  val return : 'a -> ('a,'s) t 
  val bind : ('a,'s ) t -> ('a -> ('b,'s) t ) -> ('b,'s) t
  val put : 's -> (unit,'s) t 
  val get :  ('s,'s) t
end = struct 
 type ('a,'s) t = ('s -> ('a * 's))
 let return v = fun s -> (v,s)
 let bind (v : ('a,'s) t) (f : 'a -> ('b,'s) t) : ('b,'s) t = fun s -> 
   let a,s' = v s in 
   let a',s'' = f a s' in 
   (a',s'')
 let put s = fun _ -> (), s 
 let get = fun s -> s,s 
end 


module PState : sig 
  type ('a, 'b, 'c) t = 'b -> 'a * 'c
  val return : 'a -> ('a,'b,'b) t
  val bind : ('b,'a,'c)t -> ('b -> ('d,'c, 'e) t ) -> ('d,'a,'e) t
  val put : 's -> (unit,'b,'s)t
  val get : ('s,'s,'s) t 
end  = struct 
 type ('a,'s1,'s2) t = 's1 -> ('a * 's2)
 let return v = fun s -> (v,s)
 let bind v f = fun s -> 
   let a,s' = v s in 
   let a',s'' = f a s' in 
   (a',s'')
 let put s = fun _ -> (), s 
 let get = fun s -> s,s 
end 
  
\end{bluetext}

\begin{redcode}
let v = State.(perform  x <-- return 1 ; y <-- return 2 ; let _ =
print_int (x+y) in return (x+y) );;
\end{redcode}
\begin{bluecode}
val v : (int, '_a) State.t = <fun>  
\end{bluecode}

\begin{redcode}
let v = State.(perform x <-- return 1 ; y <-- return 2 ; z <-- get ; put (x+y+z) ; 
  z<-- get ; let _ = print_int z in return (x+y+z));;
\end{redcode}
\begin{bluecode}
 val v : (int, int) State.t = <fun>  
\end{bluecode}

\begin{alternate}
  v 3;;
6- : int * int = (9, 6)
\end{alternate}


\begin{redcode}
let v = PState.(perform x <-- return 1 ; y <-- return 2 ; z <-- get ; put (x+y+z) ; 
z<-- get ; let _ = print_int z in return (x+y+z));;
\end{redcode}

\begin{bluecode}
val v : (int, int, int) PState.t = <fun>
\end{bluecode}

\begin{alternate}
v 3 ;;
6- : int * int = (9, 6)  
\end{alternate}

\begin{redcode}
let v = PState.(perform x <-- return 1 ; y <-- return 2 ; z <-- get ; 
put (string_of_int (x+y+z)) ; return z );;
\end{redcode}
\begin{bluecode}
val v : (int, int, string) PState.t = <fun>
\end{bluecode}

\begin{alternate}
# v 3;;
v 3;;
- : int * string = (3, "6")
\end{alternate}

\end{enumerate}


%%% Local Variables: 
%%% mode: latex
%%% TeX-master: "../master"
%%% End: 

\section{bigarray}
\label{sec:bigarray}

This implementation allows efficient sharing of large numerical arrays
between Caml code and C or Fortran numerical libraries. Your are
encouraged to \verb|open Bigarray|. Big arrays support the
ad-hoc polymorphic operations (comparison, hashing,marshall)

Element kinds

The abstract type \verb|type ('a,'b) kind| captures type 'a for
values read or written in the array, while 'b which represents the
actual content of the big array.

Array layouts


%%% Local Variables: 
%%% mode: latex
%%% TeX-master: "../master"
%%% End: 

\section{sexplib}
\label{sec:sexplib}
Basic Usage

\begin{bluetext}
#require "sexplib.top";;
\end{bluetext}
\begin{ocamlcode}
open Sexplib
open Std
type t = A of int list | B with sexp;;
module S = Sexp;;
module C = Conv;;
\end{ocamlcode}

\begin{ocamlcode}
 sub_modules "Sexplib";;
module This_module_name_should_not_be_used :
    module Type :
    module Parser :
    module Lexer :
    module Pre_sexp :
        module Annot :
        module Parse_pos :
        module Annotated :
        module Of_string_conv_exn :
    module Sexp_intf :
        module type S =
            module Parse_pos :
            module Annotated :
            module Of_string_conv_exn :
    module Sexp :
        module Parse_pos :
        module Annotated :
        module Of_string_conv_exn :
    module Path :
    module Conv :
        module Exn_converter :
    module Conv_error :
    module Exn_magic :
    module Std :
        module Hashtbl :
            module type HashedType =
            module type S =
            module Make :
        module Big_int :
        module Nat :
        module Num :
        module Ratio :
        module Lazy :
\end{ocamlcode}

Build
\todo{build with sexplib}

Debug 
\begin{bluetext}
camlp4o -parser Pa_type_conv.cma pa_sexp_conv.cma   sexp.ml -printer o
\end{bluetext}


Modules

\verb|Sexp| Contains all I/O-functions for Sexp, module \verb|Conv|
helper functions converting OCaml-valus of \verb|standard-types| to
Sexp. Moduel \verb|Path| supports sub-expression extraction and
substitution.

Sexp 
\mint{ocaml}/type t = Sexplib.Type.t = Atom of string | List of t list/
Syntax

\verb|with sexp | or \verb|with sexp_of| or \verb|with of_sexp|. 
\verb|signatures| are also  well supported. When packed, you should
use \verb|TYPE_CONV_PATH| to make the location right. Common utilities
are exported by \verb|Std|.

we hope \verb/sexp_of_t |- t_of_sexp / to be an \verb|id| function
\begin{ocamlcode}
let f = exp_of_int |- int_of_exp 
Enum.(let a = range ~until:max_int min_int in 
        fold2 (fun l r a -> a & (l=r)) true a (map f a) );;  
\end{ocamlcode}

%%% Local Variables: 
%%% mode: latex
%%% TeX-master: "../master"
%%% End: 

\section{bin-prot}
\label{sec:bin-prot}

\section{fieldslib}
\label{sec:fieldslib}

\section{variantslib}
\label{sec:variantslib}

\section{delimited continuations}
\label{sec:cont-delim-cont}
Continuatioins
A conditional banch selects a continuation from the two possible
futures; rasing an exception discards. Traditional way to handle
continuations explicitly in a program is to transform a program into
cps style. Continuation captured by call/cc is the {\bf whole} continuation
that includes all the future computation.. In practice, most of the
continuations that we want to manipulate are only a part of
computation. Such continuations are called {\bf delimited continuations} or
{\bf partial continuations}.


\begin{enumerate}
\item cps transform \\
  there are multiple ways to do cps transform, here are two.

  
  \begin{bluetext}
------------------------------------------
   [x] --> x
   [\x. M] --> \k . k (\x . [M])
   [M N] --> \k. [M] (\m . m [N] k)
------------------------------------------


------------------------------------------
   [x] --> \k . k x
   [\x. M] --> \k. k (\x.[M])
   [M N] --> \k. [M] (\m . [N] (\n. m n k))
------------------------------------------


[callcc (\k. body)] = \outk. (\k. [body] outk) (\v localk. outk v)
   
  \end{bluetext}

  
\item experiment

\begin{alternate}
#load "delimcc.cma";;
\end{alternate}
\begin{alternate}
Delimcc.shift;;
- : 'a Delimcc.prompt -> (('b -> 'a) -> 'a) -> 'b = <fun>
\end{alternate}

\begin{bluetext}
reset (fun () -> M ) --> push_prompt p (fun () -> M )
shift (fun k -> M) --> shift p (fun k -> M )
\end{bluetext}
in racket you should have \textit{(require racket/control)}
and then \textit{(reset expr ...+)}
\textit{(shift id expr ...+)}


\begin{ocamlcode}
module D = Delimcc
(** set the prompt *)  
let p = D.new_prompt ()
let (reset,shift),abort  = D.(push_prompt &&& shift &&& abort ) p;;
let foo x = reset (fun () -> shift (fun cont -> if x = 1 then cont 10 else 20 ) + 100 )
\end{ocamlcode}

\begin{alternate}
foo 1 ;;
- : int = 110
foo 2  ;;
- : int = 20
5 * reset (fun () -> shift (fun k -> 2 * 3 ) + 3 * 4 );;
- : int = 30
reset (fun () -> 3 + shift (fun k -> 5 * 2) ) - 1 ;;
- : int = 9
\end{alternate}
\begin{bluetext}
val p : '_a D.prompt = <abstr>
val reset : (unit -> '_a) -> '_a = <fun>
val shift : (('_a -> '_b) -> '_b) -> '_a = <fun>
val abort : '_a -> 'b = <fun> 
\end{bluetext}

\begin{ocamlcode}
let p = D.new_prompt ()
let (reset,shift),abort  = D.(push_prompt &&& shift &&& abort ) p;;
\end{ocamlcode}

\begin{alternate}
reset (fun () -> if (shift (fun k -> k(2 = 3))) then "hello" else "hi ") ^ "world";;
- : string = "hi world"
reset (fun () -> if (shift (fun k -> "laji")) then "hello" else "hi ") ^ "world";;
- : string = "lajiworld"
reset (fun _ -> "hah");;
- : string = "hah"
\end{alternate}


\begin{ocamlcode}
let make_operator () =  
  let p = D.new_prompt () in 
  let (reset,shift),abort = D.(push_prompt &&& shift &&& abort) p in 
  p,reset,shift,abort
\end{ocamlcode}

Delimited continuations seems not able to handle answer type polymorphism.

\begin{bluetext}
exception Str of [`Found of int | `NotFound]  
\end{bluetext}

\begin{ocamlcode}
let times lst  = 
  let rec times_aux lst = match lst with 
    | [] -> 1 
    | 0 :: xs -> shift (fun _ -> 0 )
    | x :: xs -> begin 
      (* printf "entering %d\n" x ; *)
      let v = x * times_aux xs in 
      (* printf "exiting %d\n" x ;  *)
      v
    end in 
  reset (fun () -> times_aux lst )
\end{ocamlcode}

Store the continuation, the type system is not friendly to the
continutations, but fortunately we have \textit{side effects} at hand, we can
store it. (This is pretty hard in Haskell )

\begin{ocamlcode}
let p,reset,shift,abort = make_operator() in 
  let c = ref None in 
  begin 
   reset (fun () -> 3 + shift (fun k -> c:= Some k ;  0) - 1)  ; 
   Option.get (!c) 20 
   end ;;
          Characters 81-139:
     reset (fun () -> 3 + shift (fun k -> c:= Some k ;  0) - 1)  ; 
     ^^^^^^^^^^^^^^^^^^^^^^^^^^^^^^^^^^^^^^^^^^^^^^^^^^^^^^^^^^
     Warning 10: this expression should have type unit.
   \end{ocamlcode}
\begin{ocamlcode}   
- : int = 22
\end{ocamlcode}
\begin{ocamlcode}
let cont = 
  let p,reset,shift,abort = make_operator() in 
  let c = ref None in 
  let rec id lst = match lst with 
    | [] -> shift (fun k -> c:=Some k ; [] )
    |x :: xs -> x :: id xs in 
  let xs = reset (fun () -> id [1;2;3;4]) in 
  xs, Option.get (!c);;
\end{ocamlcode}
\begin{ocamlcode}
val cont : int list * (int list -> int list) = ([], <fun>)
\end{ocamlcode}
\begin{alternate}
# let a,b = cont ;;
val a : int list = []

val b : int list -> int list = <fun>
# b [];;
- : int list = [1; 2; 3; 4]
\end{alternate}



\begin{ocamlcode}
type tree = Empty | Node of  tree * int  * tree 
let walk_tree = 
  let cont = ref None in 
  let p,reset,shift,abort = make_operator() in 
  let yield n = shift (fun k -> cont := Some k; print_int n ) in 
  let rec walk2 tree = match tree with 
    |Empty -> ()
    |Node (l,v,r) -> 
      walk2 l ;
      yield v ; 
      walk2 r in 
  fun tree -> (reset (fun _ -> walk2 tree ), cont);;
\end{ocamlcode}
\begin{ocamlcode}
val walk_tree : tree_t -> unit * ('_a -> unit) option Batteries.ref =
\end{ocamlcode}

\begin{alternate}
# let _, cont = walk_tree tree1 ;;
1val cont : ('_a -> unit) option Batteries.ref = {contents = Some <fun>}
# Option.get !cont ();;
2- : unit = ()
# Option.get !cont ();;
3- : unit = ()
# Option.get !cont ();;
- : unit = ()
# Option.get !cont ();;
- : unit = ()
\end{alternate}

It's quite straightforward to implement yield using delimited
continuation, since each time shifting will escape the control, and you store the continuation, later it can be resumed.


\begin{bluetext}
(** defer the continuation *)  
shift (fun k -> fun () -> k "hello")
\end{bluetext}

By wrapping continuations, we can \textbf{access the information outside} of the enclosing
reset while staying within reset lexically.

suppose this type check

\begin{alternate}
  let f x = reset (fun () -> shift (fun k -> fun () -> k "hello") ^ "world" ) x
  f : unit -> string 
\end{alternate}

\item Answer type modification (serious)
  in the following context,
  \verb|reset (fun () -> [...] ^  "word" )|, the value returned by
  reset appears to be a string. An answer type is a type of the enclosing
  \emph{reset}.

\item reorder delimited continuations \\
  if we apply a continuation at the tail position, the captured computation is simply
  resumed. If we apply a continuation at the non-tail position, we can perform
  additional computation after resumed computation finishes.

  Put differently, we can switch the execution order of the surrounding context.

\begin{ocamlcode}
let p,reset,shift,abort = make_operator () in
    reset (fun () -> 1 + (shift (fun k -> 2 * k 3 )));;
\end{ocamlcode}
\begin{ocamlcode}
- : int = 8    
\end{ocamlcode}

\begin{ocamlcode}
let p,reset,shift,abort = make_operator () in 
   let either a b = shift (fun k -> k a ; k b ) in 
   reset (fun () -> 
   let x = either  0  1 in 
   print_int x ; print_newline ());;
 \end{ocamlcode}
\begin{ocamlcode} 
 0
 1  
\end{ocamlcode}
\item useful links \\
  \href{http://blog.fitzell.ca/2009/01/seaside-partial-continuations.html}{sea
    side} \\
  \href{http://pllab.is.ocha.ac.jp/~asai/cw2011tutorial/}{shift and
    reset tutorial} \\
  \href{http://pllab.is.ocha.ac.jp/~asai/cw2011tutorial/main-e.pdf}{shift
    reset tutorial} \\
  \href{http://docs.racket-lang.org/reference/cont.html#(part._.Classical_.Control_.Operators)}{racket
    control operators} \\
  \href{http://okmij.org/ftp/continuations/caml-shift.pdf}{caml-shift-paper.pdf} \\
  \href{http://okmij.org/ftp/continuations/caml-shift-talk.pdf}{caml-shift-talk} \\

\end{enumerate}

\section{shcaml}
A shell library. (you can refer \verb|Shell| module of shell package)

All modules in the system are submodules of the \verb|Shcaml| module,
except ofr the module \verb|Shtop|




\section{deriving}

Build 


For debuging

\begin{bluetext}
cd `camlp4 -where`
ln -s `ocamlfind query deriving-ocsigen`/pa_deriving.cma 
\end{bluetext}

So you could type \verb|camlp4o -parser pa_deriving.cma test.ml| 


Toplevel
\verb|#require "deriving-ocsigen.syntax";;| 


For building, a typical tags file is as follows.
\begin{bluetext}
true : pkg_deriving-ocsigen
<test.ml> : syntax_camlp4o, pkg_deriving-ocsigen.syntax
\end{bluetext}
\inputminted{ocaml}{library/code/deriving/test.ml}

\section{Modules}

\begin{itemize}
\item BatEnum
  \begin{itemize}
  \item utilities


\begin{ocamlcode}
  range ~until:20 3
  filter, concat, map, filter_map
  (--), (--^) (|>) (@/) (/@)
  No_more_elements (*interface for dev to raise (in Enum.make next)*)
  icons, lcons, cons
\end{ocamlcode}

  \item don't play effects with enum
  \item idea??? how about divide enum to two; one is just for iterator
    the other is for lazy evaluation. (iterator is lazy???)
  \end{itemize}
\item Set (\emph{one comparison, one container})


\begin{ocamlcode}
Set.IntSet
Set.CharSet
Set.RopeSet
Set.NumStringSet
\end{ocamlcode}
for polymorphic set 

\begin{ocamlcode}
split
union
empty
add
\end{ocamlcode}
 why polymorphic set is dangerous? Because in Haskell, \textit{Eq a =>} is implicitly
 you want to make your comparison method is unique, otherwise you
 union two sets, how to make sure they use the same comparison, here
 we use abstraction types, one comparison, one container
 we can not override polymorphic = behavior, polymorphic = is pretty bad practice
 for complex data structure, mostly not you want, so write compare by yourself

As follows, compare is the right semantics.
\begin{alternate}
# Set.IntSet.(compare (of_enum (1--5))  (of_enum (List.enum [5;3;4;2;1])));;
- : int = 0
# Set.IntSet.(of_enum (1--5) = of_enum (List.enum [5;3;4;2;1]));;
- : bool = false
\end{alternate}


\item caveat
  \begin{itemize}
  \item module syntax

 \begin{ocamlcode}
module Enum = struct
  include Enum include Labels include Exceptionless
end
\end{ocamlcode}


    floating nested modules up (Enum.include, etc)
    include Enum, will expose all Enum have to the following context, so Enum.Labels
    is as Labels, so you can now include Labels, but \emph{Labels.v will override Enum.v},
    maybe you want it, and \emph{module Enum still has Enum.Labels.v}, we just duplicated
    the nested module into toplevel
  \end{itemize}
\end{itemize}

%%% Local Variables: 
%%% mode: latex
%%% TeX-master: "master"
%%% End: 


\chapter{Runtime}
\label{sec:runtime}
\begin{enumerate}
\item values \\
  integer-like \textit{int,char,true, false, [], (), and some variants} (batteries dump)
  \textit{pointer} (word-aligned, the bottom 2 bits of every pointer always 00,
  3 bits 000 for 64-bit)

\begin{bluetext}
% 32 bit 
+----------------------------------------+---+---+
| pointer                                | 0 | 0 |
+----------------------------------------+---+---+

+--------------------------------------------+---+
| integer (31 or 63 bits)                    | 1 |
+--------------------------------------------+---+

% why ?
% GC needs this information
% if the algorithm uses arrays of 32/64bit numbers,
% then you can use a Bigarray

+---------------+---------------+---------------+- - - - -
| header        | word[0]       | word[1]       | ....
+---------------+---------------+---------------+- - - - -
                  ^
                  |
              pointer (a value)


+---------------+----------------+------------------+
| header        | 'a' 'b' 'c' 'd' 'e' 'f' '\O' '\1' |
+---------------+----------------+------------------+
                  ^
                  |
              an OCaml string

+---------------+---------------+---------------+- - - - -
| header        | value[0]      | value[1]      | ....
+---------------+---------------+---------------+- - - - -
                  ^
                  |
              an OCaml array

+---------------+---------------+
| header        | arg[0]        |
+---------------+---------------+
                  ^
                  |
              a variant with one arg


+---------------+---------------+----------+--+--+---------------+
| size of the block in words               | col | tag byte      |
+---------------+---------------+----------+--+--+---------------+
 ^                                         <- 2b-><--- 8 bits --->
 |
offset -4 or -8
% 32 platform, it's 22bits long : the reason for the annoying 16MByte limit
% for string
% the tag byte is multipurpose
% in the variant-with-parameter example above, it tells you which
% variant it is. In the string case, it contains a little bit of runtime
% type information. In other cases it can tell the gc that it's a lazy value
% or opaque data that the gc should not scan



+---------------+---------------+---------------+- - - - -
| header        | float[0]                      | ....
+---------------+---------------+---------------+- - - - -
                  ^
                  |
              an OCaml float array

% in the file <byterun/mlvalues.h>
\end{bluetext}

  \begin{tabular}{|p{3cm}|p{12cm}|}
    \hline 
    any int, char & stored directly as a value, shifted left by 1 bit, with LSB=1\\
    \hline 
    (), [], false & stored as OCaml int 0 (native 1) \\
    \hline 
    true & stored as OCaml int 1 \\
    \hline 
    variant type t = Foo | Bar | Baz
    (no parameters)  & stored as OCaml int 0,1,2 \\
    \hline 
    variant type t = Foo | Bar of int & the varient with no parameters are stored
    as OCaml int 0,1,2, etc. counting just the variants that have no parameters.
    The variants with parameters are stored as blocks, counting just the variants with
    parameters. The parameters are stored as words in the block itself. Note there is
    a limit around {\bf 240 variants with parameters that applies to each type},
    but no limit on the number of variants without parameters you can have. {\bf this limit arises because of the size of the tag byte and the fact that some of high numbered tags are reserved} \\
    \hline 
    list [1;2;3] & This is represented as 1::2::3::[] where [] is a value in OCaml int 0,
    and h::t is a block with tag 0 and two parameters. This representation is exactly
    the same as if list was a variant \\
    \hline 
    tuples, struct and array & These are all represented identically, as a simple
    array of values, the tag is 0. The only difference is that an array can be allocated
    with variable size, but structs and tuples always have a fixed size.
    \\
    \hline
    struct or array where every elements is a float & These are treated as a special case.
    The tag has special value \verb|Dyn_array_tag| (254) so that the GC knows how to deal with
    these. {\bf Note this exception does not apply to tuples that contains floats, beware
      anyone who would declare a vector as (1.0,2.0)}. \\
    \hline
    any string & strings are byte arrays in OCaml, but they have quite a clever representation to make it very efficient to get their length, and at the same time make them directly
    compatible with C strings. The tag is \verb|String_tag| (252).
    \\
    \hline 
  \end{tabular}

 here we see the module  Obj
\begin{alternate}
Obj.("gshogh" |> repr |> tag);;
- : int = 252
\end{alternate}

\begin{alternate}
let a = [|1;2;3|] in Obj.(a|>repr|>tag);;
- : int = 0
Obj.(a |> repr |> size);;
- : int = 3
\end{alternate}

string has a clever algorithm
\begin{alternate}
Obj.("ghsoghoshgoshgoshgoshogh"|> repr |> size);;
- : int = 4 (4*8 = 32 )
"ghsoghoshgoshgoshgoshogh" |> String.length;;
24 (padding 8 bits)
\end{alternate}

like all heap blocks, strings contain a header defining
the size of the string in  machine words.

\begin{alternate}
("aaaaaaaaaaaaaaaa"|>String.length);;
- : int = 16
# Obj.("aaaaaaaaaaaaaaaa"|>repr |> size);;
- : int = 3
\end{alternate}
padding will tell you how many words are padded actually

\begin{bluetext}
number_of_words_in_block * sizeof(word) + last_byte_of_block - 1
\end{bluetext}

The null-termination comes handy when passing a string to C, but is
not relied upon to compute the length (in Caml), allowing the string
to contain nulls.


  

\begin{bluetext}
repr : 'a -> t (id)
obj : t -> 'a (id)
magic : 'a -> 'b (id)

is_block : t -> bool = "caml_obj_is_block"
is_int : t -> bool = "%obj_is_int"

tag : t -> int ="caml_obj_tag" % get the tag field 
set_tag : t -> int -> unit = "caml_obj_set_tag"

size : t -> int = "%obj_size" % get the size field 

field : t -> int -> t = "%obj_field" % handle the array part 
set_field : t -> int -> t -> unit = "%obj_set_field"

double_field : t -> int -> float
set_double_field : t -> int -> float -> unit

new_block : int -> int -> t = "caml_obj_block"

dup : t -> t = "caml_obj_dup"

truncate : t -> int -> unit = "caml_obj_truncate"
add_offset : t -> Int32.t -> t = "caml_obj_add_offset"

marshal : t -> string 
\end{bluetext}


\begin{alternate}
Obj.(None |> repr |> is_int);;
- : bool = true
Obj.("ghsogho" |> repr |> is_block);;
- : bool = true
Obj.(let f x = x |> repr |> is_block in (f Bar, f (Baz 3)));;
- : bool * bool = (false, true)
\end{alternate}

\end{enumerate}

%%% Local Variables: 
%%% mode: latex
%%% TeX-master: "../master"
%%% End: 


\chapter{GC}
\label{sec:gc}
\begin{enumerate}
\item heap \\
  Most OCaml blocks are created in the minor(young) heap.
  \begin{enumerate}[(a)]
  \item minor heap ( \textit{32K words for 32 bit, 64K for 64 bit by default})
    in my mac,  i use ``ledit ocaml -init x'' to avoid loading startup
    scripts, then 
\begin{alternate}
Gc.stat ()
\end{alternate}
\begin{ocamlcode}
{Gc.minor_words = 104194.; Gc.promoted_words = 0.; Gc.major_words = 43979.;
 Gc.minor_collections = 0; Gc.major_collections = 0; Gc.heap_words = 126976;
 Gc.heap_chunks = 1; Gc.live_words = 43979; Gc.live_blocks = 8446;
 Gc.free_words = 82997; Gc.free_blocks = 1; Gc.largest_free = 82997;
 Gc.fragments = 0; Gc.compactions = 0; Gc.top_heap_words = 126976;
 Gc.stack_size = 52}  
\end{ocamlcode}

\begin{alternate}
78188 lsr 16 ;;
- : int = 1
\end{alternate}


    

\begin{bluetext}
+---------------------------------------------------------+
| unallocated                  |///allocated part/////////|
+---------------------------------------------------------+
 ^                              ^
 |                              |
caml_young_limit             caml_young_ptr
                      <----- allocation proceeds
                            in this direction
\end{bluetext}

    
    Consider \textit{the array of two elements}, the total size of this object \textit{will be 3 words (header + 2 words)}, so 24 bytes for 64-bit , so the fast path for allocation is
    subtract size from caml\_young\_ptr.
    If caml\_young\_ptr $<$ caml\_young\_limit, then take the slow path through the garbage collector.
    The fast path just \textbf{ five machine instructions and no branches}. But even
    five instructions are costly in inner loops, be careful.
  \item major heap \\
    when the minor heap runs out, it triggers a \textbf{minor collection}. The minor
    collection starts at all the local roots and \textit{oldifies} them, basically copies
    them by reallocating those objects (recursively) \textbf{ to the major heap}. After
    this, any object left in the minor heap \textbf{ are unreachable}, so the minor heap
    can be reused by resetting \textbf{ caml\_young\_ptr }.

\begin{bluetext}
           reachable        reachable
+---------+---------+------+----+----+-----------------+
|         |/////////|      |///-->///|                 |
+---------+---------+------+----+----+-----------------+
           ^                ^
           |                |
        local root       local root
\end{bluetext}

    At runtime the garbage collector \textit{always} knows what is a pointer, and what is an int
    or opaque data (like a string). Pointers get scanned so the GC can find unreachable
    blocks. Ints and opaque data must not be scanned. \textit{This is the reason for having a tag
    bit for integer-like values}, and one of the uses of the tag byte in the header.

\begin{bluetext}
                  "Tag byte space"
+----------+
| 0        | Array, tuple, etc.
+----------+
| 1        |
+----------+
| 2        |
~          ~
|          | Tags in the range 0..245 are used for variants
~          ~
| 245      |
+----------+
| 246      | Lazy (before being forced)
+----------+
| 247      | Closure
+----------+
| 248      | Object                            ^
+----------+                                   |  Block contains
| 249      | Used to implement closures        |  values which the
+----------+                                   |  GC should scan
| 250      | Used to implement lazy values     |
+----------+ <---------------------------  No_scan_tag
| 251      | Abstract data                     |
+----------+                                   |  Block contains
| 252      | String                            |  opaque data
+----------+                                   |  which GC must
| 253      | Double                            V  not scan
+----------+
| 254      | Array of doubles
+----------+
| 255      | Custom block
+----------+

\end{bluetext}

    so, in the normal course of events, a small, long-lived object will start on the
    minor heap and be copied into the major heap. \textbf{ Large objects go straight to
      the major heap}
    But there is another important structure used in the major heap, called the
    \textbf{ page table}. The garbage collector must at all times know which pieces of
    memory belong to the major heap, and which pieces of memory do not, and it uses
    the page table to track this.
    One reason \textbf{ why we always want to know where the major heap lies }
    is so we can avoid
    scanning pointers which point to C structs outside the OCaml heap.
    The GC will not stray beyond its own heap, and treats all pointers outside as
    opaque (it doesn't touch them or follow them).
    In OCaml 3.10 the page table was implemented as a simple bitmap, with 1 bit per page
    of virtual memory (major heap chunks are always page-aligned). This was
    unsustainable for 64 bit address spaces where memory allocations can be very very
    \textbf{ far apart}, so in OCaml 3.11 this was changed to a sparse hash table.
    Because of the page table, C pointers can be stored directly as values, which
    saves time and space. (However, if your C pointer later gets freed, you must NULL
    the value-the reason is that the same memory address might later get malloced
    for the OCaml major heap, thus \textit{suddenly} becoming a \textit{valid} address again.
    THIS usually results in crash ).
    In a functional language \textbf{ which does not allow any mutable references}, there's one
    guarantee you can make which is there could \textbf{ never be a pointer going from the major heap
      to something in the minor heap}, so when an object in an immutable language graduates from the
    minor heap to the major heap, it is fixed forever(until it becomes unreachable), and can not
    point back to the minor heap.
    But ocaml is impure, so if the minor heap collection worked exactly as previous, then the outcome
    wouldn't be good, maybe some object is not pointed at \textbf{ by any local root}, so it would
    be \textit{unreachable} and would \textit{disappear}, leaving a \textbf{ dangling pointer}.
    \textbf{ one solution would be to check the major heap, but that would be massively time-consuming:
      minor-collections are supposed to be very quick }
    What OCaml does instead is to have a separate \textit{refs} list. This contains a list of pointers
    that point \textbf{ from the major heap to the minor heap}. During a minor heap collection, the
    refs list is consulted for additional roots(and after the minor heap collection, the refs list
    can be started anew).

    The refs list however has to be updated, and it gets \textbf{ updated potentially every time we modify a mutable
      field in a struct}. The code calls the c function \textbf{ caml\_modify} which both mutates the struct a
    nd decides whether this is a major$\rightarrow$minor pointer to be
    added to the refs list.

    If you use mutable fields then this is \textbf{ much slower} than a
    simple assignment. However, \textbf{ mutable integers} are ok, and
    don't trigger the extra call. You can also \textbf{ mutate fields}
    yourself, eg. from c functions or using Obj, \textbf{ provied you can
    guarantee that this won't generate a pointer between the major and
    minor heaps}.

    The OCaml gc does not collect the major heap in one go. It spreads
    the work over small \textbf{ slices}, and splices are grouped into
    whole \emph{phases} of work.

    \emph{A slice} is just a defined amount of work.

    The phases are mark and sweep, and some additional sub-passes
    dealing with weak pointers and finalization.

    Finally there is \emph{a compaction phase} which is triggered when
    there is no other work to do and the estimate of free space in the
    heap has reached some threshold. This is tunable. You can schedule
    when to compact the heap -- while waiting for a key-press or
    between frames in a live simulation.

    There is also a penalty for doing a slice of the major heap -- for
    example if the minor heap is exhausted, then some activity in the
    major heap is unavoidable. However if you make the \textbf{ minor heap
      large enough}, you can completely control when GC work is
    done. You can also move \emph{large structures out of the major
      heap entirely}, 
    
    
  \end{enumerate}
\item module Gc

\begin{bluetext}
Gc.compact () ;;
let checkpoint p = Gc.compact () ; prerr_endline ("checkpoint at poisition " ^ p )
\end{bluetext}
The checkpoint function does two things:
\textit{Gc.compact () } does a full major round of garbage
collection and compacts the heap. This is the most aggressive form of
Gc available, and it's highly likely to \textit{segfault} if the heap is corrupted.
\textit{prerr\_endline} prints a message to stderr and crucially
also flushes stderr, so you will see the message printed immediately.

you \textbf{should} grep for \verb|caml_heap_check| in byterun for details

\begin{ocamlcode}

void caml_compact_heap (void)
{
  char *ch, *chend;
                                          Assert (caml_gc_phase == Phase_idle);
  caml_gc_message (0x10, "Compacting heap...\n", 0);

#ifdef DEBUG
  caml_heap_check ();
#endif


#ifdef DEBUG
void caml_heap_check (void)
{
  heap_stats (0);
}
#endif


#ifdef DEBUG
  ++ major_gc_counter;
  caml_heap_check ();
#endif


\end{ocamlcode}


\item tune \\
  problems can arise when you're building up ephemeral
  data structures which are larger than the minor heap.
  The data structure won't stay around overly long, but
  it is a bit too large. Triggering major GC slices more
  often can cause static data to be walked and re-walked
  more often than is necessary.
  \href{http://elehack.net/michael/blog/2010/06/ocaml-memory-tuning}{tuning}  sample

  \begin{ocamlcode}
let _ =
  let gc = Gc.get () in
    gc.Gc.max_overhead <- 1000000;
    gc.Gc.space_overhead <- 500;
    gc.Gc.major_heap_increment <- 10_000_000;
    gc.Gc.minor_heap_size <- 10_000_000;
    Gc.set gc

\end{ocamlcode}

\end{enumerate}

%%% Local Variables: 
%%% mode: latex
%%% TeX-master: "../master"
%%% End: 



\chapter{Object-oriented}


\section{Simple Object Concepts}

\inputminted[fontsize=\scriptsize, fontsize=\scriptsize, lastline=8]{ocaml}{oo/code/simple.ml}


obj\#method, the actual method gets called is determined at runtime.


\inputminted[fontsize=\scriptsize, fontsize=\scriptsize, firstline=8,lastline=15]{ocaml}{oo/code/simple.ml}


.. is a row variable 
\inputminted[fontsize=\scriptsize, fontsize=\scriptsize, firstline=15,lastline=18]{ocaml}{oo/code/simple.ml}


\textbf{ \{<>\} }represents a \textbf{functional update} (only
fields), which produces a new object


Some other examples
\inputminted[fontsize=\scriptsize, fontsize=\scriptsize, firstline=18,lastline=67]{ocaml}{oo/code/simple.ml}


Something to Notice

Field expression \textbf{could not} refer to other fields, nor to
itself, after you get the object you can have initializer. The object
\textit{does not exist} when the field values are be computed. For the
initializer, you can call \verb|self#blabla|
\inputminted[fontsize=\scriptsize, fontsize=\scriptsize, firstline=66,lastline=80]{ocaml}{oo/code/simple.ml}


Private method
\inputminted[fontsize=\scriptsize, fontsize=\scriptsize, firstline=81,lastline=88]{ocaml}{oo/code/simple.ml}


Subtyping


Supports \textit{width and depth subtyping, contravariant and
  covariant} for subtyping of recursive object types, \textit{first
  assume it is right} then prove it using such assumption. Sometimes,
type annotation and coersion both needed, when t2 is recursive or t2
has polymorphic structure.

\begin{ocamlcode}
  e : t1 :> t2
\end{ocamlcode}


Simulate narrowing(downcast)

\inputminted[fontsize=\scriptsize, fontsize=\scriptsize, ]{ocaml}{oo/code/downcast.ml}

It's doable, since \verb|exn| is open and its tag is global, and you
can store the tag information uniformly. But onething to notice is
that you can not write safe code, since exn is extensible, you can not
guarantee that you match is exhuastive.

You can also implement using polymorphic variants, this is essentially
the same thing, since \verb|Polymorphic Variants| is also global and
extensible.


\inputminted[fontsize=\scriptsize, fontsize=\scriptsize, ]{ocaml}{oo/code/downcast2.ml}



\section{Modules vs Objects}

\begin{enumerate}
\item Objects (data entirely hidden)
\item Self recursive type is so natural in objects, isomorphic-like
  equivalence is free in oo.
\item Example

\inputminted[fontsize=\scriptsize, fontsize=\scriptsize, ]{ocaml}{oo/code/obj_module.ml}
\end{enumerate}


\section{More about class}


\todo{Write later}

\chapter{complex language features}
\label{sec:compl-lang-feat}



\href{http://mirror.ocamlcore.org/ocaml-tutorial.org/streams.html}{streams}

\begin{enumerate}
\item stream expression

  \begin{redcode}

let rec walk dir = 
    let items = try 
      Array.map (fun fn -> let path = Filename.concat dir fn in 
             try if Sys.is_directory path then `Dir path else `File path
             with e -> `Error(path,e) ) (Sys.readdir dir)
      with e -> [| `Error (dir,e) |] in 
      Array.fold_right 
        (fun item rest -> match item with 
            |`Dir path -> [< 'item ; walk path; rest >]
            | _ -> [< 'item; rest >]) items [< >];;


(** alternative without syntax extension *)
let rec walk dir =
  let items =
    try
      Array.map
        (fun fn ->
           let path = Filename.concat dir fn
           in
             try if Sys.is_directory path then `Dir path else `File path
             with | e -> `Error (path, e))
        (Sys.readdir dir)
    with | e -> [| `Error (dir, e) |]
  in
    Array.fold_right
      (fun item rest ->
         match item with
         | `Dir path ->
             Stream.icons item (Stream.lapp (fun _ -> walk path) rest)
         | _ -> Stream.icons item rest)
      items Stream.sempty


            
Stream.(walk "/Users/bob" |> take 10 |> iter 
s      ((function `Dir s -> "dir :" ^ s | `File s -> "file: " ^ s | `Error (s,e) -> "error: " ^ s ^ " " ^ Printexc.to_string e) |- print_string |- print_newline) );;
            
   \end{redcode}
   
   \begin{bluecode}
- : string ->
    [> `Dir of string | `Error of string * exn | `File of string ]
    Batteries.Stream.t

error: /Users/bob/.#.log Sys_error("/Users/bob/.#.log: No such file or directory")
file: /Users/bob/.aboutenvfiles
file: /Users/bob/.bash_history
file: /Users/bob/.bashrc
file: /Users/bob/.bashrc~
dir :/Users/bob/.cabal
file: /Users/bob/.cabal/.DS_Store
dir :/Users/bob/.cabal/bin
file: /Users/bob/.cabal/bin/alex
file: /Users/bob/.cabal/bin/bf

    
 \end{bluecode}
\item module Stream


\begin{alternate}
Stream.npeek;;
- : int -> 'a Batteries.Stream.t -> 'a list = <fun>
Stream.next;;
- : 'a Stream.t -> 'a = <fun>
\end{alternate}



\begin{redcode}
let lines_stream_of_channel chan = Stream.from (fun _ -> 
    try Some (input_line chan) with End_of_file -> None );;
\end{redcode}

\begin{bluecode}  
val lines_stream_of_channel : BatIO.input -> string Batteries.Stream.t =
\end{bluecode}


it raises \textit{Stream.Failure} on an empty stream,
i.e. \textit{Stream.next}

\begin{redcode}
let line_stream_of_string string =
  Stream.of_list (Str.(split (regexp "\n") string))
\end{redcode}

\item Constructing streams \\
  \begin{bluetext}
    Stream.from
    Stream.of_list
    Stream.of_string (* char t *)
    Stream.of_channel (* char t *)
  \end{bluetext}

\item Consuming streams \\

\begin{bluetext}
   Stream.peek
   Stream.junk
\end{bluetext}

\begin{bluecode}
let paragraph lines =
  let rec next para_lines i =
    match Stream.peek lines,para_lines with
    | None, [] -> None
    | Some "", [] ->
      Stream.junk lines (* still a white paragraph *)
      next para_lines i
    | Some "", _ | None, _ ->
      Some (String.concat "\n" (List.rev para_lines)) (* a new paragraph*)
    | Some line, _ ->
      Stream.junk lines ;
      next (line :: para_line ) i in
  Stream.from (next [])    
\end{bluecode}

\begin{redcode}
let stream_fold f stream init = 
    let result = ref init in 
    Stream.iter (fun x -> result := f x !result) stre  am; !result;;
  \end{redcode}

\begin{bluecode}  
val stream_fold : ('a -> 'b -> 'b) -> 'a Batteries.Stream.t -> 'b -> 'b =
  <fun>
\end{bluecode}

\begin{redcode}
let stream_concat streams = 
  let current_stream = ref None in 
  let rec next i = 
    try 
      let stream = match !current_stream with 
        | Some stream -> stream 
        | None -> 
          let stream = Stream.next streams in 
          current_stream := Some stream ; 
          stream in 
      try Some (Stream.next stream)
      with Stream.Failure -> (current_stream := None ; next i)
    with Stream.Failure -> None in 
  Stream.from next
\end{redcode}

\item \textit{copying or sharing} streams \\
  this was called \textit{dup} in Enum
  \begin{bluecode}
(** create 2 buffers to store some pre-fetched value *)
let stream_tee stream = 
  let next self other i = 
    try 
      if Queue.is_empty self 
      then 
        let value = Stream.next stream in 
        Queue.add value other ;
        Some value
      else 
        Some (Queue.take self)
    with Stream.Failure -> None in 
  let q1,q2 = Queue.create (), Queue.create () in 
  (Stream.from (next q1 q2), Stream.from (next q2 q1))
\end{bluecode}

\item convert arbitray data types to streams \\
  if the datat type defines an \textit{iter} function, and you don't
  mind using threads, you can use a \textit{producer-consumer}
  arrangement to invert control.

\begin{redcode}
let elements iter coll = 
  let channel = Event.new_channel () in 
  let producer () = 
    let _ = iter (fun x -> Event.(sync (send channel (Some x )))) coll in 
    Event.(sync (send channel None)) in 
  let consumer i = 
    Event.(sync (receive channel)) in 
  ignore (Thread.create producer ()) ; 
  Stream.from consumer    
\end{redcode}

  \begin{bluecode}
val elements : (('a -> unit) -> 'b -> 'c) -> 'b -> 'a Batteries.Stream.t =    
\end{bluecode}

Keep in mind that these techniques spawn producer threads which carry
a few risks: they only terminate when they have finished iterating,
and any change to the original data structure while iterating may
produce unexpected results.


\end{enumerate}
%%% Local Variables: 
%%% mode: latex
%%% TeX-master: "../master"
%%% End: 

\section{GADT}

\inputminted[fontsize=\scriptsize, fontsize=\scriptsize, ]{ocaml}{lang/code/gadt.ml}

\todo{read ml 2011 workshop paper}

%%% Local Variables: 
%%% mode: latex
%%% TeX-master: "../master"
%%% End: 

\section{First Class Module}

First class mdoule  means module could be passed as a parameter.

Here is a simple example

\inputminted[fontsize=\scriptsize, ]{ocaml}{lang/code/module/intro.ml}

 
Here the argument \verb|m| is a module. This is already possible with
objects and records, but now modules are also allowed.  We introduce
three syntaxes

\begin{ocamlcode}
  (module module_expr: package_type)
  (*packing:  mdoule -> expression *)
  (val expr: package_type)
  (*unpacking: expression -> module  *)
  (module package_type)
  (*a new syntax to type-expr*)
\end{ocamlcode}


Parametric algorithms
\inputminted[fontsize=\scriptsize, fontsize=\scriptsize, ]{ocaml}{code/lang/param.ml}

Notice \verb|with type t = int| is necessary here.

The next is a fancy example to illustrate lebiniz equivalence, readers
should try to digest it. Something to reminder, now the simple type
may be a very complex module type.
\begin{ocamlcode}
type ('a, 'b) eq = (module EqTC with type a = 'a and type b = 'b)))
\end{ocamlcode}


\inputminted[fontsize=\scriptsize, fontsize=\scriptsize, ]{ocaml}{lang/code/module/leibniz.ml}


%%% Local Variables: 
%%% mode: latex
%%% TeX-master: "../master"
%%% End: 

\section{Pahantom Types}

A simple example

\inputminted[fontsize=\scriptsize]{ocaml}{lang/code/phantom_lambda.ml}
\inputminted[fontsize=\scriptsize]{ocaml}{lang/code/phantom_si.ml}
\inputminted[fontsize=\scriptsize]{ocaml}{lang/code/phantom_bat.ml}
\inputminted[fontsize=\scriptsize]{ocaml}{lang/code/phantom.ml}

A fancy example. 
\inputminted[fontsize=\scriptsize]{ocaml}{lang/code/phantom_dim_array.ml}



\subsection{Useful links}
\href{http://camltastic.blogspot.com/2008/05/phantom-types.html}{jones}

\href{http://www.quora.com/What-are-good-applications-of-phantom-types}{jambo}

\href{http://caml.inria.fr/pub/ml-archives/caml-list/2001/09/081c77179ee2a3787233902a51633122.en.html}{caml}

\href{https://ocaml.janestreet.com/?q=node/11}{jane}



%%% Local Variables: 
%%% mode: latex
%%% TeX-master: "../master"
%%% End: 


\section{posit}
\href{https://ocaml.janestreet.com/?q=node/99}{jane}


%%% Local Variables: 
%%% mode: latex
%%% TeX-master: "../master"
%%% End: 

\section{private types}
\label{sec:fancy-types-1}


Private types 

Private type stand between abstract type and concrete types. You can
coerce your private type back to the concrete type (zero-performance),
but backward is \textbf{not allowed}.

For ordinary private type, you can still do pattern match, print the
result in toplevel, and debugger. A big advantage for private type
abbreviation is that for parameterized type(like container) coercion,
you can still do the coercion pretty fast(optimization), and some
parameterized types(not containers) can still do such coercions while
abstract types can not do. Since ocaml does not provide ad-hoc
polymorphism, or type functions like Haskell, this is pretty
straight-forward.

\inputminted[fontsize=\scriptsize, ]{ocaml}{./lang/code/priv.ml}

%%% Local Variables: 
%%% mode: latex
%%% TeX-master: "../master.tex"
%%% End: 

\section{Subtyping}
%%% Local Variables: 
%%% mode: latex
%%% TeX-master: "../master"
%%% End: 

\section{Explicit nameing of type variables}
\label{sec:expl-name-type}

The type constructor it introduces can be used in places where a type
variable is \verb|not allowed|.

\begin{redcode}
let f (type t) () = 
    let module M = struct exception E of t end in
    (fun x -> M.E x ), (function M.E x -> Some x | _ -> None);;
val f : unit -> ('a -> exn) * (exn -> 'a option) = <fun>  
\end{redcode}

The exception defined in local module can not be captured by other
exception handler except wild catch.

Another example:
\begin{redcode}
let sort_uniq (type s) (cmp : s -> s -> int) = 
    let module S = Set.Make(struct type t = s let compare = cmp end) in 
    fun l -> S.elements (List.fold_right S.add l S.empty);;
val sort_uniq : ('a -> 'a -> int) -> 'a list -> 'a list = <fun>  
\end{redcode}
The functor needs a type constructor(type variable is not allowed)


\section{The module Language}
\label{sec:module-language}


\chapter{subtle bugs}
\label{sec:subtle-bugs}

\section{Reload duplicate modules }

 this is fragile when you load some modules like syntax extension, or toploop modules. use \textit{ocamlobjinfo} to
  see which modules are loaded exactly


Polymorphic comparisons

\todo{polymorphic comparison}
\href{https://ocaml.janestreet.com/?q=node/33}{jane}


\chapter{interoperating with C}
\label{sec:inter-with-c}
\todo{Write later}
\input{./c/M4.tex}


\chapter{Book}
\subsection{Developing Applications with Objective Caml}

\begin{enumerate}
  \item caveat
    \begin{enumerate}
    \item + (modulo the boundary, \emph{will not be checked})
    \item $1.0/0.0 \rightarrow \infty $
    \item $+. -. *. /. **$  mod ceil floor sqrt exp log log10 cos sin tan acos asin atan  
    \item $asin 3.14  \rightarrow nan    $
    \item \verb|char_of_int 255| $\rightarrow$ \verb|'\255'| (can not display)
    \item \verb|char_of_int int_of_char string_of_int int_of_string|
      \verb|string_of_int 2551 -> ``2551''|
    \item string (length $\le 2^{24} - 6$ )
    \item == (\textit{physical equal}) (=, != <>)


\begin{alternate}
true == true;;
- : bool = true
# 3 == 3;;
- : bool = true
# 1. == 1.;;
- : bool = false
\end{alternate}

    \item int * int * int is different from (int * int ) * int
    \item unreasonable parametric equality \verb|(=) : 'a -> 'a -> bool|
    \item recursive declaration

\begin{redcode}
let rec ones = 1 :: ones;;
\end{redcode}

\begin{bluecode}
val ones : int list =
  [1; 1; 1; 1; 1; 1; 1; 1; 1; 1; 1; 1; 1; 1; 1; 1; 1; 1; 1; 1; 1; 1; 1; 1; 1;
   1; 1; 1; 1; 1; 1; 1; 1; 1; 1; 1; 1; 1; 1; 1; 1; 1; 1; 1; 1; 1; 1; 1; 1; 1;
   1; 1; 1; 1; 1; 1; 1; 1; 1; 1; 1; 1; 1; 1; 1; 1; 1; 1; 1; 1; 1; 1; 1; 1; 1;
   1; 1; 1; 1; 1; 1; 1; 1; 1; 1; 1; 1; 1; 1; 1; 1; 1; 1; 1; 1; 1; 1; 1; 1; 1;
   1; 1; 1; 1; 1; 1; 1; 1; 1; 1; 1; 1; 1; 1; 1; 1; 1; 1; 1; 1; 1; 1; 1; 1; 1;
   1; 1; 1; 1; 1; 1; 1; 1; 1; 1; 1; 1; 1; 1; 1; 1; 1; 1; 1; 1; 1; 1; 1; 1; 1;
   1; 1; 1; 1; 1; 1; 1; 1; 1; 1; 1; 1; 1; 1; 1; 1; 1; 1; 1; 1; 1; 1; 1; 1; 1;
   1; 1; 1; 1; 1; 1; 1; 1; 1; 1; 1; 1; 1; 1; 1; 1; 1; 1; 1; 1; 1; 1; 1; 1; 1;
   1; 1; 1; 1; 1; 1; 1; 1; 1; 1; 1; 1; 1; 1; 1; 1; 1; 1; 1; 1; 1; 1; 1; 1; 1;
   1; 1; 1; 1; 1; 1; 1; 1; 1; 1; 1; 1; 1; 1; 1; 1; 1; 1; 1; 1; 1; 1; 1; 1; 1;
   1; 1; 1; 1; 1; 1; 1; 1; 1; 1; 1; 1; 1; 1; 1; 1; 1; 1; 1; 1; 1; 1; 1; 1; 1;
   1; 1; 1; 1; 1; 1; 1; 1; 1; 1; 1; 1; 1; 1; 1; 1; 1; 1; 1; 1; 1; 1; 1; 1;
   ...]
\end{bluecode}


\begin{redcode}
 let special_size l = 
    let rec size_aux prev = function 
      |[] -> 0 
      |_ :: l1  -> if List.memq l1 prev then 1 else 1 + size_aux (l1::prev) l1 in size_aux [l]  l;;
    \end{redcode}
\begin{bluecode}    
  val special_size : 'a list -> int = <fun>
\end{bluecode}

\begin{alternate}
# special_size ones;;
- : int = 1
# let rec twos = 1 :: 2 :: twos in special_size twos;;
- : int = 2
# special_size [];;
- : int = 0
\end{alternate}  
\item combine patterns \\
  p1 | .. |  pn (all name is forbidden within these patterns) 
 'a' .. 'e' 

 \begin{alternate}
let test 'a' .. 'e' = true;;
^^^^^^^^^^^^^^^^^
 \end{alternate}

\begin{bluecode}
Warning 8: this pattern-matching is not exhaustive.
Here is an example of a value that is not matched:
'f'
val test : char -> bool = <fun>
\end{bluecode}

    \item records

\begin{alternate}
type complex = {re:float;img:float};;
type complex = { re : float; img : float; }
# let add {re; img} {re; img} = 3;;
val add : complex -> complex -> int = <fun>
# let add {re; img} {re; img} = {re = re +. re; img = img +. img};;
val add : complex -> complex -> complex = <fun>
 \end{alternate}

    \item \emph{redefinition marsks the previous one, while values of the masked types
        still exist, but it now turns to be an abstract type }
    \item exception
      \begin{enumerate}
      \item \verb|Match_failure Division_by_zero Failure|
      \item exception Name of t -- monomorphic , extensible sum Type \\
        when pattern match your exception, its type should be fixed 
      \item control flow
      \end{enumerate}
    \item {\bf disagree over interface} \\
      when toplevel loads the same module (only the name is the same),
      it will check the interface is equal, this sucks since ocaml has
      flat namespace for module 
    \end{enumerate}
  \item sharing \\
    for structured values, it will be sharing , however,
    \emph{vectors of floats don't share}

\begin{alternate}
let a = Array.create 3 0.;;
val a : float array = [|0.; 0.; 0.|]
# a.(0)==a.(1);;
- : bool = false
\end{alternate}


  \item weak type variables \\

\begin{alternate}
  let b = ref []
  (* b should '_a list ref, since b is not pure, cannot be shared *)
  let a = []
  (* a : 'a list *) 
  let a = None
  (* a : 'a option *)n
  let a = Array.create 3 None
  (* '_a option array *)
# type ('a,'b) t ={ch1 : 'a list; mutable ch2 : 'b list};;
type ('a, 'b) t = { ch1 : 'a list; mutable ch2 : 'b list; }
# let v = {ch1=[];ch2=[]};;
val v : ('a, '_b) t = {ch1 = []; ch2 = []}     
\end{alternate}
 \textit{mutable sharing conflicts with polymorphism}

  \item library
    \begin{enumerate}
    \item List \\

\begin{bluecode}
      @ length hd tl nth rev append rev_append concat flatten
      iter map rev_map left_fold fold_right iter2 map2 rev_map2
      fold_left2 fold_right2 for_all exists for_all2 exists2 
      mem memq find filter partition assoc assq remove_assoc remove_assq
      split combine sort statble_sort fast_sort merge
    \end{bluecode}

\begin{alternate}    
# List.assq 3 [3,4;1,2];;
- : int = 4
# List.assq 3. [3.,4;1.,2];;
Exception: Not_found.
\end{alternate}

    \item Array \\
      \verb|Array.create_matrix creates Non-Rectangular matrices|

\begin{bluecode}
length get set make create init -- when you don't want to initialize
make_matrix (int->int->'a -> 'a array array) create_matrix;
append concat sub copy fill ('a array -> int -> int -> 'a -> int)
blit (Array.Labels.blit), to_list, of_list map iteri mapi fold_left
fold_right sort stable_sort fast_sort unsafe_get unsafe_set copy
\end{bluecode}

    \item IO \\

\begin{bluecode}
open_in open_out close_in close_out input_line
input : Batteries.Legacy.in_channel -> string -> int -> int -> int = <fun> 
output: Batteries.Legacy.out_channel -> string -> int -> int -> unit =<fun> 
read_line print_string print_newline print_endline
\end{bluecode}

    \item stack (imperative data structure actually)

\begin{bluecode}
exceptin Empty
create
type 'a t = { mutable c : 'a list }
(* mutable to delay initialization *)
push pop top clear copy is_empty length iter enum copy
of_enum print
module Exceptionless
  top : 'a t -> 'a option, pop
\end{bluecode}

    \item stream \textbf{imperative}

\begin{bluecode}
'a t
exception Failure
exception Error of string
from
of_list of_string of_channel iter empty peek junk count npeek
iapp icons ising lapp lcons lsing
sempty slazy dump npeek
\end{bluecode}

      syntax extension (for my experience, use it in shell, but not in
      tuareg toplevel)
\begin{redcode}
  let concat_stream a b = [<a;b>]
\end{redcode}
\begin{bluecode}
val concat_stream :
  'a Batteries.Stream.t -> 'a Batteries.Stream.t -> 'a Batteries.Stream.t =
\end{bluecode}
   expression not preceded by an \` considered to be sub-stream
   destructive pattern matching (camlp5 or extended parser can merge)
   consumed (error), failure
    \item Array List String Hashtbl Buffer Queue
    \item Sort

\begin{redcode}
module X = Sort ;;
\end{redcode}

\begin{bluecode}
module X :
  sig
    val list : ('a -> 'a -> bool) -> 'a list -> 'a list
    val array : ('a -> 'a -> bool) -> 'a array -> unit
    val merge : ('a -> 'a -> bool) -> 'a list -> 'a list -> 'a list
  end
\end{bluecode}

    \item Weak (vector of weak pointers) abstract type

\begin{bluecode}
sig
  type 'a t = 'a Weak.t
end 
\end{bluecode}


    \item Printf

\begin{bluecode}
%t -> (output->unit)
%t%s -> (output->unit)->string->unit
\end{bluecode}
they all should be processed at \textbf{compile time}


    \item Digest \\
      hash functions return a fingerprint of their entry (reversible) 

\begin{bluecode}
   val string : string -> t -- fingerprint of a string
   val file : string -> t -- fingerprint of a file 
\end{bluecode}

    \item Marshal estimate data size

\begin{alternate}
type external_flag = No_sharing | Closures

let size x = x |> flip Marshal.to_string [] |> flip Marshal.data_size 0;;           ;;
val size : 'a -> int = <fun>
# size 3;;
- : int = 1
# size 3.;;
- : int = 9
# size "ghsogho";;
- : int = 8
# size "ghsogho1";;
- : int = 9
# size "ghsogho1ah";;
- : int = 11
# size 111;;
- : int = 2
\end{alternate}


    \item Sys

\begin{bluecode}
os_type interactive word_size max_string_length
max_array_length time argv getenv command file_exists
remove rename chdir getcwd 
\end{bluecode}

\begin{alternate}
# float (Sys.max_string_length ) /. (2. ** 57.);;
- : float = 0.999999999999999889
\end{alternate}


    \item Arg Filename Printexc
    \item Printexc

\begin{redcode}
# module P = Printexc;;
\end{redcode}

\begin{bluecode}
module P :
  sig
    val to_string : exn -> string
    val catch : ('a -> 'b) -> 'a -> 'b
    val get_backtrace : unit -> string
    val record_backtrace : bool -> unit
    val backtrace_status : unit -> bool
    val register_printer : (exn -> string option) -> unit
    val pass : ('a -> 'b) -> 'a -> 'b
    val print : 'a BatInnerIO.output -> exn -> unit
    val print_backtrace : 'a BatInnerIO.output -> unit
  end
\end{bluecode}


    \item Num
    \item \verb|Arith_status|

\begin{redcode}
# module X = Arith_status;;
\end{redcode}
\begin{bluecode}
module X :
  sig
    val arith_status : unit -> unit
    val get_error_when_null_denominator : unit -> bool
    val set_error_when_null_denominator : bool -> unit
    val get_normalize_ratio : unit -> bool
    val set_normalize_ratio : bool -> unit
    val get_normalize_ratio_when_printing : unit -> bool
    val set_normalize_ratio_when_printing : bool -> unit
    val get_approx_printing : unit -> bool
    val set_approx_printing : bool -> unit
    val get_floating_precision : unit -> int
    val set_floating_precision : int -> unit
  end
\end{bluecode}


    \item Dynlink \\
      choice at execution time, load a new module and hide the
      code code (hot-patch)
      actually (\#load is kinda hot-patch), however to write it in programs
      \emph{more flexible} than \#load , load requires its name are fixed, and
      load will check .mli file, Dynlink \textbf{does not} do this check, while when you
      want to do X.blabla, it still checks, so still don't work, only side
      effects will work.

\begin{redcode}
#direcotry "+dynlink";;
#load "dynlink.cma";;
Dynlink.loadfile "test.cmo";;
\end{redcode}

    \end{enumerate}

  \item syntaxes
  \item expr

\begin{bluecode}
exp	::=value-path  -- value-name or module-path.value-name
 	| constant  
 	| ( expr )  
 	| begin expr end  
 	| ( expr :  typexpr )  
 	| expr ,  expr  { , expr } -- tuple
 	| constr  expr  -- constructor
 	| `tag-name  expr  -- polymorphic variant
 	| expr ::  expr  -- list 
 	| [ expr  { ; expr } ]  
 	| [| expr  { ; expr } |]  
 	| { field =  expr  { ; field =  expr } }  
 	| { expr with  field =  expr  { ; field =  expr } }  
 	| expr  { argument }+ -- application  
 	| prefix-symbol  expr  -- prefix operator
 	| expr  infix-op  expr  
 	| expr .  field  
 	| expr .  field <-  expr  -- still an expression
 	| expr .(  expr )  
 	| expr .(  expr ) <-  expr  
 	| expr .[  expr ]  
 	| expr .[  expr ] <-  expr  
 	| if expr then  expr  [ else expr ]  
 	| while expr do  expr done  
 	| for ident =  expr  ( to |  downto ) expr do  expr done  
 	| expr ;  expr  
 	| match expr with  pattern-matching  
 	| function pattern-matching  
 	| fun multiple-matching  -- multiple parameters matching
 	| try expr with  pattern-matching  
 	| let [rec] let-binding   { and let-binding } in  expr  
 	| new class-path  
 	| object class-body end  
 	| expr #  method-name  
 	| inst-var-name  
 	| inst-var-name <-  expr  
 	| ( expr :>  typexpr )  
 	| ( expr :  typexpr :>  typexpr )  
 	| {< inst-var-name =  expr  { ; inst-var-name =  expr } >}  
 	| assert expr  
 	| lazy expr  
 
argument::=expr  
 	| ~ label-name  
 	| ~ label-name :  expr  
 	| ? label-name  
 	| ? label-name :  expr  
 
pattern-matching::=
 [|] pattern [when expr]-> expr { |pattern  [when expr] ->  expr }  
 
multiple-matching::= { parameter }+  [when expr]-> expr  
 
let-binding::=pattern =  expr  
 	| value-name  { parameter }  [: typexpr] =  expr  
 
parameter::=pattern  
 	| ~ label-name  
 	| ~ ( label-name  [: typexpr] )  
 	| ~ label-name :  pattern  
 	| ? label-name  
 	| ? ( label-name  [: typexpr]  [= expr] )  
 	| ? label-name :  pattern  
 	| ? label-name : (  pattern  [: typexpr]  [= expr] )
\end{bluecode}        
      
\begin{alternate}
  let f ?test:(Some x ) y = x + y;;
  ^^^^^^^^^^^^^^^^^^^^^^^^^
\end{alternate}

\begin{bluecode}
Warning 8: this pattern-matching is not exhaustive.
Here is an example of a value that is not matched:
None
val f : ?test:int -> int -> int = <fun>
\end{bluecode}

  \item pattern

\begin{bluecode}
pattern	::=	value-name  
 	| _  
 	| constant  
 	| pattern as  value-name  
 	| ( pattern )  
 	| ( pattern :  typexpr )  
 	| pattern |  pattern  
 	| constr  pattern  
 	| `tag-name  pattern  
 	| #typeconstr-name  -- object ?
 	| pattern  { , pattern }  
 	| { field =  pattern  { ; field =  pattern } }  
 	| [ pattern  { ; pattern } ]  
 	| pattern ::  pattern  
 	| [| pattern  { ; pattern } |]  
 	| lazy pattern
\end{bluecode}

  \item toplevel-phrase

\begin{bluecode}
toplevel-input::= { toplevel-phrase } ;;  
 
toplevel-phrase::=definition  
 	| expr  
 	| #ident  directive-argument  
 
directive-argument::=epsilon
 	| string-literal  
 	| integer-literal  
 	| value-path
defition ::= let [rec] let-binding {and let-binding}
        | external value-name : typexpr = external-declartion
        | type-definition
        | exception-defition
        | class-definition
        | classtype-definition
        | module module-name {(module-name : module-type)} [:module-type] = module-expr
        | module type module-name = module-type
        | open module-path
        | include module-expr 
\end{bluecode}

  \item type-definition

\begin{bluecode}

type-definition	::= type typedef  { and typedef }  
 
typedef	::= [type-params]  typeconstr-name  [type-information]  
 
type-information::=
  [type-equation] [type-representation]{ type-constraint }  
type-equation::= = typexpr  
 
type-representation::=
          = constr-decl  { | constr-decl }  
 	| = { field-decl  { ; field-decl } }  

type-params::=	type-param  
 	| ( type-param  { , type-param } )  
 
type-param::=	' ident  
 	| + ' ident  
 	| - ' ident  
 
constr-decl::=	constr-name  
 	| constr-name of  typexpr  { * typexpr }  
 
field-decl::=	field-name :  poly-typexpr  
 	| mutable field-name :  poly-typexpr  
type-constraint	::=constraint ' ident =  typexpr
\end{bluecode}

\begin{alternate}
# type t;;
type t
\end{alternate}

\item  interoperating with C

  Difficutilies 
  \begin{enumerate}
  \item Machine reperesentation of data
  \item GC \\
    calling a c function from ocaml must not modify the memory in ways incompatible
    with ocaml gc.
  \item Exceptions \\
    C does not support exceptions, different mechanisms for aborting computations,
    this complicates ocaml's exception handling
  \item sharing common resources \\
    input-output. each language maintains its own input-output buffers.
  \end{enumerate}

  Communications
  \begin{enumerate}
  \item external declarations \\
    it associates a c function definition with an ocaml name, while giving the
    type of the latter.

    \begin{bluetext}
      external caml_name : type = "C_name"
      val caml_name : type
    \end{bluetext}
    both workds, but in the latter case, calls to the c function \textit{first go}
    through the general function application mechanism of ocaml. This is slightly
    less efficient, but hides the implementation of the function as a c function.
  \item external functions with more than five arguments
    \begin{bluetext}
      external caml_name : type = "C_name_bytecode" "C_name_native"
    \end{bluetext}
  \end{enumerate}

  
\end{enumerate}

\subsubsection{chap7 Development Tools}
\label{sec:chap7-devel-tools}
\begin{enumerate}

\item Command names  \\
  
  \begin{tabular}{|c|c|}
    \hline
    ocaml & toplevel top \\
    \hline
    ocamlrun & bytecode interpreter \\
    \hline
    ocamlc & bytecode batch compiler \\
    \hline
    ocamlopt & native code batch compiler \\
    \hline
    ocamlc.opt & \textit{optimized} bytecode batch compiler \\
    \hline
    ocamlopt.opt & \textit{optimized} native code batch compiler \\
    \hline
    ocamlmktop & new \textit{toplevel} constructor \\
    \hline
  \end{tabular}

  The optimized compilers are themselves compiled with the Objective Caml native compiler.
  They compile \textit{faster} but are otherwise \textit{identical} to their unoptimized counterparts.
\item compilation unit \\
  For the interactive system, the unit of compilation corresponds to a phrase of the language. For the batch compiler, the unit of compilation is two files: the source file, and the interface file
  
  \begin{tabular}{|c|c|}
    \hline
    extension & meaning \\
    .ml & source \\
    .mli & interface \\
    .cmi & compiled interface \\
    .cmo & object file (byte) \\
    .cma & library object file(bytecode) \\
    .cmx & object file (native) \\
    .cmxa & library object file(native) \\
    \hline
    .c & c source \\
    .o & c object file (native) \\
    .a & c library object file (native) \\
    \hline
  \end{tabular}

  
  The \textit{compiled interface} is used for both the bytecode and native code compiler.

\item ocamlc \\
  
  \begin{tabular}{|c|c|}
    \hline
    -a & construct a runtime library \\
    -c & compile \textit{without} linking \\
    -o name\_of\_executable & specify the name of the executable \\
    -linkall & link with \textit{all} libraries used \\
    -i & \textit{display all } compiled global declarations \\
    -pp command & preprocessor \\
    -unsafe & turn off index checking \\
    -v & display version \\
    -w list & choose among the list the level of warning message \\
    -impl file & indicate that \textit{file} is a caml source(.ml) \\
    -intf file & as a caml interface(.mli) \\
    -I dir & add directory in the list of directories \\
    \hline
    -thread & light process \\
    -g, -noassert & linking \\
    -custom, -cclib, -ccopt, -cc & standalone executable \\
    -make-runtime, -use-runtime & runtime \\
    -output-obj & c interface \\
  \end{tabular}

  warning messages.

  \begin{tabular}{|c|c|}
    A/a & enable/disable all messages \\
    F/f & partial application in a sequence \\
    P/p & incomplete pattern matching \\
    U/u & missing cases in pattern matching \\
    X/x & enable/disable all other messages \\
    M/m and V/v & for hidden object \\
  \end{tabular}
  the compiler chooses the (A) by default.
  turn off some warnings sometimes is helpful, for example
  \begin{bluetext}
	ocamlbuild -cflags -w,aPF top_level.cma    
  \end{bluetext}

\item ocamlopt   \\
  \begin{tabular}{|c|c|}
    -compact & optimize the produced code for space \\
    -S & keeps the assembly code in a file \\
    -inline level & set the aggressiveness of inlining \\
  \end{tabular}

\item Toplevel
  \begin{tabular}{|c|c|}
    -I dir & adds the directory \\
    -unsafe & no bounds checking \\
  \end{tabular}
\item ocamlmktop \\
  it's ofen used for pulling native object code libraries(typically written in C) into
  a new toplevel.
  \textit{
    -cclib libname, -ccopt optioin, -custom, -I dir -o exectuable
  }

  \begin{bluetext}
    ocamlmktop -custom -o mytoplevel graphics.cma \
    -cclib -I/usr/X11/lib -cclib -lX11
  \end{bluetext}
  
  This \textit{standalone} exe(-custom) wil be \textit{linked} to the library X11(libX11.a) which in turn will be looked up in the path \textit{/usr/X11/lib}

  A standalone exe is a program that \textit{does not } depend on OCaml installation to run.
  The OCaml native compiler produces standalone executables by default. But without \textit{-custom} option, the bytecode compiler produces an executable which requires the \textit{bytecode interpreter ocamlrun}

  \begin{redcode}
ocamlc test.ml -o a
ocamlc -custom test.ml -o b
\end{redcode}

\begin{bluecode}
-rwxr-xr-x   1 bob  staff    12225 Dec 23 16:31 a
-rwxr-xr-x   1 bob  staff   198804 Dec 23 16:31 b
\end{bluecode}

\begin{alternate}
bash-3.2$ cat a | head -n 1
#!/Users/bob/SourceCode/ML/godi/bin/ocamlrun
\end{alternate}
% $

without \textit{-custom}, it depends on \textit{ocamlrun}. With \textit{-custom}, it contains the \textit{Zinc} interpreter as well as the program bytecode, this file can be executed directly or copied to another machien(using the same CPU/Operating System).

Still, the inclusion of machine code means that stand-alone executalbes are not protable
to other systems or other architectures.

\item optimization \\
  It is necessary to not create \textit{intermediate closures} in the case of application on several arguments. For example, when the function \textit{add} is applied with two integers, it is not useful to create the first closure corresponding to the function of applying add to the first argument. It is necessary to note that the creation of a closure would \textit{allocate} certain memory space for the environment and would require the recovery of that memory space in the future. \textit{Automatic memory recovery} is the second major performance concern, along with environment.


\item chap10 Program Analysis Tool \\
  \begin{enumerate}
  \item ocamldep \\

    
    \begin{tabular}{|c|c|}
      -I & add dir \\
      -impl,-intf & \\
      -ml(i)-synonym <e> & cosider <e> as a synonym of .ml(i) extension \\
      -modules & Print module dependencies in raw form(not suitable for make) \\
      -native & generate dependencies for a pure native-code project \\
      -slash & for windows \& unix \\
    \end{tabular}

    
\begin{redcode}
ocamldep -modules *.ml      
\end{redcode}

\begin{bluecode}
ta.ml: Array Printf
tb.ml: Array Ta
\end{bluecode}

  \begin{redcode}
ocamldep *.ml    
\end{redcode}


\begin{bluecode}
ta.cmo:
ta.cmx:
tb.cmo: ta.cmo
tb.cmx: ta.cmx
\end{bluecode}

other examples

\begin{bluetext}
ocamlfind ocamldep -modules dir_top_level_util.ml > dir_top_level_util.ml.depends
ocamlfind ocamldep -pp 'camlp4of -parser pa_mikmatch_pcre.cma' -modules dir_top_level.ml > dir_top_level.ml.depends
\end{bluetext}

\item debug

  \#(un)trace command ,\#untrace\_all.
  \begin{alternate}
let verify_div a b q r = a = b * q + r ;;
val verify_div : int -> int -> int -> int -> bool = <fun>
# #trace verif_div ;;
Unbound value verif_div.
# #trace verify_div ;;
verify_div is now traced.
\end{alternate}

\begin{redcode}
verify_div 11 5 2 1 ;;  
\end{redcode}


\begin{bluecode}
verify_div <-- 11
verify_div --> <fun>
verify_div* <-- 5
verify_div* --> <fun>
verify_div** <-- 2
verify_div** --> <fun>
verify_div*** <-- 1
verify_div*** --> true
- : bool = true  
\end{bluecode}

\begin{bluetext}
let rec belongs_to (e:int) = function 
    | [] -> false 
    | t :: q -> (e=t) || belongs_to e q;;
    val belongs_to : int -> int list -> bool = <fun>
# #trace belongs_to;;
belongs_to is now traced.
# belongs_to 4 [3;5;7;4];;
belongs_to <-- 4
belongs_to --> <fun>
belongs_to* <-- [3; 5; 7; 4]
belongs_to <-- 4
belongs_to --> <fun>
belongs_to* <-- [5; 7; 4]
belongs_to <-- 4
belongs_to --> <fun>
belongs_to* <-- [7; 4]
belongs_to <-- 4
belongs_to --> <fun>
belongs_to* <-- [4]
belongs_to* --> true
belongs_to* --> true
belongs_to* --> true
belongs_to* --> true
- : bool = true
\end{bluetext}

\begin{bluetext}
# let rec belongs to (e : int) = function
[] -> false
| t :: q -> belongs to e q || (e = t) ; ;
val belongs_to : int -> int list -> bool = <fun> # #trace belongs to ;;
belongs_to is now traced.
# belongs to 3 [3;5;7] ;;
belongs_to <-- 3
belongs_to --> <fun>
belongs_to* <-- [3; 5; 7]
belongs_to <-- 3
belongs_to --> <fun>
belongs_to* <-- [5; 7]
belongs_to <-- 3
belongs_to --> <fun>
belongs_to* <-- [7]
belongs_to <-- 3
belongs_to --> <fun>
belongs_to* <-- []
belongs_to* --> false
belongs_to* --> false
belongs_to* --> false
belongs_to* --> true
- : bool = true  
\end{bluetext}


Trace providing a mechanism for the efficiency analysis of recursive functions, not that friendly, however, no idented output.
To make things worse, trace \textit{does not show the value corresponding to an argument of a parameterized type}. The toploop can show
only monomorphic types.

Moreover, it only keeps the inferred types of
\textit{global declarations}. Therefore after compilation of the
expression, the toplevel in fact \textit{no longer } processes any
furthuer type information about the expression.

Only global type declarations are kept in the environment of the
toplevel loop, \textit{local functions} can not be traced for the same reasons
as above
\begin{bluetext}
let rec belongs_to e = function 
    | [] -> false 
    | t :: q -> (e=t) || belongs_to e q;;
    val belongs_to : 'a -> 'a list -> bool = <fun>
# belongs_to 4 [3;5;7;4];;
- : bool = true
# #trace belongs_to;;
belongs_to is now traced.
# belongs_to 4 [3;5;7;4];;
belongs_to <-- <poly>
belongs_to --> <fun>
belongs_to* <-- [<poly>; <poly>; <poly>; <poly>]
belongs_to <-- <poly>
belongs_to --> <fun>
belongs_to* <-- [<poly>; <poly>; <poly>]
belongs_to <-- <poly>
belongs_to --> <fun>
belongs_to* <-- [<poly>; <poly>]
belongs_to <-- <poly>
belongs_to --> <fun>
belongs_to* <-- [<poly>]
belongs_to* --> true
belongs_to* --> true
belongs_to* --> true
belongs_to* --> true
- : bool = true
\end{bluetext}

\item ocamldbg

  The \textit{-g} option produces a \textit{.cmo} file with the
  debugging information. (bytecode only)
  \end{enumerate}
\end{enumerate}
%%% Local Variables: 
%%% mode: latex
%%% TeX-master: "../master"
%%% End: 





\subsection{Ocaml for scientists}
\label{sec:ocaml-scientists}
\begin{itemize}
\item caveat
  \begin{itemize}
  \item string char
    \verb|'a' = '\097'|
    \verb|"Hello world".[4]|

\begin{alternate}
  [|1;2;3|].(1)
  2 
\end{alternate}

  \item objects

\begin{ocamlcode}

(* it's a type class type *)
class type number = object
  method im:float
  method re:float 
end
\end{ocamlcode}

\begin{ocamlcode}
class complex x y = object 
    val x = x
    val y = y
    method re:float = x
    method im:float = y
end ;;
let b : number = new complex 3. 4.
\end{ocamlcode}

\begin{alternate}
# let b = new complex 3. 4.;;
val b : complex = <obj>
# let b : number = new complex 3. 4.;;
val b : number = <obj>
 \end{alternate}

\begin{ocamlcode} 
# let make_z x y = object
    val x : float = x
    val y : float = y
    method re = x
    method im = y
    end;;
  \end{ocamlcode}
\begin{ocamlcode}  
val make_z : float -> float -> < im : float; re : float > = <fun>
\end{ocamlcode}

class type is kinda interface

  
\begin{ocamlcode}
# let abs_number (z:number) = 
       let sqr x = x *. x in 
       sqrt (sqr z#re +. sqr z#im);;
     \end{ocamlcode}
     
think class as a module 


  \item asr (arith) (**) lsr
  \item elements

\begin{alternate}
  [1;2;3;4] |> Set.of_list |> Set.elements;;
  - : int list = [1; 2; 3; 4]
\end{alternate}


  \end{itemize}
\item convention
\item GMP (GNU library for arbitrary precision arithmetic)

\begin{ocamlcode}
module type INT_RANGE = sig
type t
val make : int -> int -> t
end 
\end{ocamlcode}


\item Hashtbl(create, Make)
  Hahsing is another form of structural comparison and should not be applied
  {\bf to abstract types}
  \emph{Semantically equivalent sets are likely to produce different hashes}
  notice \textit{Map.empty is polymorphic, Hashtbl.empty is monomorphic}
\end{itemize}


%%% Local Variables: 
%%% mode: LaTex
%%% TeX-master: "../master"
%%% End: 


\subsection{caltech ocaml book}


  
polymorphic variants
  \begin{enumerate}
  \item simple example

\begin{alternate}
let string_of_number = function `Integer i -> i;;
val string_of_number : [< `Integer of 'a ] -> 'a = <fun>
\end{alternate}
    
\begin{ocamlcode}  
# let string_of_number = function
    |`Integer i -> i
    |_ -> invalid_arg "string_of_number";;
  \end{ocamlcode}
\begin{ocamlcode}  
  val string_of_number : [> `Integer of 'a ] -> 'a = <fun>
\end{ocamlcode}  

\begin{ocamlcode}
let test0 = function 
  |`Int i -> i

let test1 = function 
  |`Int i -> i 
  | _ -> invalid_arg "invalid arg in test1"

let test2 = function 
  |x -> test0 x

let test3 = function 
  |x -> test1 x

(* let test4 : [> `Real of 'a | `Int of 'a ] -> 'a = function 
   |`Real x -> x *)
   | x -> test0 (x:> [< `Int of 'a])  *)

let test5 = function 
  |`Real x -> x 
  | x -> test1 x 
  
\end{ocamlcode}

\begin{ocamlcode}
val test0 : [< `Int of 'a ] -> 'a = <fun>
val test1 : [> `Int of 'a ] -> 'a = <fun>
val test2 : [< `Int of 'a ] -> 'a = <fun>
val test3 : [> `Int of 'a ] -> 'a = <fun>
val test5 : [> `Int of 'a | `Real of 'a ] -> 'a = <fun>
\end{ocamlcode}

for open union, it's easy to reuse, but \textbf{unsafe},
for closed union, hard to use, since the type checker is
conservative


\begin{alternate}

test1 `Test;;
Exception: Invalid_argument "invalid arg in test1".

test0 `Test;;
Characters 6-11:
  test0 `Test;;
        ^^^^^
Error: This expression has type [> `Test ]
       but an expression was expected of type [< `Int of 'a ]
       The second variant type does not allow tag(s) `Test
\end{alternate}
     






  \item \textbf{define polymorphic variant type }

\begin{alternate}
type number = [> `Integer of int | `Real of float ];;
       ^^^^^^^^^^^^^^^^^^^^^^^^^^^^^^^^^^^^^^^^^^^^^^
Error: A type variable is unbound in this type declaration.
In type [> `Integer of int | `Real of float ] as 'a
the variable 'a is unbound

type 'a number = 'a constraint 'a = [>`Integer of int | `Real of float]

let zero : 'a number = `Zero;;
val zero : [> `Integer of int | `Real of float | `Zero ] number = `Zero


type number = [< `Integer of int | `Real of float ];;
       ^^^^^^^^^^^^^^^^^^^^^^^^^^^^^^^^^^^^^^^^^^^^^^
Error: A type variable is unbound in this type declaration.
In type [< `Integer of int | `Real of float ] as 'a
the variable 'a is unbound
# type number = [ `Integer of int | `Real of float ];;
type number = [ `Integer of int | `Real of float ]


\end{alternate}

  \item \textbf{sub-typing for polymorphic variants}

\begin{ocamlcode}
  [`A] :> [`A | `B]
\end{ocamlcode}  
since you know how to handle A and B, then you know how to handle A

\begin{alternate}
let f x = (x:[`A] :> [`A | `B ]);;
val f : [ `A ] -> [ `A | `B ] = <fun>
\end{alternate}

ocaml does has width and depth subtyping
if t1 :> t1' and t2 :> t2' then (t1,t2) :> (t1',t2')

\begin{alternate}
let f x = (x:[`A] * [`B] :> [`A|`C] * [`B | `D]);; 
val f : [ `A ] * [ `B ] -> [ `A | `C ] * [ `B | `D ] = <fun>


let f x = (x : [ `A | `B ] -> [ `C ] :> [ `A ] -> [ `C | `D ]);;
val f : ([ `A | `B ] -> [ `C ]) -> [ `A ] -> [ `C | `D ] = <fun>
\end{alternate}

  \item variance notation \\
    if you don't write the + and -, ocaml will \textbf{infer} them for you ,
    but when you write abstract type in module type signatures, it makes sense.
    variance annotations \textbf{allow you to expose the subtyping properties} of your type
    in an interface, without exposing the representation.

\begin{ocamlcode}
type (+'a, +'b) t = 'a * 'b
type (-'a,+'b) t = 'a -> 'b 
module M : sig
  type (+'a,'+b) t
end = struct
  type ('a,'b) t = 'a * 'b 
end
\end{ocamlcode}
ocaml did the check when you define it, so you can not define it arbitrarily

  \item \textbf{co-variant} helps polymorphism

\begin{alternate}
module M : sig
    type +'a t
    val embed : 'a -> 'a t
  end = struct
    type 'a t = 'a
    let embed x = x
end ;;
M.embed []  ;;
- : 'a list M.t = <abstr>
\end{alternate}


  \item example

\begin{alternate}
type suit = [ `Club | `Diamond | `Heart | `Spade ]
  
let winner = function `Heart -> true | #suit -> false;;
val winner : [< suit ] -> bool = <fun>
let winner2 = function `Unknown -> true |#suit -> false;;
val winner2 : [< `Club | `Diamond | `Heart | `Spade | `Unknown ] -> bool =
  <fun>

(* the variant tag does not belong to a particular type *)

let winner3 : (suit -> bool) = function `Unknown -> true | #suit -> false;;
                                          ^^^^^^^^
Warning 11: this match case is unused.
val winner3 : suit -> bool = <fun>

\end{alternate}

  \end{enumerate}
%%% Local Variables: 
%%% mode: latex
%%% TeX-master: "../master"
%%% End: 

\subsection{The functional approach to programming}
\label{sec:funct-appr-progr}

%%% Local Variables: 
%%% mode: latex
%%% TeX-master: "../master"
%%% End: 

\subsection{practical ocaml}
\label{sec:practical-ocaml}


\begin{enumerate}
\item chap30 \\

  \begin{bluetext}
external functions_can_be_defined: unit -> unit = "int_c_code"     
  \end{bluetext}
\end{enumerate}

%%% Local Variables: 
%%% mode: latex
%%% TeX-master: "../master"
%%% End: 

\subsection{hol-light}
\label{sec:hol-light}
\begin{itemize}
\item \href{http://code.google.com/p/hol-light/}{hol-light} 
\end{itemize}

%%% Local Variables: 
%%% mode: latex
%%% TeX-master: "../master"
%%% End: 

\section{UNIX system programming in ocaml}
\label{sec:unix-syst-progr}

\subsection{chap1}
\label{sec:chap1}

\begin{enumerate}
\item Modules Sys and Unix \\
  \textbf{Sys} containts those functions common to Unix and Windows.
  \textbf{Unix} contains everything specific to Unix.

  The \textit{Sys} and \textit{Unix} modules can override certain
  functions of the \textit{Pervasives} module
  \begin{alternate}
Unix.stdin;;
- : Batteries.Unix.file_descr = <abstr>
Pervasives.stdin;;
- : in_channel = <abstr>
\end{alternate}

\begin{ocamlcode}
  <prog.{native,byte}> : use_unix
  ocamlmktop -o ocamlunix unix.cma
\end{ocamlcode}

When running a program from a shell, the shell passes \textbf{arguments} and
\textbf{environment} to the program. When a program terminates
prematurely because \textit{an exception was raised but not caught}, it makes
an implicit call to \textit{exit 2}. For \textit{at\_exit}, the last
function to be registered is called first, and it can not be
unregistered. However, we can walk around it using global variables.

\begin{ocamlcode}
  Sys.argv, Sys.getenv , Unix.environment, 
  Pervasives.exit, Pervasives.at_exit, Unix.handle_unix_error
\end{ocamlcode}
\begin{alternate}
Sys.argv;;
\end{alternate}
\begin{ocamlcode}
- : string array =
[|"/Users/bob/SourceCode/ML/godi/bin/ocaml"; "dynlink.cma";
"camlp4of.cma"; "-warn-error"; "+a-4-6-27..29"|]
\end{ocamlcode}
\begin{alternate}
  Unix.environment ();;
\end{alternate}
\begin{ocamlcode}
- : string array =
[|"TERM=dumb"; "SHELL=/bin/bash";
  "TMPDIR=/var/folders/R4/R4awSXDIH6GpuuMmaVeCzU+++TI/-Tmp-/";
  "LIBRARY_PATH=/opt/local/lib/";
  "EMACSDATA=/Applications/Aquamacs.app/Contents/Resources/etc";
  "Apple_PubSub_Socket_Render=/tmp/launch-mcHkKo/Render";
  "EMACSPATH=/Applications/Aquamacs.app/Contents/MacOS/bin";
  "INCLUDE_PATH=/opt/local/include/"; "EMACS=t"; "USER=bob";
  "LD_LIBRARY_PATH=/opt/local/lib/"; "COMMAND_MODE=unix2003"; "TERMCAP=";
  "SSH_AUTH_SOCK=/tmp/launch-g9AcyQ/Listeners";
  "__CF_USER_TEXT_ENCODING=0x1F5:0:0"; "COLUMNS=68";
  "PATH=/opt/local/sbin:/usr/local/smlnj/bin:/usr/local/lib:/Applications/MATLAB_R2010b.app/bin:~/SourceCode/scala/scala-2.9.0.final/bin:/Users/bob/SourceCode/scripts:~/lib/emacs/customize:/usr/local/git/bin:/Users/bob/Racket/bin:/Users/bob/.cabal/bin:/Users/bob/SourceCode/ML/godi/bin:/Users/bob/SourceCode/ML/godi/sbin:/usr/texbin/:/bin:/usr/bin:/opt/local/bin/:/usr/local/lib/:/usr/local/bin/";
  "_=/usr/local/bin/ledit"; "C_INCLUDE_PATH=/opt/local/include/";
  "PWD=/Users/bob/SourceCode/Notes/ocaml-book";
  "TEXINPUTS=.:/Applications/Aquamacs.app/Contents/Resources/lisp/aquamacs/edit-modes/auctex/latex:";
  "EMACSLOADPATH=/Applications/Aquamacs.app/Contents/Resources/lisp:/Applications/Aquamacs.app/Contents/Resources/leim";
  "SHLVL=3"; "HOME=/Users/bob"; "LOGNAME=bob";
  "CAMLP4_EXAMPLE=/Users/bob/SourceCode/ML/godi/build/distfiles/ocaml-3.12.0/camlp4/examples/";
  "DISPLAY=/tmp/launch-sXEeNT/org.x:0"; "INSIDE_EMACS=23.3.50.1,comint";
  "EMACSDOC=/Applications/Aquamacs.app/Contents/Resources/etc";
  "SECURITYSESSIONID=616cd3"|]
\end{ocamlcode}

\item ERROR handling \\
  \begin{bluetext}
    exception Unix_error of error * string * string
    type error = E2BIG | ... |EUNKNOWERR of int 
  \end{bluetext}
  The second arg  of \textit{Unix\_error} is the name of the system
  call that raised the error, the third, if possible, identifies the
  object on which the error occured (i.e. file name).
  \textit{Unix.handle\_unix\_error}, if this raises the exception
  \textit{Unix\_error}, displays the message, and \textit{exit 2}


  \begin{ocamlcode}
let handle_unix_error2 f arg = let open Unix in 
  try
     f arg
  with Unix_error(err, fun_name, arg) ->
  prerr_string Sys.argv.(0);
  prerr_string ": \"";
  prerr_string fun_name;
  prerr_string "\" failed";
  if String.length arg > 0 then begin
     prerr_string " on \"";
     prerr_string arg;
     prerr_string "\"" end;
     prerr_string ": ";
     prerr_endline (error_message err);
     exit 2;;  
   \end{ocamlcode}
   
   \begin{bluetext}
val handle_unix_error2 : ('a -> 'b) -> 'a -> 'b = <fun>     
\end{bluetext}

\begin{bluetext}
  let rec restart_on_EINTR f x =
  try f x with Unix_error (EINTR, _, _) -> restart_on_EINTR f x  
\end{bluetext}

\begin{alternate}
finally;;
- : (unit -> unit) -> ('a -> 'b) -> 'a -> 'b = <fun>
finally (fun _ -> print_endline "finally") (fun _ -> failwith "haha") ();;
\end{alternate}
\begin{ocamlcode}
finally
Exception: Failure "haha".
\end{ocamlcode}

In case the program fails, i.e. raises an exception, \textit{the finalizer is
run and the exception  ex is raised again}. If \textbf{both} the main function
and the finalizer fail, the finalizer's exception is raised.
\end{enumerate}

\subsection{chap2}
\label{sec:chap2}

\begin{enumerate}
\item Files \\
  \textbf{File} covers \textit{standard files, directories, symbolic
    links, special files(devices), named pipes, sockets}
\item \textbf{Filename}  module \\
  makes filename cross platform
  \begin{bluetext}
    val current_dir_name : string
    val parent_dir_name : string
    val dir_sep : string
    val concat : string -> string -> string
    val is_relative : string -> bool
    val is_implicit : string -> bool
    val check_suffix : string -> string -> bool
    val chop_suffix : string -> string -> string
    val chop_extension : string -> string
    val basename : string -> string
    val dirname : string -> string
    val temp_file : ?temp_dir:string -> string -> string -> string
    val open_temp_file :
      ?mode:open_flag list ->
      ?temp_dir:string -> string -> string -> string * out_channel
    val temp_dir_name : string
    val quote : string -> string
  \end{bluetext}

  non-directory files can have \textbf{many parents}(we say that they have many
  \textbf{hard links}). There are also \textit{symbolic links} which
  can be seen as \textit{non-directory} files containing a path, conceptually,
  this path can be obtained by reading the contents of the symbolic
  link like an ordinary file. Whenever a symbolic link occurs in the
  \textbf{middle} of  a path, we have to follow its path
  transparently.
  \begin{bluetext}
    p/s/q -> l/q (l is absolute)
    p/s/q -> p/l/q (l is relative)
  \end{bluetext}
  \begin{bluetext}
    Sys.getcwd, Sys.chdir, Unix.chroot
  \end{bluetext}
  \textit{Unix.chroot p} makes the node p, which should be a
  directory, the root of the \textit{restricted} view of the
  hierarchy. Absolute paths are then interpreted according to this new
  root p (and .. at the new root is itself).
  Due to hard links, a file can have many different names.

\begin{ocamlcode}
Unix.(link, unlink,symlink,rename);;
\end{ocamlcode}
\begin{ocamlcode}
- : (string -> string -> unit) * (string -> unit) *
    (string -> string -> unit) * (string -> string -> unit)    
  \end{ocamlcode}
  
  \textit{unlink f} is like \textit{rm -f f}, \textit{link f1 f2} is
  like \textit{ln f1 f2}, \textit{symlink f1 f2} is like \textit{ln -s
  f1 f2}, rename f1 f2 is like \textit{mv f1 f2}

  A file descriptor represents a pointer to a file along with other
  information like the current read/write position in the file, the
  access rights, etc. \textbf{file\_descr}

  \begin{ocamlcode}
    Unix.(stdin,stdout,stderr);;
  \end{ocamlcode}
  
  \begin{ocamlcode}
  - : Batteries.Unix.file_descr * Batteries.Unix.file_descr *
    Batteries.Unix.file_descr    
  \end{ocamlcode}
  without redirections, the three descriptors refer to the terminal.
  \begin{bluetext}
    cmd > f ; cmd 2 > f
  \end{bluetext}
\item Meta attributes, types and permissions \\


  \begin{alternate}
Unix.(stat,lstat,fstat);;
  \end{alternate}
\begin{ocamlcode}  
  (string -> Batteries.Unix.stats) *
  (string -> Batteries.Unix.stats) *
  (Batteries.Unix.file_descr -> Batteries.Unix.stats)    
\end{ocamlcode}
  \textit{lstat} returns information about the symbolic link itself,
  while \textit{stat} returns information about the file that link
  points to.
  \begin{alternate}
Unix.(lstat &&& stat) "/usr/bin/al";;    
  \end{alternate}
  \begin{ocamlcode}
({Batteries.Unix.st_dev = 234881026; Batteries.Unix.st_ino = 843893;
  Batteries.Unix.st_kind = Batteries.Unix.S_LNK; (* link *)
  Batteries.Unix.st_perm = 493; Batteries.Unix.st_nlink = 1;
  Batteries.Unix.st_uid = 0; Batteries.Unix.st_gid = 0;
  Batteries.Unix.st_rdev = 0; Batteries.Unix.st_size = 46;
  (* pretty  small as a link *)
  Batteries.Unix.st_atime = 1273804908.;
  Batteries.Unix.st_mtime = 1273804908.;
  Batteries.Unix.st_ctime = 1273804908.},

 {Batteries.Unix.st_dev = 234881026; Batteries.Unix.st_ino = 840746;
  Batteries.Unix.st_kind = Batteries.Unix.S_REG; (*  regular file *)
  Batteries.Unix.st_perm = 493; Batteries.Unix.st_nlink = 1;
  Batteries.Unix.st_uid = 0; Batteries.Unix.st_gid = 80;
  Batteries.Unix.st_rdev = 0; Batteries.Unix.st_size = 163;
  (* maybe bigger *)
  Batteries.Unix.st_atime = 1323997427.;
  Batteries.Unix.st_mtime = 1271968805.;
  Batteries.Unix.st_ctime = 1273804911.})    
\end{ocamlcode}

  A file is uniquely identified by the pair made of its device
  number(typically the disk partition where it is located)
  \textit{st\_dev} and its inode number \textit{st\_ino}

  All the users and groups on the machine are usually described in the
  \textit{/etc/passwd, /etc/groups} files.
  \begin{bluetext}
    st_uid
    st_gid
    getpwnam, getgrnam, (by name, get passwd_entry, group_entry)
    getpwuid, getgrgid (by id)
    getlogin, getgroups
    chown, fchown
  \end{bluetext}

  \begin{ocamlcode}
Unix.getlogin () |> Unix.getpwnam;;    
\end{ocamlcode}

\begin{ocamlcode}
{Batteries.Unix.pw_name = "bob"; Batteries.Unix.pw_passwd = "********";
 Batteries.Unix.pw_uid = 501; Batteries.Unix.pw_gid = 20;
 Batteries.Unix.pw_gecos = "bobzhang"; Batteries.Unix.pw_dir = "/Users/bob";
 Batteries.Unix.pw_shell = "/bin/bash"}

\end{ocamlcode}

for access rights, executable, writable, readable by the user owner,
group owner, other users. For a directory, the executable permission
means the right to enter it, and read permission the right to list its
contents. The special bits do not have meaning unless the \textbf{x}
bit is set. The bit \textit{t} allows sub-directories to inherit the
permissions of the parent directory. On a directory, the bit
\textit{s} allows the use of the directory's \textit{uid} or
\textit{gid} rather than the user's to create directories. For an
executable file, the bit \textit{s} allows the chaning at executation
time of the user's effective identity or group with the system calls
\textit{setuid} and \textit{setgid}

\begin{alternate}
Unix.(setuid, getuid);;
- : (int -> unit) * (unit -> int) = (<fun>, <fun>)  
\end{alternate}

\item operations on directries \\
  only the kernel can write in directories(when files are
  created). Opening a directory in write mode is \textit{prohibited}.

  \begin{alternate}
Unix.(opendir,readdir,rewinddir,closedir);;    
\end{alternate}

\begin{ocamlcode}
- : (string -> Batteries.Unix.dir_handle) *
    (Batteries.Unix.dir_handle -> string) *
    (Batteries.Unix.dir_handle -> unit) * (Batteries.Unix.dir_handle -> unit)  
  \end{ocamlcode}

  \textit{rewinddir} repositions the descriptor at the \textbf{beginning} of
  the directory.

  \begin{ocamlcode}
    mkdir, rmdir
  \end{ocamlcode}
  We can only remove a directory that is \textbf{already empty}. It is
  thus necessary to first recursively empty the contents of the
  directory and then remove the directory.

  \begin{ocamlcode}
exception Hidden of exn 
(** add a tag to exn *)
let hide_exn f x = try f x with exn -> raise (Hidden exn)
(** strip the tag of exn *)
let reveal_exn f x = try f x with Hidden exn -> raise exn 
\end{ocamlcode}
\item  File manipulation \\

  \begin{alternate}
Unix.openfile;;    
\end{alternate}

\begin{ocamlcode}
- : string ->
    Batteries.Unix.open_flag list ->
    Batteries.Unix.file_perm -> Batteries.Unix.file_descr
  \end{ocamlcode}
  Most programs use \textit{0o666} means \textit{rw-rw-rw-}. with the default
  creation mask of \textit{0o022}, the file is thus created with the permission
  \textit{rw-r--r--}. With a more lenient mask of 0o002, the file is
  created with the permissions \textit{rw-rw-r--}. The third argument
  can be anything as \textit{O\_CREATE} is not specified.
  And to write to an empty file without caring any previous content,
  we use
  \begin{ocamlcode}
    Unix.openfile filename [O_WRONLY; O_TRUNC; O_CREAT] 0o666
  \end{ocamlcode}
  If the file is scripts, we create it with execution permission:
  \begin{ocamlcode}
    Unix.openfile filename [O_WRONLY; O_TRUNC; O_CREAT] 0o777
  \end{ocamlcode}
  If we want it to be confidential,
  \begin{ocamlcode}
    Unix.openfile filename [O_WRONLY; O_TRUNC; O_CREAT] 0o600
  \end{ocamlcode}
  The \textit{O\_NONBLOCK} flag guarantees that if the file is a named pipe or a
  special file then the file opening and subsequent reads and writes
  wil be non-blocking. The \textit{O\_NOCTYY} flag guarantees that if
  the file is a control terminal, it won't become the controlling
  terminal of the calling process.

  \begin{alternate}
    Unix.(read,single_write);;
  \end{alternate}
  \begin{ocamlcode}
  - : (Batteries.Unix.file_descr -> string -> int -> int -> int) *
    (Batteries.Unix.file_descr -> string -> int -> int -> int)
  \end{ocamlcode}    
  The \textit{string} hold the read bytes or the bytes to write. The 3rd
  argument is the start, the forth is the number.

  For writes, the number of bytes actually written is usually the
  number of bytes requested, with two exceptions
  (i) not possible to write (i.e. disk is full) (ii) the descript is a
  pipe or a socket open in non-blocking mode(async) (iii) due to
  OCaml, too large.

  The reason for (iii) is that internally OCaml uses auxiliary buffer
  whose size is bounded by a maximal value.

  OCaml also provides \textit{Unix.write} which iterates the writes
  until all the data is written or an error occurs. The problem is
  that in case of error there's no way to know the number of bytes
  that were \textit{actually written}. \textit{single\_write}
  preserves the atomicity of writes.

  For reads, when the current position is at the end of file, read
  returns zero. The convention \textit{zero equals end of file} also
  holds for special files, \textit{i.e. pipes and sockets}. For
  example, read on a terminal returns zero if we issue a
  \textit{Ctrl-D} on the input.

  But you may consider the blocking-mode in case.

  \begin{ocamlcode}
    Unix.close : file_descr -> unit 
  \end{ocamlcode}
  In contrast to Pervasives' channels, a file descriptor does not need
  to be closed to ensure that all pending writes have been performed
  as write requests are \textit{immediately} transmitted to the
  kernel. On the other hand, the number of descriptors allocated by a
  process is limited by the kernel(several hundreds to thousands).


  \begin{ocamlcode}
let buffer_size = 8192 
let buffer = String.create buffer_size 

(** this is unsatisfactory, if we copy an executable file, we would
like the copy to be also executable. *)
let file_copy input output = Unix.(
  let fd_in = openfile input [O_RDONLY] 0 in 
  let fd_out = openfile output [O_WRONLY; O_CREAT; O_TRUNC] 0o666 in 
  let rec copy_loop () = match read fd_in buffer 0 buffer_size with 
    |0 -> ()
    |r -> write fd_out buffer 0 r |> ignore; copy_loop () in 
  copy_loop ();
  close fd_in ; 
  close fd_out 
)


let copy () = 
  if Array.length Sys.argv = 3 then begin 
    file_copy Sys.argv.(1) Sys.argv.(2)
  end 
  else begin 
    prerr_endline 
      ("Usage: " ^ Sys.argv.(0) ^ "<input_file> <output_file>"); 
    exit 1 
  end 

let _  = Unix.handle_unix_error copy () 
\end{ocamlcode}

\begin{bluetext}
ocamlbuild find.byte -- find.ml find.xxxx    
\end{bluetext}

\begin{alternate}
ocamlbuild find.byte -- find.mlx find.xxxx
_build/find.byte: "open" failed on "find.mlx": No such file or directory
\end{alternate}
\item  system call \\
  For a system call, even if it does very little work, cost dearly --
  much more than a normal function call. So we need buffer to reduce
  the number of system call. For ocaml, the \textit{Pervasives} module
  adds another layer \textit{in\_channel, out\_channel}.

\item positioning and operations specific to certain file types

  \begin{alternate}
Unix.lseek;;
- : Batteries.Unix.file_descr -> int -> Batteries.Unix.seek_command -> int =
\end{alternate}

  File descriptors provide a uniform and media-independent interface
  for data communicatioin. However this uniformity breaks when we need
  to access all the features provided by a given media.

  For normal files, specific API
  \begin{ocamlcode}
Unix.(truncate,ftruncate);;
- : (string -> int -> unit) * (Batteries.Unix.file_descr -> int -> unit) =
\end{ocamlcode}
For symbolic links
\begin{ocamlcode}
Unix.(symlink, readlink);;
- : (string -> string -> unit) * (string -> string) = (<fun>, <fun>)  
\end{ocamlcode}

special files
\begin{enumerate}
\item /dev/null  black hole. (useful for ignoring the result)
\item /dev/tty* control terminals
\item /dev/pty* pseudo-terminals
\item /dev/hd* disks
\item /proc Under linux, system parameters organized as a file system.
\end{enumerate}

many special files ignore \textit{lseek}
\item terminals \\

  \begin{alternate}
Unix.(tcgetattr, tcsetattr);;
\end{alternate}
\begin{ocamlcode}
(Batteries.Unix.file_descr -> Batteries.Unix.terminal_io) *
(Batteries.Unix.file_descr ->
     Batteries.Unix.setattr_when -> Batteries.Unix.terminal_io -> unit)
\end{ocamlcode}
  
  \begin{alternate}
Unix.(tcgetattr stdout);;    
\end{alternate}

\begin{ocamlcode}
{Batteries.Unix.c_ignbrk = false; Batteries.Unix.c_brkint = true;
 Batteries.Unix.c_ignpar = false; Batteries.Unix.c_parmrk = false;
 Batteries.Unix.c_inpck = false; Batteries.Unix.c_istrip = false;
 Batteries.Unix.c_inlcr = false; Batteries.Unix.c_igncr = false;
 Batteries.Unix.c_icrnl = true; Batteries.Unix.c_ixon = false;
 Batteries.Unix.c_ixoff = false; Batteries.Unix.c_opost = true;
 Batteries.Unix.c_obaud = 9600; Batteries.Unix.c_ibaud = 9600;
 Batteries.Unix.c_csize = 8; Batteries.Unix.c_cstopb = 1;
 Batteries.Unix.c_cread = true; Batteries.Unix.c_parenb = false;
 Batteries.Unix.c_parodd = false; Batteries.Unix.c_hupcl = true;
 Batteries.Unix.c_clocal = false; Batteries.Unix.c_isig = false;
 Batteries.Unix.c_icanon = false; Batteries.Unix.c_noflsh = false;
 Batteries.Unix.c_echo = false; Batteries.Unix.c_echoe = true;
 Batteries.Unix.c_echok = false; Batteries.Unix.c_echonl = false;
 Batteries.Unix.c_vintr = '\003'; Batteries.Unix.c_vquit = '\028';
 Batteries.Unix.c_verase = '\255'; Batteries.Unix.c_vkill = '\255';
 Batteries.Unix.c_veof = '\004'; Batteries.Unix.c_veol = '\255';
 Batteries.Unix.c_vmin = 1; Batteries.Unix.c_vtime = 0;
 Batteries.Unix.c_vstart = '\017'; Batteries.Unix.c_vstop = '\019'}  
\end{ocamlcode}

it seems that ledit will change your input, and you can not get
\textit{Unix.(tcgetattr stdin)} work.

The code below works in real terminal, but does not work in
pseudo-terminals(like Emacs )

\begin{ocamlcode}
let read_passwd message = Unix.(
match 
   try 
    let default = tcgetattr stdin in 
    let silent = {default with c_echo = false; c_echoe = false ; 
                  c_echok = false; c_echonl = false ; } in 
     Some (default, silent)
   with _ -> None 
with
 |None -> Legacy.input_line Pervasives.stdin 
 |Some (default, silent) -> 
   print_string message ; 
   Legacy.flush Pervasives.stdout ; 
   tcsetattr stdin TCSANOW silent ; 
   try 
     let s = Legacy.input_line Pervasives.stdin in 
     tcsetattr stdin TCSANOW default; s 
   with x ->      tcsetattr stdin TCSANOW default; raise x 
    
);;
\end{ocamlcode}
 Sometimes a program needs to start another and connect its standard
 input to a terminal (or pseudo-terminal). To achieve that, we must
 manually look among the pseudo-terminals(/dev/tty[a-z][a-f0-9]) and
 find one that is not already open. We can open this file and start
 the program with this file on its standard input.

 The function \textit{tcsendbreak} sends an interrupt to the
 peripheral. The second argument is the duration of the interrupt.


 \begin{bluetext}
   tcdrain, tcflush, tcflow, setsid
 \end{bluetext}

\item locks on files
  \begin{bluetext}
Unix.lockf;;
- : Batteries.Unix.file_descr -> Batteries.Unix.lock_command -> int -> unit =    
\end{bluetext}

ocaml-expect
\begin{alternate}
let p = X.spawn "ocaml" [||];;
val p : X.t = <abstr>
X.expect p ~fmatches:[(fun s -> Some s)] [] "";;
- : string = "        Objective Caml version 3.12.1"
X.send p "3;;\n";;
- : unit = ()
X.expect p ~fmatches:[(fun s -> Some s)] [] "";;
- : string = "- : int = 3"  
\end{alternate}

not very powerful
\end{enumerate}

\subsection{chap3}
\label{sec:chap3}



%%% Local Variables: 
%%% mode: latex
%%% TeX-master: "../master"
%%% End: 

\subsection{practical ocaml}
\label{sec:practical-ocaml}


\begin{enumerate}
\item chap30 \\

  \begin{bluetext}
external functions_can_be_defined: unit -> unit = "int_c_code"     
  \end{bluetext}
\end{enumerate}

%%% Local Variables: 
%%% mode: latex
%%% TeX-master: "../master"
%%% End: 

\subsection{tricks}
\label{sec:tricks}

\begin{itemize}
\item ocamlobjinfo \\
  analyzing ocaml obj info

\begin{Verbatim}[formatcom=\color{blue},fontsize=\scriptsize]
ocamlobjinfo ./_build/src/batEnum.cmo
File ./_build/src/batEnum.cmo
Unit name: BatEnum
Interfaces imported:
	720848e0b508273805ef38d884a57618	Array
	c91c0bbb9f7670b10cdc0f2dcc57c5f9	Int32
	42fecddd710bb96856120e550f33050d	BatEnum
	d1bb48f7b061c10756e8a5823ef6d2eb	BatInterfaces
	81da2f450287aeff11718936b0cb4546	BatValue_printer
	6fdd8205a679c3020487ba2f941930bb	BatInnerIO
	40bf652f22a33a7cfa05ee1dd5e0d7e4	Buffer
	c02313bdd8cc849d89fa24b024366726	BatConcurrent
	3dee29b414dd26a1cfca3bbdf20e7dfc	Char
	db723a1798b122e08919a2bfed062514	Pervasives
	227fb38c6dfc5c0f1b050ee46651eebe	CamlinternalLazy
	9c85fb419d52a8fd876c84784374e0cf	List
	79fd3a55345b718296e878c0e7bed10e	Queue
	9cf8941f15489d84ebd11297f6b92182	CamlinternalOO
	b64305dcc933950725d3137468a0e434	ArrayLabels
	64339e3c28b4a17a8ec728e5f20a3cf6	BatRef
	3aeb33d11433c95bb62053c65665eb76	Obj
	3b0ed254d84078b0f21da765b10741e3	BatMonad
	aaa46201460de222b812caf2f6636244	Lazy
Uses unsafe features: YES
Primitives declared in this module:

ocamlobjinfo /Users/bob/SourceCode/ML/godi/lib/ocaml/std-lib/camlp4/camlp4lib.cma |grep Unit
Unit name: Camlp4_import
Unit name: Camlp4_config
Unit name: Camlp4
\end{Verbatim}
  obj has many Units, each Unit itself also import some
  interfaces. ideas: you can parse the result to get an dependent graph.
\item operator associativity \\
  the \textbf{first} char decides
  @ $\rightarrow$ right ;  \verb|^| $\rightarrow$ right

\begin{alternate}
# let (^|) a b = a - b;;
val ( ^| ) : int -> int -> int = <fun>
# 3 ^| 2 ^| 1;;
- : int = 2
\end{alternate}

\item literals

\begin{bluecode}
30l => int32
30L => int64
30n => nativeint
\end{bluecode}


\item \verb|{re ;_}| some labels were intentionally omitted \\
  this is a new feature in recent ocaml, it will emit an warning
  otherwise 

\item Emacs \\
  there are some many tricks I can only enum a few 
  \begin{itemize}
  \item capture the shell command
    \textit{C-u M-!} to capture the shell-command
    \textit{M-|} shell-command-on-region

  \end{itemize}
\item \textbf{dirty} compiling

\begin{alternate}
# let ic = Unix.open_process_in "ocamlc test.ml 2>&1";;
val ic : in_channel = <abstr>
# input_line ic;;
- : string = "File \"test.ml\", line 1, characters 0-1:"
# input_line ic;;
- : string = "Error: I/O error: test.ml: No such file or directory"
# input_line ic;;
Exception: End_of_file.
\end{alternate}


\item toplevellib.cma (toplevel/toploop.mli)
\item memory profiling \\
You can override a little ocaml-benchmark to measure the allocation rate
of the GC. This gives you a pretty good understanding on the fact you
are allocating too much or not.

\begin{redcode}
(** Benchmark extension   @author Sylvain Le Gall
 *)

open Benchmark;;
type t =
   {
     benchmark: Benchmark.t;
     memory_used: float;
   }
;;

let gc_wrap f x =
 (* Extend sample to add GC stat *)
 let add_gc_stat memory_used samples =
   List.map 
     (fun (name, lst) ->
        name,
        List.map 
          (fun bt -> 
             { 
               benchmark = bt; 
               memory_used = memory_used;
             }
          )
          lst
     )
     samples
 in
(* Call throughput1 and add GC stat *)
 let () = 
   print_string "Cleaning memory before benchmark"; print_newline ();    
   Gc.full_major ()
 in
 let allocated_before = 
   Gc.allocated_bytes ()
 in
 let samples =
   f x
 in
 let () = 
   print_string "Cleaning memory after benchmark"; print_newline ();
   Gc.full_major ()
 in
 let memory_used = 
   ((Gc.allocated_bytes ()) -. allocated_before) 
 in
   add_gc_stat memory_used samples
;;

let throughput1
     ?min_count ?style
     ?fwidth    ?fdigits
     ?repeat    ?name
     seconds 
     f x =

 (* Benchmark throughput1 as it should be called *) 
 gc_wrap 
   (throughput1
      ?min_count ?style
      ?fwidth    ?fdigits
      ?repeat    ?name
      seconds f) x
;;

let throughputN 
     ?min_count ?style
     ?fwidth    ?fdigits
     ?repeat    
     seconds name_f_args =
 List.flatten
   (List.map
      (fun (name, f, args) ->
        throughput1 
          ?min_count ?style
          ?fwidth    ?fdigits
          ?repeat    ~name:name
          seconds f args)
      name_f_args)
;;
let latency1 
     ?min_cpu ?style 
     ?fwidth  ?fdigits 
     ?repeat  n 
     ?name    f x =
 gc_wrap 
   (latency1
     ?min_cpu ?style 
     ?fwidth  ?fdigits 
     ?repeat  n 
     ?name    f) x
;;

let latencyN 
     ?min_cpu ?style 
     ?fwidth  ?fdigits 
     ?repeat  
     n name_f_args =
 List.flatten
   (List.map
      (fun (name, f, args) ->
        latency1 
          ?min_cpu   ?style
          ?fwidth    ?fdigits
          ?repeat    ~name:name
          n          f args)
      name_f_args)
;;
\end{redcode}

\end{itemize}



%%% Local Variables: 
%%% mode: latex
%%% TeX-master: "master"
%%% End: 


\subsection{ocaml blogs}
\label{sec:ocaml-blogs}
\href{http://ygrek.org.ua/p/ocaml.html}{ygrek} \\
\href{http://elehack.net/michael/blog/}{michal} \\
\href{http://eigenclass.org/R2/}{eigenclass} \\
\href{http://syntaxexclamation.wordpress.com/}{syntax} \\
\href{http://martin.jambon.free.fr/ocaml.html}{jambon} \\
\href{http://www.x9c.fr/}{Xavier Clerc} \\
\href{http://www.pps.jussieu.fr/~li/}{Zheng li} \\
\href{http://pauillac.inria.fr/~xleroy/teaching.html}{xleroy/teaching} \\
\href{http://alaska-kamtchatka.blogspot.com/}{alaska} \\
\href{http://erratique.ch/software/}{erratique} \\
\href{http://dutherenverseauborddelatable.wordpress.com/category/informatique-computer-science/ocaml/}{duther} \\
\href{http://www.univ-orleans.fr/lifo/Members/David.Teller/opensource.html}{David Teller} \\
\href{http://www.cl.cam.ac.uk/teaching/Lectures/funprog-jrh-1996/index.html}{john harisson} \\
\href{http://www.cl.cam.ac.uk/~mjcg/Teaching/FuncProg/FuncProg.html}{Mike Gordon} \\
\href{http://www.cs.hmc.edu/~keller/cs60book/}{Robert Keller} \\
\href{http://alexott.net/en/index.html}{alexott} \\
\href{http://padator.org/INDEX.php}{Yoann Padioleau}


%%% Local Variables:
%%% mode: LaTex 
%%% TeX-master: "master"
%%% End: 



\end{document}


%%% Local Variables: 
%%% mode: latex
%%% TeX-master: "master"
%%% End: 
